%% Generated by Sphinx.
\def\sphinxdocclass{report}
\documentclass[letterpaper,10pt,english]{sphinxmanual}
\ifdefined\pdfpxdimen
   \let\sphinxpxdimen\pdfpxdimen\else\newdimen\sphinxpxdimen
\fi \sphinxpxdimen=.75bp\relax



\usepackage{cmap}
\usepackage[T1]{fontenc}
\usepackage{amsmath,amssymb,amstext}
\usepackage{babel}
\usepackage{times}
\usepackage[Bjarne]{fncychap}
\usepackage[dontkeepoldnames]{sphinx}

\usepackage{geometry}

% Include hyperref last.
\usepackage{hyperref}
% Fix anchor placement for figures with captions.
\usepackage{hypcap}% it must be loaded after hyperref.
% Set up styles of URL: it should be placed after hyperref.
\urlstyle{same}

\addto\captionsenglish{\renewcommand{\figurename}{Figure}}
\addto\captionsenglish{\renewcommand{\tablename}{Table}}
\addto\captionsenglish{\renewcommand{\literalblockname}{Listing}}

\addto\captionsenglish{\renewcommand{\literalblockcontinuedname}{continued from previous page}}
\addto\captionsenglish{\renewcommand{\literalblockcontinuesname}{continues on next page}}

\addto\extrasenglish{\def\pageautorefname{page}}

\setcounter{tocdepth}{0}



\title{F311 Documentation}
\date{Nov 29, 2017}
\release{17.9.27.0}
\author{Julio Trevisan}
\newcommand{\sphinxlogo}{\vbox{}}
\renewcommand{\releasename}{Release}
\makeindex

\begin{document}

\maketitle
\sphinxtableofcontents
\phantomsection\label{\detokenize{index::doc}}


Welcome!


\chapter{Introduction}
\label{\detokenize{intro:introduction}}\label{\detokenize{intro::doc}}\label{\detokenize{intro:f311-astronomy-related-python-3-api-and-scripts}}
\sphinxcode{f311} is a Python 3 package containing many resources on selected topics in Astronomy.
Once installed, the package makes available a collection of scripts and an API
(application programming interface).


\section{Using the API}
\label{\detokenize{intro:using-the-api}}
The API is organized in sub-packages, which can be imported as follows:

\begin{sphinxVerbatim}[commandchars=\\\{\}]
\PYG{k+kn}{import} \PYG{n+nn}{f311}\PYG{n+nn}{.}\PYG{n+nn}{pyfant} \PYG{k}{as} \PYG{n+nn}{pf}
\PYG{k+kn}{import} \PYG{n+nn}{f311}\PYG{n+nn}{.}\PYG{n+nn}{convmol} \PYG{k}{as} \PYG{n+nn}{cm}
\PYG{k+kn}{import} \PYG{n+nn}{f311}\PYG{n+nn}{.}\PYG{n+nn}{explorer} \PYG{k}{as} \PYG{n+nn}{ex}
\PYG{k+kn}{import} \PYG{n+nn}{f311}\PYG{n+nn}{.}\PYG{n+nn}{filetypes} \PYG{k}{as} \PYG{n+nn}{ft}
\PYG{k+kn}{import} \PYG{n+nn}{f311}\PYG{n+nn}{.}\PYG{n+nn}{physics} \PYG{k}{as} \PYG{n+nn}{ph}
\PYG{k+kn}{import} \PYG{n+nn}{f311}\PYG{n+nn}{.}\PYG{n+nn}{aosss} \PYG{k}{as} \PYG{n+nn}{ao}
\end{sphinxVerbatim}

For convenience, the symbols of the subpackages are exposed at the root package level, so
another way to import the API is:

\begin{sphinxVerbatim}[commandchars=\\\{\}]
\PYG{k+kn}{import} \PYG{n+nn}{f311}
\end{sphinxVerbatim}


\section{Contributing to this project}
\label{\detokenize{intro:contributing-to-this-project}}
If you would like to contribute to this project, you can clone the source code on
\sphinxhref{https://github.com/trevisanj/f311}{GitHub}.


\section{List of scripts}
\label{\detokenize{intro:list-of-scripts}}

\subsection{f311.aosss \textendash{} “Adaptive Optics Systems Simulation Support”}
\label{\detokenize{intro:f311-aosss-adaptive-optics-systems-simulation-support}}

\subsubsection{Graphical applications}
\label{\detokenize{intro:graphical-applications}}\begin{itemize}
\item {} 
{\hyperref[\detokenize{autoscripts/script-wavelength-chart::doc}]{\sphinxcrossref{\DUrole{doc}{wavelength-chart.py}}}}: Draws chart showing spectral lines of interest, spectrograph wavelength ranges, ESO atmospheric .

\end{itemize}


\subsubsection{Command-line tools}
\label{\detokenize{intro:command-line-tools}}\begin{itemize}
\item {} 
{\hyperref[\detokenize{autoscripts/script-create-simulation-reports::doc}]{\sphinxcrossref{\DUrole{doc}{create-simulation-reports.py}}}}: Creates HTML reports from WebSim-COMPASS output files

\item {} 
{\hyperref[\detokenize{autoscripts/script-create-spectrum-lists::doc}]{\sphinxcrossref{\DUrole{doc}{create-spectrum-lists.py}}}}: Create several .splist (spectrum list) files from WebSim-COMPASS output files; groups r

\item {} 
{\hyperref[\detokenize{autoscripts/script-get-compass::doc}]{\sphinxcrossref{\DUrole{doc}{get-compass.py}}}}: Downloads WebSim-COMPASS simulations

\item {} 
{\hyperref[\detokenize{autoscripts/script-list-mosaic-modes::doc}]{\sphinxcrossref{\DUrole{doc}{list-mosaic-modes.py}}}}: Lists MOSAIC Spectrograph modes

\item {} 
{\hyperref[\detokenize{autoscripts/script-organize-directory::doc}]{\sphinxcrossref{\DUrole{doc}{organize-directory.py}}}}: Organizes simulation directory (creates folders, moves files, creates ‘index.html’)

\end{itemize}


\subsection{f311.convmol \textendash{} Conversion of molecular lines files.}
\label{\detokenize{intro:f311-convmol-conversion-of-molecular-lines-files}}

\subsubsection{Graphical applications}
\label{\detokenize{intro:id1}}\begin{itemize}
\item {} 
{\hyperref[\detokenize{autoscripts/script-convmol::doc}]{\sphinxcrossref{\DUrole{doc}{convmol.py}}}}: Conversion of molecular lines data to PFANT format

\item {} 
{\hyperref[\detokenize{autoscripts/script-mced::doc}]{\sphinxcrossref{\DUrole{doc}{mced.py}}}}: Editor for molecular constants file

\item {} 
{\hyperref[\detokenize{autoscripts/script-moldbed::doc}]{\sphinxcrossref{\DUrole{doc}{moldbed.py}}}}: Editor for molecules SQLite database

\end{itemize}


\subsubsection{Command-line tools}
\label{\detokenize{intro:id2}}\begin{itemize}
\item {} 
{\hyperref[\detokenize{autoscripts/script-hitran-scraper::doc}]{\sphinxcrossref{\DUrole{doc}{hitran-scraper.py}}}}: Retrieves molecular lines from the HITRAN database {[}Gordon2016{]}

\item {} 
{\hyperref[\detokenize{autoscripts/script-nist-scraper::doc}]{\sphinxcrossref{\DUrole{doc}{nist-scraper.py}}}}: Retrieves and prints a table of molecular constants from the NIST Chemistry Web Book.

\end{itemize}


\subsection{f311.explorer \textendash{} Object-oriented framework to handle file types:}
\label{\detokenize{intro:f311-explorer-object-oriented-framework-to-handle-file-types}}

\subsubsection{Graphical applications}
\label{\detokenize{intro:id3}}\begin{itemize}
\item {} 
{\hyperref[\detokenize{autoscripts/script-abed::doc}]{\sphinxcrossref{\DUrole{doc}{abed.py}}}}: Abundances file editor

\item {} 
{\hyperref[\detokenize{autoscripts/script-ated::doc}]{\sphinxcrossref{\DUrole{doc}{ated.py}}}}: Atomic lines file editor

\item {} 
{\hyperref[\detokenize{autoscripts/script-cubeed::doc}]{\sphinxcrossref{\DUrole{doc}{cubeed.py}}}}: Data Cube Editor, import/export WebSim-COMPASS data cubes

\item {} 
{\hyperref[\detokenize{autoscripts/script-explorer::doc}]{\sphinxcrossref{\DUrole{doc}{explorer.py}}}}: F311 Explorer \textendash{}  list, visualize, and edit data files (\_à {\color{red}\bfseries{}la\_} File Manager)

\item {} 
{\hyperref[\detokenize{autoscripts/script-mained::doc}]{\sphinxcrossref{\DUrole{doc}{mained.py}}}}: Main configuration file editor.

\item {} 
{\hyperref[\detokenize{autoscripts/script-mled::doc}]{\sphinxcrossref{\DUrole{doc}{mled.py}}}}: Molecular lines file editor.

\item {} 
{\hyperref[\detokenize{autoscripts/script-optionsed::doc}]{\sphinxcrossref{\DUrole{doc}{optionsed.py}}}}: PFANT command-line options file editor.

\item {} 
{\hyperref[\detokenize{autoscripts/script-splisted::doc}]{\sphinxcrossref{\DUrole{doc}{splisted.py}}}}: Spectrum List Editor

\item {} 
{\hyperref[\detokenize{autoscripts/script-tune-zinf::doc}]{\sphinxcrossref{\DUrole{doc}{tune-zinf.py}}}}: Tunes the “zinf” parameter for each atomic line in atomic lines file

\end{itemize}


\subsubsection{Command-line tools}
\label{\detokenize{intro:id4}}\begin{itemize}
\item {} 
{\hyperref[\detokenize{autoscripts/script-create-grid::doc}]{\sphinxcrossref{\DUrole{doc}{create-grid.py}}}}: Merges several atmospheric models into a single file (\_i.e.\_, the “grid”)

\item {} 
{\hyperref[\detokenize{autoscripts/script-cut-atoms::doc}]{\sphinxcrossref{\DUrole{doc}{cut-atoms.py}}}}: Cuts atomic lines file to wavelength interval specified

\item {} 
{\hyperref[\detokenize{autoscripts/script-cut-molecules::doc}]{\sphinxcrossref{\DUrole{doc}{cut-molecules.py}}}}: Cuts molecular lines file to wavelength interval specified

\item {} 
{\hyperref[\detokenize{autoscripts/script-cut-spectrum::doc}]{\sphinxcrossref{\DUrole{doc}{cut-spectrum.py}}}}: Cuts spectrum file to wavelength interval specified

\item {} 
{\hyperref[\detokenize{autoscripts/script-plot-spectra::doc}]{\sphinxcrossref{\DUrole{doc}{plot-spectra.py}}}}: Plots spectra on screen or creates PDF file

\item {} 
{\hyperref[\detokenize{autoscripts/script-vald3-to-atoms::doc}]{\sphinxcrossref{\DUrole{doc}{vald3-to-atoms.py}}}}: Converts VALD3 atomic/molecular lines file to PFANT atomic lines file.

\end{itemize}


\subsection{f311.pyfant \textendash{} Python interface to the PFANT spectral synthesis software (Fortran)}
\label{\detokenize{intro:f311-pyfant-python-interface-to-the-pfant-spectral-synthesis-software-fortran}}

\subsubsection{Graphical applications}
\label{\detokenize{intro:id5}}\begin{itemize}
\item {} 
{\hyperref[\detokenize{autoscripts/script-x::doc}]{\sphinxcrossref{\DUrole{doc}{x.py}}}}: PFANT Launcher \textendash{} Graphical Interface for Spectral Synthesis

\end{itemize}


\subsubsection{Command-line tools}
\label{\detokenize{intro:id6}}\begin{itemize}
\item {} 
{\hyperref[\detokenize{autoscripts/script-copy-star::doc}]{\sphinxcrossref{\DUrole{doc}{copy-star.py}}}}: Copies stellar data files (such as main.dat, abonds.dat, dissoc.dat) to local directory

\item {} 
{\hyperref[\detokenize{autoscripts/script-link::doc}]{\sphinxcrossref{\DUrole{doc}{link.py}}}}: Creates symbolic links to PFANT data files as an alternative to copying these (sometimes large) files into local directoy

\item {} 
{\hyperref[\detokenize{autoscripts/script-merge-molecules::doc}]{\sphinxcrossref{\DUrole{doc}{merge-molecules.py}}}}: Merges several PFANT molecular lines file into a single one

\item {} 
\DUrole{xref,std,std-doc}{run-multi.py{}`}: Runs pfant and nulbad in “multi mode” (equivalent to Tab 4 in \sphinxcode{x.py \textless{}autoscripts/script-run-multi.py}: Runs pfant and nulbad in “multi mod)

\item {} 
{\hyperref[\detokenize{autoscripts/script-run4::doc}]{\sphinxcrossref{\DUrole{doc}{run4.py}}}}: Runs the four Fortran binaries in sequence: \sphinxtitleref{innewmarcs}, \sphinxtitleref{hydro2}, \sphinxtitleref{pfant}, \sphinxtitleref{nulbad}

\end{itemize}

\begin{sphinxadmonition}{hint}{Hint:}
You can use \sphinxcode{programs.py} to list available scripts.
\end{sphinxadmonition}


\section{List of acronyms}
\label{\detokenize{intro:list-of-acronyms}}
\sphinxstylestrong{API} \textendash{} application programming interface

\sphinxstylestrong{GUI} \textendash{} graphical user interface

\sphinxstylestrong{FWHM} \textendash{} full with at half maximum


\section{Acknowledgement}
\label{\detokenize{intro:acknowledgement}}
The project started in 2015 at IAG-USP (Institute of Astronomy, Geophysics and Atmospheric Sciences
at University of São Paulo, Brazil).

Partially funded by FAPESP - Research Support Foundation of the State of São Paulo, Brazil (2015-2017).


\chapter{F311 Installation}
\label{\detokenize{install::doc}}\label{\detokenize{install:f311-installation}}
Python 3 version required: Python 3.4.6+, Python 3.5.3+, or Python 3.6+ (\sphinxurl{https://packaging.python.org/guides/migrating-to-pypi-org/})


\section{Method 1: Using Anaconda without virtual environment}
\label{\detokenize{install:method-1-using-anaconda-without-virtual-environment}}
This will make Anaconda’s Python 3 the default \sphinxcode{python} command for your user account.
Make sure you don’t mind this, otherwise follow Method 2.

First install Anaconda or Miniconda. When you do so, please make sure that you \sphinxstylestrong{answer “yes” to this (or similar) question}:

\begin{sphinxVerbatim}[commandchars=\\\{\}]
Do you wish the installer to prepend the Miniconda3 install location
to PATH in your /home/j/.bashrc ? [yes\textbar{}no]
\PYGZgt{}\PYGZgt{} yes
\end{sphinxVerbatim}

After Anaconda/Miniconda installation, close the terminal and open it again so that your PATH is updated.
\sphinxstylestrong{Or if your shell is bash}, you can just type \sphinxcode{source \textasciitilde{}/.bashrc} on the terminal.

Now install some packages using pip:

\begin{sphinxVerbatim}[commandchars=\\\{\}]
\PYG{n}{pip} \PYG{n}{install} \PYG{n}{numpy} \PYG{n}{scipy} \PYG{n}{matplotlib} \PYG{n}{astropy} \PYG{n}{configobj} \PYG{n}{bs4} \PYG{n}{robobrowser} \PYG{n}{requests} \PYG{n}{fortranformat} \PYG{n}{tabulate} \PYG{n}{rows} \PYG{n}{pyqt5} \PYG{n}{a99} \PYG{n}{f311}
\end{sphinxVerbatim}


\section{Method 2: Using Anaconda virtual environment}
\label{\detokenize{install:method-2-using-anaconda-virtual-environment}}
This method uses a \sphinxstylestrong{conda} virtual environment. It works with a separate installation of Python and related packages.

First you will need to have Anaconda or Miniconda installed. If you have none of these installed yet, just install Miniconda.

Once Anaconda/Miniconda is installed, create a new virtual environment called “astroenv” (or any name you like):

\begin{sphinxVerbatim}[commandchars=\\\{\}]
\PYG{n}{conda} \PYG{n}{create} \PYG{o}{\PYGZhy{}}\PYG{o}{\PYGZhy{}}\PYG{n}{name} \PYG{n}{astroenv} \PYG{n}{python}\PYG{o}{=}\PYG{l+m+mf}{3.5} \PYG{c+c1}{\PYGZsh{} or 3.6}
\end{sphinxVerbatim}

Activate this new virtual environment:

\begin{sphinxVerbatim}[commandchars=\\\{\}]
\PYG{n}{source} \PYG{n}{activate} \PYG{n}{astroenv}
\end{sphinxVerbatim}

Now install the packages:

\begin{sphinxVerbatim}[commandchars=\\\{\}]
\PYG{n}{pip} \PYG{n}{install} \PYG{n}{numpy} \PYG{n}{scipy} \PYG{n}{matplotlib} \PYG{n}{astropy} \PYG{n}{configobj} \PYG{n}{bs4} \PYG{n}{lxml} \PYG{n}{robobrowser} \PYG{n}{requests} \PYG{n}{fortranformat} \PYG{n}{tabulate} \PYG{n}{rows} \PYG{n}{pyqt5} \PYG{n}{a99} \PYG{n}{f311}
\end{sphinxVerbatim}

\sphinxstylestrong{Note} Every time you want to work with F311, you will need to activate the environment:

\begin{sphinxVerbatim}[commandchars=\\\{\}]
\PYG{n}{source} \PYG{n}{activate} \PYG{n}{astroenv}
\end{sphinxVerbatim}

To deactivate the environment:

\begin{sphinxVerbatim}[commandchars=\\\{\}]
\PYG{n}{source} \PYG{n}{deactivate}
\end{sphinxVerbatim}


\section{Method 3: Developer mode}
\label{\detokenize{install:method-3-developer-mode}}
This allows you to pull the most recent version of the code directly from GitHub.

First, follow Method 1 or 2 above, \sphinxstylestrong{but do not install f311}, \sphinxstyleemphasis{i.e.}, the pip command should be:

\begin{sphinxVerbatim}[commandchars=\\\{\}]
\PYG{n}{pip} \PYG{n}{install} \PYG{n}{numpy} \PYG{n}{scipy} \PYG{n}{matplotlib} \PYG{n}{astropy} \PYG{n}{configobj} \PYG{n}{bs4} \PYG{n}{lxml} \PYG{n}{robobrowser} \PYG{n}{requests} \PYG{n}{fortranformat} \PYG{n}{tabulate} \PYG{n}{rows} \PYG{n}{pyqt5} \PYG{n}{a99}
\end{sphinxVerbatim}

Second, clone the f311 GitHub repository:

\begin{sphinxVerbatim}[commandchars=\\\{\}]
\PYG{n}{git} \PYG{n}{clone} \PYG{n}{ssh}\PYG{p}{:}\PYG{o}{/}\PYG{o}{/}\PYG{n}{git}\PYG{n+nd}{@github}\PYG{o}{.}\PYG{n}{com}\PYG{o}{/}\PYG{n}{trevisanj}\PYG{o}{/}\PYG{n}{f311}\PYG{o}{.}\PYG{n}{git}
\end{sphinxVerbatim}

or

\begin{sphinxVerbatim}[commandchars=\\\{\}]
\PYG{n}{git} \PYG{n}{clone} \PYG{n}{http}\PYG{p}{:}\PYG{o}{/}\PYG{o}{/}\PYG{n}{github}\PYG{o}{.}\PYG{n}{com}\PYG{o}{/}\PYG{n}{trevisanj}\PYG{o}{/}\PYG{n}{f311}
\end{sphinxVerbatim}

Finally, install F311 in \sphinxstylestrong{developer} mode:

\begin{sphinxVerbatim}[commandchars=\\\{\}]
\PYG{n}{cd} \PYG{n}{f311}
\PYG{n}{python} \PYG{n}{setup}\PYG{o}{.}\PYG{n}{py} \PYG{n}{develop}
\end{sphinxVerbatim}


\section{Troubleshooting installation}
\label{\detokenize{install:troubleshooting-installation}}

\subsection{MatPlotLib and PyQt5}
\label{\detokenize{install:matplotlib-and-pyqt5}}
\begin{sphinxVerbatim}[commandchars=\\\{\}]
\PYG{n+ne}{ValueError}\PYG{p}{:} \PYG{n}{Unrecognized} \PYG{n}{backend} \PYG{n}{string} \PYG{l+s+s2}{\PYGZdq{}}\PYG{l+s+s2}{qt5agg}\PYG{l+s+s2}{\PYGZdq{}}\PYG{p}{:} \PYG{n}{valid} \PYG{n}{strings} \PYG{n}{are} \PYG{p}{[}\PYG{l+s+s1}{\PYGZsq{}}\PYG{l+s+s1}{GTKAgg}\PYG{l+s+s1}{\PYGZsq{}}\PYG{p}{,} \PYG{l+s+s1}{\PYGZsq{}}\PYG{l+s+s1}{template}\PYG{l+s+s1}{\PYGZsq{}}\PYG{p}{,} \PYG{l+s+s1}{\PYGZsq{}}\PYG{l+s+s1}{pdf}\PYG{l+s+s1}{\PYGZsq{}}\PYG{p}{,}
\PYG{l+s+s1}{\PYGZsq{}}\PYG{l+s+s1}{GTK3Agg}\PYG{l+s+s1}{\PYGZsq{}}\PYG{p}{,} \PYG{l+s+s1}{\PYGZsq{}}\PYG{l+s+s1}{cairo}\PYG{l+s+s1}{\PYGZsq{}}\PYG{p}{,} \PYG{l+s+s1}{\PYGZsq{}}\PYG{l+s+s1}{TkAgg}\PYG{l+s+s1}{\PYGZsq{}}\PYG{p}{,} \PYG{l+s+s1}{\PYGZsq{}}\PYG{l+s+s1}{pgf}\PYG{l+s+s1}{\PYGZsq{}}\PYG{p}{,} \PYG{l+s+s1}{\PYGZsq{}}\PYG{l+s+s1}{MacOSX}\PYG{l+s+s1}{\PYGZsq{}}\PYG{p}{,} \PYG{l+s+s1}{\PYGZsq{}}\PYG{l+s+s1}{GTK}\PYG{l+s+s1}{\PYGZsq{}}\PYG{p}{,} \PYG{l+s+s1}{\PYGZsq{}}\PYG{l+s+s1}{WX}\PYG{l+s+s1}{\PYGZsq{}}\PYG{p}{,} \PYG{l+s+s1}{\PYGZsq{}}\PYG{l+s+s1}{GTKCairo}\PYG{l+s+s1}{\PYGZsq{}}\PYG{p}{,} \PYG{l+s+s1}{\PYGZsq{}}\PYG{l+s+s1}{Qt4Agg}\PYG{l+s+s1}{\PYGZsq{}}\PYG{p}{,} \PYG{l+s+s1}{\PYGZsq{}}\PYG{l+s+s1}{svg}\PYG{l+s+s1}{\PYGZsq{}}\PYG{p}{,} \PYG{l+s+s1}{\PYGZsq{}}\PYG{l+s+s1}{agg}\PYG{l+s+s1}{\PYGZsq{}}\PYG{p}{,}
\PYG{l+s+s1}{\PYGZsq{}}\PYG{l+s+s1}{ps}\PYG{l+s+s1}{\PYGZsq{}}\PYG{p}{,} \PYG{l+s+s1}{\PYGZsq{}}\PYG{l+s+s1}{emf}\PYG{l+s+s1}{\PYGZsq{}}\PYG{p}{,} \PYG{l+s+s1}{\PYGZsq{}}\PYG{l+s+s1}{WebAgg}\PYG{l+s+s1}{\PYGZsq{}}\PYG{p}{,} \PYG{l+s+s1}{\PYGZsq{}}\PYG{l+s+s1}{gdk}\PYG{l+s+s1}{\PYGZsq{}}\PYG{p}{,} \PYG{l+s+s1}{\PYGZsq{}}\PYG{l+s+s1}{WXAgg}\PYG{l+s+s1}{\PYGZsq{}}\PYG{p}{,} \PYG{l+s+s1}{\PYGZsq{}}\PYG{l+s+s1}{CocoaAgg}\PYG{l+s+s1}{\PYGZsq{}}\PYG{p}{,} \PYG{l+s+s1}{\PYGZsq{}}\PYG{l+s+s1}{GTK3Cairo}\PYG{l+s+s1}{\PYGZsq{}}\PYG{p}{]}
\end{sphinxVerbatim}

\sphinxstylestrong{Solution}: update Matplotlib to version 1.4 or later


\subsection{Problems with package bs4}
\label{\detokenize{install:problems-with-package-bs4}}
\begin{sphinxVerbatim}[commandchars=\\\{\}]
\PYG{n}{bs4}\PYG{o}{.}\PYG{n}{FeatureNotFound}\PYG{p}{:} \PYG{n}{Couldn}\PYG{l+s+s1}{\PYGZsq{}}\PYG{l+s+s1}{t find a tree builder with the features you requested: lxml. Do you need to install a parser library?}
\end{sphinxVerbatim}

\sphinxstylestrong{Solution}: install package “lxml”: \sphinxcode{pip install lxml}


\chapter{Package f311}
\label{\detokenize{f3110::doc}}\label{\detokenize{f3110:package-f311}}
The root package provides the collaboration model implementation.

The collaboration model allows \sphinxcode{f311.COLLABORATORS} to contribute with
\begin{itemize}
\item {} 
\sphinxcode{DataFile} subclasses, i.e., implement handling (load, save, etc.) of new file types;

\item {} 
\sphinxcode{Vis} subclasses, i.e., implement visualizations for these files;

\item {} 
Standalone scripts placed in the directory \sphinxcode{packagename/scripts}

\end{itemize}


\section{Print file handling classes information}
\label{\detokenize{f3110:print-file-handling-classes-information}}
\begin{sphinxVerbatim}[commandchars=\\\{\}]
\PYG{l+s+sd}{\PYGZdq{}\PYGZdq{}\PYGZdq{}Lists different subsets of DataFile subclasses\PYGZdq{}\PYGZdq{}\PYGZdq{}}
\PYG{k+kn}{import} \PYG{n+nn}{f311}

\PYG{n}{titles} \PYG{o}{=} \PYG{p}{(}\PYG{l+s+s2}{\PYGZdq{}}\PYG{l+s+s2}{text}\PYG{l+s+s2}{\PYGZdq{}}\PYG{p}{,} \PYG{l+s+s2}{\PYGZdq{}}\PYG{l+s+s2}{binary}\PYG{l+s+s2}{\PYGZdq{}}\PYG{p}{,} \PYG{l+s+s2}{\PYGZdq{}}\PYG{l+s+s2}{1D spectrum}\PYG{l+s+s2}{\PYGZdq{}}\PYG{p}{)}
\PYG{n}{allclasses} \PYG{o}{=} \PYG{p}{(}\PYG{n}{f311}\PYG{o}{.}\PYG{n}{classes\PYGZus{}txt}\PYG{p}{(}\PYG{p}{)}\PYG{p}{,} \PYG{n}{f311}\PYG{o}{.}\PYG{n}{classes\PYGZus{}bin}\PYG{p}{(}\PYG{p}{)}\PYG{p}{,} \PYG{n}{f311}\PYG{o}{.}\PYG{n}{classes\PYGZus{}sp}\PYG{p}{(}\PYG{p}{)}\PYG{p}{)}

\PYG{k}{for} \PYG{n}{title}\PYG{p}{,} \PYG{n}{classes} \PYG{o+ow}{in} \PYG{n+nb}{zip}\PYG{p}{(}\PYG{n}{titles}\PYG{p}{,} \PYG{n}{allclasses}\PYG{p}{)}\PYG{p}{:}
    \PYG{n+nb}{print}\PYG{p}{(}\PYG{l+s+s2}{\PYGZdq{}}\PYG{l+s+se}{\PYGZbs{}n}\PYG{l+s+s2}{*** Classes that can handle }\PYG{l+s+si}{\PYGZob{}\PYGZcb{}}\PYG{l+s+s2}{ files***}\PYG{l+s+s2}{\PYGZdq{}}\PYG{o}{.}\PYG{n}{format}\PYG{p}{(}\PYG{n}{title}\PYG{p}{)}\PYG{p}{)}
    \PYG{k}{for} \PYG{n+nb+bp}{cls} \PYG{o+ow}{in} \PYG{n}{classes}\PYG{p}{:}
        \PYG{n+nb}{print}\PYG{p}{(}\PYG{l+s+s2}{\PYGZdq{}}\PYG{l+s+si}{\PYGZob{}:25\PYGZcb{}}\PYG{l+s+s2}{: }\PYG{l+s+si}{\PYGZob{}\PYGZcb{}}\PYG{l+s+s2}{\PYGZdq{}}\PYG{o}{.}\PYG{n}{format}\PYG{p}{(}\PYG{n+nb+bp}{cls}\PYG{o}{.}\PYG{n+nv+vm}{\PYGZus{}\PYGZus{}name\PYGZus{}\PYGZus{}}\PYG{p}{,} \PYG{n+nb+bp}{cls}\PYG{o}{.}\PYG{n+nv+vm}{\PYGZus{}\PYGZus{}doc\PYGZus{}\PYGZus{}}\PYG{o}{.}\PYG{n}{strip}\PYG{p}{(}\PYG{p}{)}\PYG{o}{.}\PYG{n}{split}\PYG{p}{(}\PYG{l+s+s2}{\PYGZdq{}}\PYG{l+s+se}{\PYGZbs{}n}\PYG{l+s+s2}{\PYGZdq{}}\PYG{p}{)}\PYG{p}{[}\PYG{l+m+mi}{0}\PYG{p}{]}\PYG{p}{)}\PYG{p}{)}
\end{sphinxVerbatim}

The output should be something like:

\begin{sphinxVerbatim}[commandchars=\\\{\}]
*** Classes that can handle text files***
FileAbXFwhm              : {}`x.py{}` Differential Abundances and FWHMs (Python source)
FileAbonds               : PFANT Stellar Chemical Abundances
FileAbsoru2              : \PYGZdq{}Absoru2\PYGZdq{} file
FileAtoms                : PFANT Atomic Lines
FileConfigConvMol        : Python source containing \PYGZsq{}config\PYGZus{}conv = ConfigConv(...)
FileDissoc               : PFANT Stellar Dissociation Equilibrium Information
FileFCF                  : File containing Franck\PYGZhy{}Condon Factors (FCFs)
FileHmap                 : PFANT Hygrogen Lines Map
FileKuruczMolecule       : Kurucz molecular lines file
FileKuruczMoleculeBase   : Base class for the two types of Kurucz molecular lines file
FileKuruczMoleculeOld    : Kurucz molecular lines file, old format \PYGZsh{}0
FileKuruczMoleculeOld1   : Kurucz molecular lines file, old format \PYGZsh{}1
FileMain                 : PFANT Stellar Main Configuration
FileModTxt               : MARCS Atmospheric Model (text file)
FileMolConsts            : Python source containing \PYGZsq{}fobj = MolConsts(...)
FileMolecules            : PFANT Molecular Lines
FileOpa                  : MARCS \PYGZdq{}.opa\PYGZdq{} (opacity model) file format.
FileOptions              : {}`x.py{}` Command\PYGZhy{}line Options
FilePar                  : WebSim\PYGZhy{}COMPASS \PYGZdq{}.par\PYGZdq{} (parameters) file
FilePartit               : PFANT Partition Function
FilePlezTiO              : Plez molecular lines file, TiO format
FilePy                   : Configuration file saved as a .py Python source script
FilePyConfig             : Base class for config files. Inherit and set class variable \PYGZsq{}modulevarname\PYGZsq{} besides usual
FileSpectrum             : Base class for all files representing a single 1D spectrum
FileSpectrumNulbad       : PFANT Spectrum ({}`nulbad{}` output)
FileSpectrumPfant        : PFANT Spectrum ({}`pfant{}` output)
FileSpectrumXY           : \PYGZdq{}Lambda\PYGZhy{}flux\PYGZdq{} Spectrum (2\PYGZhy{}column text file)
FileToH                  : PFANT Hydrogen Line Profile
FileVald3                : VALD3 atomic or molecular lines file

*** Classes that can handle binary files***
FileFullCube             : FITS Data Cube (\PYGZdq{}full\PYGZdq{} opposed to \PYGZdq{}sparse\PYGZdq{})
FileGalfit               : FITS file with frames named INPUT\PYGZus{}*, MODEL\PYGZus{}*, RESIDUAL\PYGZus{}*, which is the output of Galfit software
FileHitranDB             : HITRAN Molecules Catalogue
FileModBin               : PFANT Atmospheric Model (binary file)
FileMolDB                : Database of Molecular Constants
FileMoo                  : Atmospheric model or grid of models (with opacities included)
FileSQLiteDB             : Represents a SQLite database file.
FileSparseCube           : FITS Sparse Data Cube (storage to take less disk space)
FileSpectrumFits         : FITS Spectrum
FileSpectrumList         : FITS Spectrum List

*** Classes that can handle 1D spectrum files***
FileSpectrum             : Base class for all files representing a single 1D spectrum
FileSpectrumFits         : FITS Spectrum
FileSpectrumNulbad       : PFANT Spectrum ({}`nulbad{}` output)
FileSpectrumPfant        : PFANT Spectrum ({}`pfant{}` output)
FileSpectrumXY           : \PYGZdq{}Lambda\PYGZhy{}flux\PYGZdq{} Spectrum (2\PYGZhy{}column text file)
\end{sphinxVerbatim}


\subsection{API reference}
\label{\detokenize{f3110:api-reference}}
\DUrole{xref,std,std-doc}{autodoc/f311}


\chapter{Spectral synthesis}
\label{\detokenize{pyfant::doc}}\label{\detokenize{pyfant:spectral-synthesis}}
Welcome!!

pyfant is a Python interface for the \sphinxhref{http://trevisanj.github.io/PFANT}{PFANT Spectral Synthesis
Software} for Astronomy.

Spectral synthesis softwares have a fundamental role in Astronomy.
It is a crucial step in determining stellar properties
- such as temperature, metallicity, and chemical abundances -
in which the synthetic spectrum (or a combination of several of these) is compared with the
measured spectrum of a star or a whole stellar population either by the full spectrum fitting,
spectral energy distribution or specific spectral lines and regions.
It is of great interest that the software has a comprehensive and intuitive user interface and
easiness of parameter input and its multiple variations, and also tools for incorporating data
like atomic/molecular lines, atmospheric models, etc.

Package \sphinxcode{f311.pyfant} provides a Python interface to the PFANT Fortran binaries, including the
ability to run the Fortran binaries in parallel in a multi-processing scheme via API or GUI.

manipulate and save PFANT data files using \sphinxcode{f311.filetypes}, allow for complex batch operations.


\section{Applications}
\label{\detokenize{pyfant:applications}}
The applications related to package f311.pyfant are listed below. For them to work, you need to
\sphinxhref{http://trevisanj.github.io/PFANT/install.html}{install PFANT}.

The \sphinxhref{http://trevisanj.github.io/PFANT/quick.html}{PFANT Quick Start} serves as a guide to
using some of these applications.


\subsection{Graphical applications}
\label{\detokenize{pyfant:graphical-applications}}\begin{itemize}
\item {} 
\sphinxcode{x.py} \textendash{} PFANT Launcher \textendash{} Graphical Interface for Spectral Synthesis

\end{itemize}


\subsection{Command-line tools}
\label{\detokenize{pyfant:command-line-tools}}\begin{itemize}
\item {} 
\sphinxcode{copy-star.py} \textendash{} Copies stellar data files (such as main.dat, abonds.dat, dissoc.dat) to local directory

\item {} 
\sphinxcode{link.py} \textendash{} Creates symbolic links to PFANT data files as an alternative to copying these (sometimes large) files into local directory

\item {} 
\sphinxcode{run4.py} \textendash{} Runs the four Fortran binaries in sequence: \sphinxcode{innewmarcs}, \sphinxcode{hydro2}, \sphinxcode{pfant}, \sphinxcode{nulbad}

\item {} 
\sphinxcode{save-pdf.py} \textendash{} Looks for files “\sphinxstyleemphasis{.norm” inside directories session-} and saves one figure per page in a PDF file

\end{itemize}


\section{Coding using the API}
\label{\detokenize{pyfant:coding-using-the-api}}
This section contains a series of examples on how to use the PFANT Fortran executables from a
Python script. These “bindings” to the Fortran binaries, together with the ability to load,
manipulate and save PFANT data files using \sphinxcode{f311.filetypes}, allow for complex batch operations.


\subsection{Spectral synthesis}
\label{\detokenize{pyfant:id1}}
\begin{sphinxVerbatim}[commandchars=\\\{\}]
\PYG{l+s+sd}{\PYGZdq{}\PYGZdq{}\PYGZdq{}Runs synthesis over short wavelength range, then plots normalized and convolved spectrum\PYGZdq{}\PYGZdq{}\PYGZdq{}}

\PYG{k+kn}{import} \PYG{n+nn}{f311}\PYG{n+nn}{.}\PYG{n+nn}{pyfant} \PYG{k}{as} \PYG{n+nn}{pf}
\PYG{k+kn}{import} \PYG{n+nn}{f311}\PYG{n+nn}{.}\PYG{n+nn}{explorer} \PYG{k}{as} \PYG{n+nn}{ex}
\PYG{k+kn}{import} \PYG{n+nn}{matplotlib}\PYG{n+nn}{.}\PYG{n+nn}{pyplot} \PYG{k}{as} \PYG{n+nn}{plt}


\PYG{k}{if} \PYG{n+nv+vm}{\PYGZus{}\PYGZus{}name\PYGZus{}\PYGZus{}} \PYG{o}{==} \PYG{l+s+s2}{\PYGZdq{}}\PYG{l+s+s2}{\PYGZus{}\PYGZus{}main\PYGZus{}\PYGZus{}}\PYG{l+s+s2}{\PYGZdq{}}\PYG{p}{:}
    \PYG{c+c1}{\PYGZsh{} Copies files main.dat and abonds.dat to local directory (for given star)}
    \PYG{n}{pf}\PYG{o}{.}\PYG{n}{copy\PYGZus{}star}\PYG{p}{(}\PYG{n}{starname}\PYG{o}{=}\PYG{l+s+s2}{\PYGZdq{}}\PYG{l+s+s2}{sun\PYGZhy{}grevesse\PYGZhy{}1996}\PYG{l+s+s2}{\PYGZdq{}}\PYG{p}{)}
    \PYG{c+c1}{\PYGZsh{} Creates symbolic links to all non\PYGZhy{}star\PYGZhy{}specific files, such as atomic \PYGZam{} molecular lines,}
    \PYG{c+c1}{\PYGZsh{} partition functions, etc.}
    \PYG{n}{pf}\PYG{o}{.}\PYG{n}{link\PYGZus{}to\PYGZus{}data}\PYG{p}{(}\PYG{p}{)}

    \PYG{c+c1}{\PYGZsh{} \PYGZsh{} First run}
    \PYG{c+c1}{\PYGZsh{} Creates object that will run the four Fortran executables (innewmarcs, hydro2, pfant, nulbad)}
    \PYG{n}{obj} \PYG{o}{=} \PYG{n}{pf}\PYG{o}{.}\PYG{n}{Combo}\PYG{p}{(}\PYG{p}{)}
    \PYG{c+c1}{\PYGZsh{} synthesis interval start (angstrom)}
    \PYG{n}{obj}\PYG{o}{.}\PYG{n}{conf}\PYG{o}{.}\PYG{n}{opt}\PYG{o}{.}\PYG{n}{llzero} \PYG{o}{=} \PYG{l+m+mi}{6530}
    \PYG{c+c1}{\PYGZsh{} synthesis interval end (angstrom)}
    \PYG{n}{obj}\PYG{o}{.}\PYG{n}{conf}\PYG{o}{.}\PYG{n}{opt}\PYG{o}{.}\PYG{n}{llfin} \PYG{o}{=} \PYG{l+m+mi}{6535}

    \PYG{c+c1}{\PYGZsh{} Runs Fortrans and hangs until done}
    \PYG{n}{obj}\PYG{o}{.}\PYG{n}{run}\PYG{p}{(}\PYG{p}{)}

    \PYG{c+c1}{\PYGZsh{} Loads result files into memory. obj.result is a dictionary containing elements ...}
    \PYG{n}{obj}\PYG{o}{.}\PYG{n}{load\PYGZus{}result}\PYG{p}{(}\PYG{p}{)}
    \PYG{n+nb}{print}\PYG{p}{(}\PYG{l+s+s2}{\PYGZdq{}}\PYG{l+s+s2}{obj.result = }\PYG{l+s+si}{\PYGZob{}\PYGZcb{}}\PYG{l+s+s2}{\PYGZdq{}}\PYG{o}{.}\PYG{n}{format}\PYG{p}{(}\PYG{n}{obj}\PYG{o}{.}\PYG{n}{result}\PYG{p}{)}\PYG{p}{)}
    \PYG{n}{res} \PYG{o}{=} \PYG{n}{obj}\PYG{o}{.}\PYG{n}{result}
    \PYG{n}{plt}\PYG{o}{.}\PYG{n}{figure}\PYG{p}{(}\PYG{p}{)}
    \PYG{n}{ex}\PYG{o}{.}\PYG{n}{draw\PYGZus{}spectra\PYGZus{}overlapped}\PYG{p}{(}\PYG{p}{[}\PYG{n}{res}\PYG{p}{[}\PYG{l+s+s2}{\PYGZdq{}}\PYG{l+s+s2}{norm}\PYG{l+s+s2}{\PYGZdq{}}\PYG{p}{]}\PYG{p}{,} \PYG{n}{res}\PYG{p}{[}\PYG{l+s+s2}{\PYGZdq{}}\PYG{l+s+s2}{convolved}\PYG{l+s+s2}{\PYGZdq{}}\PYG{p}{]}\PYG{p}{]}\PYG{p}{)}
    \PYG{n}{plt}\PYG{o}{.}\PYG{n}{savefig}\PYG{p}{(}\PYG{l+s+s2}{\PYGZdq{}}\PYG{l+s+s2}{norm\PYGZhy{}convolved.png}\PYG{l+s+s2}{\PYGZdq{}}\PYG{p}{)}
    \PYG{n}{plt}\PYG{o}{.}\PYG{n}{show}\PYG{p}{(}\PYG{p}{)}
\end{sphinxVerbatim}

\begin{figure}[htbp]
\centering

\noindent\sphinxincludegraphics{{norm-convolved}.png}
\end{figure}


\subsection{Spectral synthesis - convolutions}
\label{\detokenize{pyfant:spectral-synthesis-convolutions}}
The following example convolves the synthetic spectrum (file “flux.norm”) with Gaussian profiles of
different FWHMs.

\begin{sphinxVerbatim}[commandchars=\\\{\}]
\PYG{c+ch}{\PYGZsh{}!/usr/bin/env python}

\PYG{l+s+sd}{\PYGZdq{}\PYGZdq{}\PYGZdq{}Runs synthesis over short wavelength range, then plots normalized and convolved spectrum\PYGZdq{}\PYGZdq{}\PYGZdq{}}

\PYG{k+kn}{import} \PYG{n+nn}{f311}\PYG{n+nn}{.}\PYG{n+nn}{pyfant} \PYG{k}{as} \PYG{n+nn}{pf}
\PYG{k+kn}{import} \PYG{n+nn}{f311}\PYG{n+nn}{.}\PYG{n+nn}{explorer} \PYG{k}{as} \PYG{n+nn}{ex}
\PYG{k+kn}{import} \PYG{n+nn}{matplotlib}\PYG{n+nn}{.}\PYG{n+nn}{pyplot} \PYG{k}{as} \PYG{n+nn}{plt}
\PYG{k+kn}{import} \PYG{n+nn}{a99}

\PYG{c+c1}{\PYGZsh{} FWHM (full width at half of maximum) of Gaussian profiles in angstrom}
\PYG{n}{FWHMS} \PYG{o}{=} \PYG{p}{[}\PYG{l+m+mf}{0.03}\PYG{p}{,} \PYG{l+m+mf}{0.06}\PYG{p}{,} \PYG{l+m+mf}{0.09}\PYG{p}{,} \PYG{l+m+mf}{0.12}\PYG{p}{,} \PYG{l+m+mf}{0.15}\PYG{p}{,} \PYG{l+m+mf}{0.20}\PYG{p}{,} \PYG{l+m+mf}{0.25}\PYG{p}{,} \PYG{l+m+mf}{0.3}\PYG{p}{,} \PYG{l+m+mf}{0.5}\PYG{p}{]}

\PYG{k}{if} \PYG{n+nv+vm}{\PYGZus{}\PYGZus{}name\PYGZus{}\PYGZus{}} \PYG{o}{==} \PYG{l+s+s2}{\PYGZdq{}}\PYG{l+s+s2}{\PYGZus{}\PYGZus{}main\PYGZus{}\PYGZus{}}\PYG{l+s+s2}{\PYGZdq{}}\PYG{p}{:}
    \PYG{c+c1}{\PYGZsh{} Copies files main.dat and abonds.dat to local directory (for given star)}
    \PYG{n}{pf}\PYG{o}{.}\PYG{n}{copy\PYGZus{}star}\PYG{p}{(}\PYG{n}{starname}\PYG{o}{=}\PYG{l+s+s2}{\PYGZdq{}}\PYG{l+s+s2}{sun\PYGZhy{}grevesse\PYGZhy{}1996}\PYG{l+s+s2}{\PYGZdq{}}\PYG{p}{)}
    \PYG{c+c1}{\PYGZsh{} Creates symbolic links to all non\PYGZhy{}star\PYGZhy{}specific files}
    \PYG{n}{pf}\PYG{o}{.}\PYG{n}{link\PYGZus{}to\PYGZus{}data}\PYG{p}{(}\PYG{p}{)}

    \PYG{c+c1}{\PYGZsh{} \PYGZsh{} 1) Spectral synthesis}
    \PYG{c+c1}{\PYGZsh{} Creates object that will run the four Fortran executables (innewmarcs, hydro2, pfant, nulbad)}
    \PYG{n}{ecombo} \PYG{o}{=} \PYG{n}{pf}\PYG{o}{.}\PYG{n}{Combo}\PYG{p}{(}\PYG{p}{)}
    \PYG{c+c1}{\PYGZsh{} synthesis interval start (angstrom)}
    \PYG{n}{ecombo}\PYG{o}{.}\PYG{n}{conf}\PYG{o}{.}\PYG{n}{opt}\PYG{o}{.}\PYG{n}{llzero} \PYG{o}{=} \PYG{l+m+mi}{6530}
    \PYG{c+c1}{\PYGZsh{} synthesis interval end (angstrom)}
    \PYG{n}{ecombo}\PYG{o}{.}\PYG{n}{conf}\PYG{o}{.}\PYG{n}{opt}\PYG{o}{.}\PYG{n}{llfin} \PYG{o}{=} \PYG{l+m+mi}{6535}
    \PYG{c+c1}{\PYGZsh{} Runs Fortrans and hangs until done}
    \PYG{n}{ecombo}\PYG{o}{.}\PYG{n}{run}\PYG{p}{(}\PYG{p}{)}
    \PYG{n}{ecombo}\PYG{o}{.}\PYG{n}{load\PYGZus{}result}\PYG{p}{(}\PYG{p}{)}
    \PYG{c+c1}{\PYGZsh{} Retains un\PYGZhy{}convolved spectrum for comparison}
    \PYG{n}{spectra} \PYG{o}{=} \PYG{p}{[}\PYG{n}{ecombo}\PYG{o}{.}\PYG{n}{result}\PYG{p}{[}\PYG{l+s+s2}{\PYGZdq{}}\PYG{l+s+s2}{norm}\PYG{l+s+s2}{\PYGZdq{}}\PYG{p}{]}\PYG{p}{]}

    \PYG{c+c1}{\PYGZsh{} \PYGZsh{} 2) Convolutions}
    \PYG{k}{for} \PYG{n}{fwhm} \PYG{o+ow}{in} \PYG{n}{FWHMS}\PYG{p}{:}
        \PYG{n}{enulbad} \PYG{o}{=} \PYG{n}{pf}\PYG{o}{.}\PYG{n}{Nulbad}\PYG{p}{(}\PYG{p}{)}
        \PYG{n}{enulbad}\PYG{o}{.}\PYG{n}{conf}\PYG{o}{.}\PYG{n}{opt}\PYG{o}{.}\PYG{n}{fwhm} \PYG{o}{=} \PYG{n}{fwhm}
        \PYG{n}{enulbad}\PYG{o}{.}\PYG{n}{run}\PYG{p}{(}\PYG{p}{)}
        \PYG{n}{enulbad}\PYG{o}{.}\PYG{n}{load\PYGZus{}result}\PYG{p}{(}\PYG{p}{)}
        \PYG{c+c1}{\PYGZsh{} Appends convolved spectrum for comparison}
        \PYG{n}{spectra}\PYG{o}{.}\PYG{n}{append}\PYG{p}{(}\PYG{n}{enulbad}\PYG{o}{.}\PYG{n}{result}\PYG{p}{[}\PYG{l+s+s2}{\PYGZdq{}}\PYG{l+s+s2}{convolved}\PYG{l+s+s2}{\PYGZdq{}}\PYG{p}{]}\PYG{p}{)}

    \PYG{c+c1}{\PYGZsh{} \PYGZsh{} 3) Plots}
    \PYG{n}{plt}\PYG{o}{.}\PYG{n}{figure}\PYG{p}{(}\PYG{p}{)}
    \PYG{n}{ex}\PYG{o}{.}\PYG{n}{draw\PYGZus{}spectra\PYGZus{}overlapped}\PYG{p}{(}\PYG{n}{spectra}\PYG{p}{)}
    \PYG{n}{K} \PYG{o}{=} \PYG{l+m+mf}{1.1}
    \PYG{n}{a99}\PYG{o}{.}\PYG{n}{set\PYGZus{}figure\PYGZus{}size}\PYG{p}{(}\PYG{n}{plt}\PYG{o}{.}\PYG{n}{gcf}\PYG{p}{(}\PYG{p}{)}\PYG{p}{,} \PYG{l+m+mi}{1000}\PYG{o}{*}\PYG{n}{K}\PYG{p}{,} \PYG{l+m+mi}{500}\PYG{o}{*}\PYG{n}{K}\PYG{p}{)}
    \PYG{n}{plt}\PYG{o}{.}\PYG{n}{tight\PYGZus{}layout}\PYG{p}{(}\PYG{p}{)}
    \PYG{n}{plt}\PYG{o}{.}\PYG{n}{savefig}\PYG{p}{(}\PYG{l+s+s2}{\PYGZdq{}}\PYG{l+s+s2}{many\PYGZhy{}convs.png}\PYG{l+s+s2}{\PYGZdq{}}\PYG{p}{)}
    \PYG{n}{plt}\PYG{o}{.}\PYG{n}{show}\PYG{p}{(}\PYG{p}{)}
\end{sphinxVerbatim}

\begin{figure}[htbp]
\centering

\noindent\sphinxincludegraphics{{many-convs}.png}
\end{figure}


\subsection{Spectral synthesis - Continuum}
\label{\detokenize{pyfant:spectral-synthesis-continuum}}
\begin{sphinxVerbatim}[commandchars=\\\{\}]
\PYG{l+s+sd}{\PYGZdq{}\PYGZdq{}\PYGZdq{}Runs synthesis over large wavelength range, then plots continuum\PYGZdq{}\PYGZdq{}\PYGZdq{}}

\PYG{k+kn}{import} \PYG{n+nn}{f311}\PYG{n+nn}{.}\PYG{n+nn}{pyfant} \PYG{k}{as} \PYG{n+nn}{pf}
\PYG{k+kn}{import} \PYG{n+nn}{f311}\PYG{n+nn}{.}\PYG{n+nn}{explorer} \PYG{k}{as} \PYG{n+nn}{ex}
\PYG{k+kn}{import} \PYG{n+nn}{matplotlib}\PYG{n+nn}{.}\PYG{n+nn}{pyplot} \PYG{k}{as} \PYG{n+nn}{plt}
\PYG{k+kn}{import} \PYG{n+nn}{a99}

\PYG{k}{if} \PYG{n+nv+vm}{\PYGZus{}\PYGZus{}name\PYGZus{}\PYGZus{}} \PYG{o}{==} \PYG{l+s+s2}{\PYGZdq{}}\PYG{l+s+s2}{\PYGZus{}\PYGZus{}main\PYGZus{}\PYGZus{}}\PYG{l+s+s2}{\PYGZdq{}}\PYG{p}{:}
    \PYG{c+c1}{\PYGZsh{} Copies files main.dat and abonds.dat to local directory (for given star)}
    \PYG{n}{pf}\PYG{o}{.}\PYG{n}{copy\PYGZus{}star}\PYG{p}{(}\PYG{n}{starname}\PYG{o}{=}\PYG{l+s+s2}{\PYGZdq{}}\PYG{l+s+s2}{sun\PYGZhy{}grevesse\PYGZhy{}1996}\PYG{l+s+s2}{\PYGZdq{}}\PYG{p}{)}
    \PYG{c+c1}{\PYGZsh{} Creates symbolic links to all non\PYGZhy{}star\PYGZhy{}specific files, such as atomic \PYGZam{} molecular lines,}
    \PYG{c+c1}{\PYGZsh{} partition functions, etc.}
    \PYG{n}{pf}\PYG{o}{.}\PYG{n}{link\PYGZus{}to\PYGZus{}data}\PYG{p}{(}\PYG{p}{)}

    \PYG{c+c1}{\PYGZsh{} Creates object that will run the four Fortran executables (innewmarcs, hydro2, pfant, nulbad)}
    \PYG{n}{obj} \PYG{o}{=} \PYG{n}{pf}\PYG{o}{.}\PYG{n}{Combo}\PYG{p}{(}\PYG{p}{)}
    \PYG{n}{oo} \PYG{o}{=} \PYG{n}{obj}\PYG{o}{.}\PYG{n}{conf}\PYG{o}{.}\PYG{n}{opt}
    \PYG{c+c1}{\PYGZsh{} synthesis interval start (angstrom)}
    \PYG{n}{oo}\PYG{o}{.}\PYG{n}{llzero} \PYG{o}{=} \PYG{l+m+mi}{2500}
    \PYG{c+c1}{\PYGZsh{} synthesis interval end (angstrom)}
    \PYG{n}{oo}\PYG{o}{.}\PYG{n}{llfin} \PYG{o}{=} \PYG{l+m+mi}{30000}
    \PYG{c+c1}{\PYGZsh{} savelength step (angstrom)}
    \PYG{n}{oo}\PYG{o}{.}\PYG{n}{pas} \PYG{o}{=} \PYG{l+m+mf}{1.}
    \PYG{c+c1}{\PYGZsh{} Turns off hydrogen lines}
    \PYG{n}{oo}\PYG{o}{.}\PYG{n}{no\PYGZus{}h} \PYG{o}{=} \PYG{k+kc}{True}
    \PYG{c+c1}{\PYGZsh{} Turns off atomic lines}
    \PYG{n}{oo}\PYG{o}{.}\PYG{n}{no\PYGZus{}atoms} \PYG{o}{=} \PYG{k+kc}{True}
    \PYG{c+c1}{\PYGZsh{} Turns off molecular lines}
    \PYG{n}{oo}\PYG{o}{.}\PYG{n}{no\PYGZus{}molecules} \PYG{o}{=} \PYG{k+kc}{True}

    \PYG{n}{obj}\PYG{o}{.}\PYG{n}{run}\PYG{p}{(}\PYG{p}{)}
    \PYG{n}{obj}\PYG{o}{.}\PYG{n}{load\PYGZus{}result}\PYG{p}{(}\PYG{p}{)}
    \PYG{n+nb}{print}\PYG{p}{(}\PYG{l+s+s2}{\PYGZdq{}}\PYG{l+s+s2}{obj.result = }\PYG{l+s+si}{\PYGZob{}\PYGZcb{}}\PYG{l+s+s2}{\PYGZdq{}}\PYG{o}{.}\PYG{n}{format}\PYG{p}{(}\PYG{n}{obj}\PYG{o}{.}\PYG{n}{result}\PYG{p}{)}\PYG{p}{)}
    \PYG{n}{res} \PYG{o}{=} \PYG{n}{obj}\PYG{o}{.}\PYG{n}{result}
    \PYG{n}{ex}\PYG{o}{.}\PYG{n}{draw\PYGZus{}spectra}\PYG{p}{(}\PYG{p}{[}\PYG{n}{res}\PYG{p}{[}\PYG{l+s+s2}{\PYGZdq{}}\PYG{l+s+s2}{cont}\PYG{l+s+s2}{\PYGZdq{}}\PYG{p}{]}\PYG{p}{]}\PYG{p}{,} \PYG{n}{setup}\PYG{o}{=}\PYG{n}{ex}\PYG{o}{.}\PYG{n}{PlotSpectrumSetup}\PYG{p}{(}\PYG{n}{fmt\PYGZus{}ylabel}\PYG{o}{=}\PYG{k+kc}{None}\PYG{p}{)}\PYG{p}{)}
    \PYG{n}{K} \PYG{o}{=} \PYG{o}{.}\PYG{l+m+mi}{75}
    \PYG{n}{a99}\PYG{o}{.}\PYG{n}{set\PYGZus{}figure\PYGZus{}size}\PYG{p}{(}\PYG{n}{plt}\PYG{o}{.}\PYG{n}{gcf}\PYG{p}{(}\PYG{p}{)}\PYG{p}{,} \PYG{l+m+mi}{1300}\PYG{o}{*}\PYG{n}{K}\PYG{p}{,} \PYG{l+m+mi}{450}\PYG{o}{*}\PYG{n}{K}\PYG{p}{)}
    \PYG{n}{plt}\PYG{o}{.}\PYG{n}{tight\PYGZus{}layout}\PYG{p}{(}\PYG{p}{)}
    \PYG{n}{plt}\PYG{o}{.}\PYG{n}{savefig}\PYG{p}{(}\PYG{l+s+s2}{\PYGZdq{}}\PYG{l+s+s2}{continuum.png}\PYG{l+s+s2}{\PYGZdq{}}\PYG{p}{)}
    \PYG{n}{plt}\PYG{o}{.}\PYG{n}{show}\PYG{p}{(}\PYG{p}{)}
\end{sphinxVerbatim}

\begin{figure}[htbp]
\centering

\noindent\sphinxincludegraphics{{continuum}.png}
\end{figure}


\subsection{Spectral synthesis - Separate atomic species}
\label{\detokenize{pyfant:spectral-synthesis-separate-atomic-species}}
PFANT atomic lines files contains wavelength, log\_gf and other tabulated information for several
(element, ionization level) atomic species.

The following code calculates isolated atomic spectra for a list of arbitrarily chosen atomic species.

\begin{sphinxVerbatim}[commandchars=\\\{\}]
\PYG{l+s+sd}{\PYGZdq{}\PYGZdq{}\PYGZdq{}Runs synthesis for specified atomic species separately. No molecules or hydrogen lines.\PYGZdq{}\PYGZdq{}\PYGZdq{}}

\PYG{k+kn}{import} \PYG{n+nn}{f311}\PYG{n+nn}{.}\PYG{n+nn}{pyfant} \PYG{k}{as} \PYG{n+nn}{pf}
\PYG{k+kn}{import} \PYG{n+nn}{f311}\PYG{n+nn}{.}\PYG{n+nn}{explorer} \PYG{k}{as} \PYG{n+nn}{ex}
\PYG{k+kn}{import} \PYG{n+nn}{matplotlib}\PYG{n+nn}{.}\PYG{n+nn}{pyplot} \PYG{k}{as} \PYG{n+nn}{plt}
\PYG{k+kn}{import} \PYG{n+nn}{f311}\PYG{n+nn}{.}\PYG{n+nn}{filetypes} \PYG{k}{as} \PYG{n+nn}{ft}
\PYG{k+kn}{import} \PYG{n+nn}{a99}

\PYG{c+c1}{\PYGZsh{} [\PYGZdq{}\PYGZlt{}element name\PYGZgt{}\PYGZlt{}ionization level\PYGZgt{}\PYGZdq{}, ...]}
\PYG{n}{MY\PYGZus{}SPECIES} \PYG{o}{=} \PYG{p}{[}\PYG{l+s+s2}{\PYGZdq{}}\PYG{l+s+s2}{Fe1}\PYG{l+s+s2}{\PYGZdq{}}\PYG{p}{,} \PYG{l+s+s2}{\PYGZdq{}}\PYG{l+s+s2}{Fe2}\PYG{l+s+s2}{\PYGZdq{}}\PYG{p}{,} \PYG{l+s+s2}{\PYGZdq{}}\PYG{l+s+s2}{Ca1}\PYG{l+s+s2}{\PYGZdq{}}\PYG{p}{,} \PYG{l+s+s2}{\PYGZdq{}}\PYG{l+s+s2}{Ca2}\PYG{l+s+s2}{\PYGZdq{}}\PYG{p}{,} \PYG{l+s+s2}{\PYGZdq{}}\PYG{l+s+s2}{Na1}\PYG{l+s+s2}{\PYGZdq{}}\PYG{p}{,} \PYG{l+s+s2}{\PYGZdq{}}\PYG{l+s+s2}{Si1}\PYG{l+s+s2}{\PYGZdq{}}\PYG{p}{]}

\PYG{k}{if} \PYG{n+nv+vm}{\PYGZus{}\PYGZus{}name\PYGZus{}\PYGZus{}} \PYG{o}{==} \PYG{l+s+s2}{\PYGZdq{}}\PYG{l+s+s2}{\PYGZus{}\PYGZus{}main\PYGZus{}\PYGZus{}}\PYG{l+s+s2}{\PYGZdq{}}\PYG{p}{:}
    \PYG{n}{pf}\PYG{o}{.}\PYG{n}{copy\PYGZus{}star}\PYG{p}{(}\PYG{n}{starname}\PYG{o}{=}\PYG{l+s+s2}{\PYGZdq{}}\PYG{l+s+s2}{sun\PYGZhy{}grevesse\PYGZhy{}1996}\PYG{l+s+s2}{\PYGZdq{}}\PYG{p}{)}
    \PYG{n}{pf}\PYG{o}{.}\PYG{n}{link\PYGZus{}to\PYGZus{}data}\PYG{p}{(}\PYG{p}{)}

    \PYG{c+c1}{\PYGZsh{} Loads full atomic lines file}
    \PYG{n}{fatoms} \PYG{o}{=} \PYG{n}{ft}\PYG{o}{.}\PYG{n}{FileAtoms}\PYG{p}{(}\PYG{p}{)}
    \PYG{n}{fatoms}\PYG{o}{.}\PYG{n}{load}\PYG{p}{(}\PYG{p}{)}


    \PYG{n}{runnables} \PYG{o}{=} \PYG{p}{[}\PYG{p}{]}
    \PYG{k}{for} \PYG{n}{elem\PYGZus{}ioni} \PYG{o+ow}{in} \PYG{n}{MY\PYGZus{}SPECIES}\PYG{p}{:}
        \PYG{n}{atom} \PYG{o}{=} \PYG{n}{fatoms}\PYG{o}{.}\PYG{n}{find\PYGZus{}atom}\PYG{p}{(}\PYG{n}{elem\PYGZus{}ioni}\PYG{p}{)}

        \PYG{c+c1}{\PYGZsh{} Creates atomic lines file object containing only one atom}
        \PYG{n}{fatoms2} \PYG{o}{=} \PYG{n}{ft}\PYG{o}{.}\PYG{n}{FileAtoms}\PYG{p}{(}\PYG{p}{)}
        \PYG{n}{fatoms2}\PYG{o}{.}\PYG{n}{atoms} \PYG{o}{=} \PYG{p}{[}\PYG{n}{atom}\PYG{p}{]}

        \PYG{n}{ecombo} \PYG{o}{=} \PYG{n}{pf}\PYG{o}{.}\PYG{n}{Combo}\PYG{p}{(}\PYG{p}{)}
        \PYG{c+c1}{\PYGZsh{} Overrides file \PYGZdq{}atoms.dat\PYGZdq{} with in\PYGZhy{}memory object}
        \PYG{n}{ecombo}\PYG{o}{.}\PYG{n}{conf}\PYG{o}{.}\PYG{n}{file\PYGZus{}atoms} \PYG{o}{=} \PYG{n}{fatoms2}
        \PYG{n}{ecombo}\PYG{o}{.}\PYG{n}{conf}\PYG{o}{.}\PYG{n}{flag\PYGZus{}output\PYGZus{}to\PYGZus{}dir} \PYG{o}{=} \PYG{k+kc}{True}
        \PYG{n}{oo} \PYG{o}{=} \PYG{n}{ecombo}\PYG{o}{.}\PYG{n}{conf}\PYG{o}{.}\PYG{n}{opt}
        \PYG{c+c1}{\PYGZsh{} Assigns synthesis range to match atomic lines range}
        \PYG{n}{oo}\PYG{o}{.}\PYG{n}{llzero}\PYG{p}{,} \PYG{n}{oo}\PYG{o}{.}\PYG{n}{llfin} \PYG{o}{=} \PYG{n}{fatoms2}\PYG{o}{.}\PYG{n}{llzero}\PYG{p}{,} \PYG{n}{fatoms2}\PYG{o}{.}\PYG{n}{llfin}
        \PYG{c+c1}{\PYGZsh{} Turns off hydrogen lines}
        \PYG{n}{oo}\PYG{o}{.}\PYG{n}{no\PYGZus{}h} \PYG{o}{=} \PYG{k+kc}{True}
        \PYG{c+c1}{\PYGZsh{} Turns off molecular lines}
        \PYG{n}{oo}\PYG{o}{.}\PYG{n}{no\PYGZus{}molecules} \PYG{o}{=} \PYG{k+kc}{True}

        \PYG{n}{runnables}\PYG{o}{.}\PYG{n}{append}\PYG{p}{(}\PYG{n}{ecombo}\PYG{p}{)}

    \PYG{n}{pf}\PYG{o}{.}\PYG{n}{run\PYGZus{}parallel}\PYG{p}{(}\PYG{n}{runnables}\PYG{p}{)}

    \PYG{c+c1}{\PYGZsh{} Draws figure}
    \PYG{n}{f} \PYG{o}{=} \PYG{n}{plt}\PYG{o}{.}\PYG{n}{figure}\PYG{p}{(}\PYG{p}{)}
    \PYG{n}{a99}\PYG{o}{.}\PYG{n}{format\PYGZus{}BLB}\PYG{p}{(}\PYG{p}{)}
    \PYG{k}{for} \PYG{n}{i}\PYG{p}{,} \PYG{p}{(}\PYG{n}{title}\PYG{p}{,} \PYG{n}{ecombo}\PYG{p}{)} \PYG{o+ow}{in} \PYG{n+nb}{enumerate}\PYG{p}{(}\PYG{n+nb}{zip}\PYG{p}{(}\PYG{n}{MY\PYGZus{}SPECIES}\PYG{p}{,} \PYG{n}{runnables}\PYG{p}{)}\PYG{p}{)}\PYG{p}{:}
        \PYG{n}{ecombo}\PYG{o}{.}\PYG{n}{load\PYGZus{}result}\PYG{p}{(}\PYG{p}{)}
        \PYG{n}{plt}\PYG{o}{.}\PYG{n}{subplot}\PYG{p}{(}\PYG{l+m+mi}{2}\PYG{p}{,} \PYG{l+m+mi}{3}\PYG{p}{,} \PYG{n}{i}\PYG{o}{+}\PYG{l+m+mi}{1}\PYG{p}{)}
        \PYG{n}{ex}\PYG{o}{.}\PYG{n}{draw\PYGZus{}spectra\PYGZus{}overlapped}\PYG{p}{(}\PYG{p}{[}\PYG{n}{ecombo}\PYG{o}{.}\PYG{n}{result}\PYG{p}{[}\PYG{l+s+s2}{\PYGZdq{}}\PYG{l+s+s2}{spec}\PYG{l+s+s2}{\PYGZdq{}}\PYG{p}{]}\PYG{p}{]}\PYG{p}{,}
                                   \PYG{n}{setup}\PYG{o}{=}\PYG{n}{ex}\PYG{o}{.}\PYG{n}{PlotSpectrumSetup}\PYG{p}{(}\PYG{n}{flag\PYGZus{}xlabel}\PYG{o}{=}\PYG{n}{i}\PYG{o}{/}\PYG{l+m+mi}{3} \PYG{o}{\PYGZgt{}}\PYG{o}{=} \PYG{l+m+mi}{1}\PYG{p}{,} \PYG{n}{flag\PYGZus{}legend}\PYG{o}{=}\PYG{k+kc}{False}\PYG{p}{)}\PYG{p}{)}
        \PYG{n}{plt}\PYG{o}{.}\PYG{n}{title}\PYG{p}{(}\PYG{n}{title}\PYG{p}{)}

    \PYG{n}{K} \PYG{o}{=} \PYG{l+m+mf}{1.}
    \PYG{n}{a99}\PYG{o}{.}\PYG{n}{set\PYGZus{}figure\PYGZus{}size}\PYG{p}{(}\PYG{n}{plt}\PYG{o}{.}\PYG{n}{gcf}\PYG{p}{(}\PYG{p}{)}\PYG{p}{,} \PYG{l+m+mi}{1300}\PYG{o}{*}\PYG{n}{K}\PYG{p}{,} \PYG{l+m+mi}{740}\PYG{o}{*}\PYG{n}{K}\PYG{p}{)}
    \PYG{n}{plt}\PYG{o}{.}\PYG{n}{tight\PYGZus{}layout}\PYG{p}{(}\PYG{p}{)}
    \PYG{n}{plt}\PYG{o}{.}\PYG{n}{savefig}\PYG{p}{(}\PYG{l+s+s2}{\PYGZdq{}}\PYG{l+s+s2}{synthesis\PYGZhy{}atoms.png}\PYG{l+s+s2}{\PYGZdq{}}\PYG{p}{)}
    \PYG{n}{plt}\PYG{o}{.}\PYG{n}{show}\PYG{p}{(}\PYG{p}{)}
\end{sphinxVerbatim}

\begin{figure}[htbp]
\centering

\noindent\sphinxincludegraphics{{synthesis-atoms}.png}
\end{figure}


\subsection{Spectral synthesis - Separate molecules}
\label{\detokenize{pyfant:spectral-synthesis-separate-molecules}}
\begin{sphinxVerbatim}[commandchars=\\\{\}]
\PYG{l+s+sd}{\PYGZdq{}\PYGZdq{}\PYGZdq{}Runs synthesis for molecular species separately. No atomic nor hydrogen lines.\PYGZdq{}\PYGZdq{}\PYGZdq{}}

\PYG{k+kn}{import} \PYG{n+nn}{f311}\PYG{n+nn}{.}\PYG{n+nn}{pyfant} \PYG{k}{as} \PYG{n+nn}{pf}
\PYG{k+kn}{import} \PYG{n+nn}{f311}\PYG{n+nn}{.}\PYG{n+nn}{explorer} \PYG{k}{as} \PYG{n+nn}{ex}
\PYG{k+kn}{import} \PYG{n+nn}{matplotlib}\PYG{n+nn}{.}\PYG{n+nn}{pyplot} \PYG{k}{as} \PYG{n+nn}{plt}
\PYG{k+kn}{import} \PYG{n+nn}{f311}\PYG{n+nn}{.}\PYG{n+nn}{filetypes} \PYG{k}{as} \PYG{n+nn}{ft}
\PYG{k+kn}{import} \PYG{n+nn}{a99}

\PYG{n}{SUBPLOT\PYGZus{}NUM\PYGZus{}ROWS} \PYG{o}{=} \PYG{l+m+mi}{2}
\PYG{n}{SUBPLOT\PYGZus{}NUM\PYGZus{}COLS} \PYG{o}{=} \PYG{l+m+mi}{2}

\PYG{k}{if} \PYG{n+nv+vm}{\PYGZus{}\PYGZus{}name\PYGZus{}\PYGZus{}} \PYG{o}{==} \PYG{l+s+s2}{\PYGZdq{}}\PYG{l+s+s2}{\PYGZus{}\PYGZus{}main\PYGZus{}\PYGZus{}}\PYG{l+s+s2}{\PYGZdq{}}\PYG{p}{:}
    \PYG{n}{pf}\PYG{o}{.}\PYG{n}{copy\PYGZus{}star}\PYG{p}{(}\PYG{n}{starname}\PYG{o}{=}\PYG{l+s+s2}{\PYGZdq{}}\PYG{l+s+s2}{sun\PYGZhy{}grevesse\PYGZhy{}1996}\PYG{l+s+s2}{\PYGZdq{}}\PYG{p}{)}
    \PYG{n}{pf}\PYG{o}{.}\PYG{n}{link\PYGZus{}to\PYGZus{}data}\PYG{p}{(}\PYG{p}{)}

    \PYG{c+c1}{\PYGZsh{} Loads full molecular lines file}
    \PYG{n}{fmol} \PYG{o}{=} \PYG{n}{ft}\PYG{o}{.}\PYG{n}{FileMolecules}\PYG{p}{(}\PYG{p}{)}
    \PYG{n}{fmol}\PYG{o}{.}\PYG{n}{load}\PYG{p}{(}\PYG{p}{)}


    \PYG{n}{runnables} \PYG{o}{=} \PYG{p}{[}\PYG{p}{]}
    \PYG{k}{for} \PYG{n}{molecule} \PYG{o+ow}{in} \PYG{n}{fmol}\PYG{p}{:}
        \PYG{n}{fmol2} \PYG{o}{=} \PYG{n}{ft}\PYG{o}{.}\PYG{n}{FileMolecules}\PYG{p}{(}\PYG{p}{)}
        \PYG{n}{fmol2}\PYG{o}{.}\PYG{n}{molecules} \PYG{o}{=} \PYG{p}{[}\PYG{n}{molecule}\PYG{p}{]}

        \PYG{n}{ecombo} \PYG{o}{=} \PYG{n}{pf}\PYG{o}{.}\PYG{n}{Combo}\PYG{p}{(}\PYG{p}{)}
        \PYG{c+c1}{\PYGZsh{} Overrides file \PYGZdq{}molecules.dat\PYGZdq{} with in\PYGZhy{}memory object}
        \PYG{n}{ecombo}\PYG{o}{.}\PYG{n}{conf}\PYG{o}{.}\PYG{n}{file\PYGZus{}molecules} \PYG{o}{=} \PYG{n}{fmol2}
        \PYG{n}{ecombo}\PYG{o}{.}\PYG{n}{conf}\PYG{o}{.}\PYG{n}{flag\PYGZus{}output\PYGZus{}to\PYGZus{}dir} \PYG{o}{=} \PYG{k+kc}{True}
        \PYG{n}{oo} \PYG{o}{=} \PYG{n}{ecombo}\PYG{o}{.}\PYG{n}{conf}\PYG{o}{.}\PYG{n}{opt}
        \PYG{c+c1}{\PYGZsh{} Assigns synthesis range to match atomic lines range}
        \PYG{n}{oo}\PYG{o}{.}\PYG{n}{llzero}\PYG{p}{,} \PYG{n}{oo}\PYG{o}{.}\PYG{n}{llfin} \PYG{o}{=} \PYG{n}{fmol2}\PYG{o}{.}\PYG{n}{llzero}\PYG{p}{,} \PYG{n}{fmol2}\PYG{o}{.}\PYG{n}{llfin}
        \PYG{c+c1}{\PYGZsh{} Turns off hydrogen lines}
        \PYG{n}{oo}\PYG{o}{.}\PYG{n}{no\PYGZus{}h} \PYG{o}{=} \PYG{k+kc}{True}
        \PYG{c+c1}{\PYGZsh{} Turns off atomic lines}
        \PYG{n}{oo}\PYG{o}{.}\PYG{n}{no\PYGZus{}atoms} \PYG{o}{=} \PYG{k+kc}{True}
        \PYG{c+c1}{\PYGZsh{} Adjusts the wavelength step according to the calculation interval}
        \PYG{n}{oo}\PYG{o}{.}\PYG{n}{pas} \PYG{o}{=} \PYG{n+nb}{max}\PYG{p}{(}\PYG{l+m+mi}{1}\PYG{p}{,} \PYG{n+nb}{round}\PYG{p}{(}\PYG{n}{oo}\PYG{o}{.}\PYG{n}{llfin}\PYG{o}{*}\PYG{l+m+mf}{1.}\PYG{o}{/}\PYG{l+m+mi}{20000}\PYG{o}{/}\PYG{l+m+mf}{2.5}\PYG{p}{)}\PYG{o}{*}\PYG{l+m+mf}{2.5}\PYG{p}{)}
        \PYG{n}{oo}\PYG{o}{.}\PYG{n}{aint} \PYG{o}{=} \PYG{n+nb}{max}\PYG{p}{(}\PYG{l+m+mf}{50.}\PYG{p}{,} \PYG{n}{oo}\PYG{o}{.}\PYG{n}{pas}\PYG{p}{)}

        \PYG{n}{runnables}\PYG{o}{.}\PYG{n}{append}\PYG{p}{(}\PYG{n}{ecombo}\PYG{p}{)}

    \PYG{n}{pf}\PYG{o}{.}\PYG{n}{run\PYGZus{}parallel}\PYG{p}{(}\PYG{n}{runnables}\PYG{p}{)}

    \PYG{n}{num\PYGZus{}panels} \PYG{o}{=} \PYG{n}{SUBPLOT\PYGZus{}NUM\PYGZus{}COLS}\PYG{o}{*}\PYG{n}{SUBPLOT\PYGZus{}NUM\PYGZus{}ROWS}
    \PYG{n}{num\PYGZus{}molecules} \PYG{o}{=} \PYG{n+nb}{len}\PYG{p}{(}\PYG{n}{runnables}\PYG{p}{)}
    \PYG{n}{ifigure} \PYG{o}{=} \PYG{l+m+mi}{0}
    \PYG{n}{a99}\PYG{o}{.}\PYG{n}{format\PYGZus{}BLB}\PYG{p}{(}\PYG{p}{)}
    \PYG{k}{for} \PYG{n}{i} \PYG{o+ow}{in} \PYG{n+nb}{range}\PYG{p}{(}\PYG{n}{num\PYGZus{}molecules}\PYG{o}{+}\PYG{l+m+mi}{1}\PYG{p}{)}\PYG{p}{:}
        \PYG{n}{not\PYGZus{}first} \PYG{o}{=} \PYG{n}{i} \PYG{o}{\PYGZgt{}} \PYG{l+m+mi}{0}
        \PYG{n}{first\PYGZus{}panel\PYGZus{}of\PYGZus{}figure} \PYG{o}{=} \PYG{p}{(}\PYG{n}{i} \PYG{o}{/} \PYG{n}{num\PYGZus{}panels} \PYG{o}{\PYGZhy{}} \PYG{n+nb}{int}\PYG{p}{(}\PYG{n}{i} \PYG{o}{/} \PYG{n}{num\PYGZus{}panels}\PYG{p}{)}\PYG{p}{)} \PYG{o}{\PYGZlt{}} \PYG{l+m+mf}{0.01}
        \PYG{n}{is\PYGZus{}panel} \PYG{o}{=} \PYG{n}{i} \PYG{o}{\PYGZlt{}} \PYG{n}{num\PYGZus{}molecules}

        \PYG{k}{if} \PYG{n}{not\PYGZus{}first} \PYG{o+ow}{and} \PYG{p}{(}\PYG{o+ow}{not} \PYG{n}{is\PYGZus{}panel} \PYG{o+ow}{or} \PYG{n}{first\PYGZus{}panel\PYGZus{}of\PYGZus{}figure}\PYG{p}{)}\PYG{p}{:}
            \PYG{n}{plt}\PYG{o}{.}\PYG{n}{tight\PYGZus{}layout}\PYG{p}{(}\PYG{p}{)}
            \PYG{n}{K} \PYG{o}{=} \PYG{l+m+mf}{1.}
            \PYG{n}{a99}\PYG{o}{.}\PYG{n}{set\PYGZus{}figure\PYGZus{}size}\PYG{p}{(}\PYG{n}{plt}\PYG{o}{.}\PYG{n}{gcf}\PYG{p}{(}\PYG{p}{)}\PYG{p}{,} \PYG{l+m+mi}{1500} \PYG{o}{*} \PYG{n}{K}\PYG{p}{,} \PYG{l+m+mi}{740} \PYG{o}{*} \PYG{n}{K}\PYG{p}{)}
            \PYG{n}{plt}\PYG{o}{.}\PYG{n}{tight\PYGZus{}layout}\PYG{p}{(}\PYG{p}{)}
            \PYG{n}{filename\PYGZus{}fig} \PYG{o}{=}\PYG{l+s+s2}{\PYGZdq{}}\PYG{l+s+s2}{synthesis\PYGZhy{}molecules\PYGZhy{}}\PYG{l+s+si}{\PYGZob{}\PYGZcb{}}\PYG{l+s+s2}{.png}\PYG{l+s+s2}{\PYGZdq{}}\PYG{o}{.}\PYG{n}{format}\PYG{p}{(}\PYG{n}{ifigure}\PYG{p}{)}
            \PYG{n+nb}{print}\PYG{p}{(}\PYG{l+s+s2}{\PYGZdq{}}\PYG{l+s+s2}{Saving figure }\PYG{l+s+s2}{\PYGZsq{}}\PYG{l+s+si}{\PYGZob{}\PYGZcb{}}\PYG{l+s+s2}{\PYGZsq{}}\PYG{l+s+s2}{...}\PYG{l+s+s2}{\PYGZdq{}}\PYG{o}{.}\PYG{n}{format}\PYG{p}{(}\PYG{n}{filename\PYGZus{}fig}\PYG{p}{)}\PYG{p}{)}
            \PYG{n}{plt}\PYG{o}{.}\PYG{n}{savefig}\PYG{p}{(}\PYG{n}{filename\PYGZus{}fig}\PYG{p}{)}
            \PYG{n}{plt}\PYG{o}{.}\PYG{n}{close}\PYG{p}{(}\PYG{p}{)}
            \PYG{n}{ifigure} \PYG{o}{+}\PYG{o}{=} \PYG{l+m+mi}{1}

        \PYG{k}{if} \PYG{n}{first\PYGZus{}panel\PYGZus{}of\PYGZus{}figure} \PYG{o+ow}{and} \PYG{n}{is\PYGZus{}panel}\PYG{p}{:}
            \PYG{n}{plt}\PYG{o}{.}\PYG{n}{figure}\PYG{p}{(}\PYG{p}{)}

        \PYG{k}{if} \PYG{n}{is\PYGZus{}panel}\PYG{p}{:}
            \PYG{n}{ecombo} \PYG{o}{=} \PYG{n}{runnables}\PYG{p}{[}\PYG{n}{i}\PYG{p}{]}
            \PYG{n}{ecombo}\PYG{o}{.}\PYG{n}{load\PYGZus{}result}\PYG{p}{(}\PYG{p}{)}

            \PYG{n}{isubplot} \PYG{o}{=} \PYG{n}{i} \PYG{o}{\PYGZpc{}} \PYG{n}{num\PYGZus{}panels} \PYG{o}{+} \PYG{l+m+mi}{1}
            \PYG{n}{plt}\PYG{o}{.}\PYG{n}{subplot}\PYG{p}{(}\PYG{n}{SUBPLOT\PYGZus{}NUM\PYGZus{}ROWS}\PYG{p}{,} \PYG{n}{SUBPLOT\PYGZus{}NUM\PYGZus{}COLS}\PYG{p}{,} \PYG{n}{isubplot}\PYG{p}{)}
            \PYG{n}{ex}\PYG{o}{.}\PYG{n}{draw\PYGZus{}spectra\PYGZus{}overlapped}\PYG{p}{(}\PYG{p}{[}\PYG{n}{ecombo}\PYG{o}{.}\PYG{n}{result}\PYG{p}{[}\PYG{l+s+s2}{\PYGZdq{}}\PYG{l+s+s2}{spec}\PYG{l+s+s2}{\PYGZdq{}}\PYG{p}{]}\PYG{p}{]}\PYG{p}{,}
               \PYG{n}{setup}\PYG{o}{=}\PYG{n}{ex}\PYG{o}{.}\PYG{n}{PlotSpectrumSetup}\PYG{p}{(}\PYG{n}{flag\PYGZus{}xlabel}\PYG{o}{=}\PYG{n}{i}\PYG{o}{/}\PYG{l+m+mi}{3} \PYG{o}{\PYGZgt{}}\PYG{o}{=} \PYG{l+m+mi}{1}\PYG{p}{,} \PYG{n}{flag\PYGZus{}legend}\PYG{o}{=}\PYG{k+kc}{False}\PYG{p}{)}\PYG{p}{)}

            \PYG{n}{\PYGZus{}title} \PYG{o}{=} \PYG{n}{fmol}\PYG{p}{[}\PYG{n}{i}\PYG{p}{]}\PYG{o}{.}\PYG{n}{description}
            \PYG{k}{if} \PYG{l+s+s2}{\PYGZdq{}}\PYG{l+s+s2}{]}\PYG{l+s+s2}{\PYGZdq{}} \PYG{o+ow}{in} \PYG{n}{\PYGZus{}title}\PYG{p}{:}
                \PYG{n}{title} \PYG{o}{=} \PYG{n}{\PYGZus{}title}\PYG{p}{[}\PYG{p}{:}\PYG{n}{\PYGZus{}title}\PYG{o}{.}\PYG{n}{index}\PYG{p}{(}\PYG{l+s+s2}{\PYGZdq{}}\PYG{l+s+s2}{]}\PYG{l+s+s2}{\PYGZdq{}}\PYG{p}{)}\PYG{o}{+}\PYG{l+m+mi}{1}\PYG{p}{]}
            \PYG{k}{else}\PYG{p}{:}
                \PYG{n}{title} \PYG{o}{=} \PYG{n}{\PYGZus{}title}\PYG{p}{[}\PYG{p}{:}\PYG{l+m+mi}{20}\PYG{p}{]}
            \PYG{n}{plt}\PYG{o}{.}\PYG{n}{title}\PYG{p}{(}\PYG{n}{title}\PYG{p}{)}
\end{sphinxVerbatim}

\begin{figure}[htbp]
\centering

\noindent\sphinxincludegraphics{{synthesis-molecules-0}.png}
\end{figure}

\begin{figure}[htbp]
\centering

\noindent\sphinxincludegraphics{{synthesis-molecules-1}.png}
\end{figure}

etc.


\subsection{Gaussian profiles as nulbad outputs}
\label{\detokenize{pyfant:gaussian-profiles-as-nulbad-outputs}}
\sphinxcode{nulbad} is one of the Fortran executables of the PFANT package. It is the one that convolves
the synthetic spectrum calculated by \sphinxcode{pfant} with a Gaussian profile specified by a “fwhm” parameter.

\begin{sphinxVerbatim}[commandchars=\\\{\}]
\PYG{l+s+sd}{\PYGZdq{}\PYGZdq{}\PYGZdq{}}
\PYG{l+s+sd}{Nulbad\PYGZsq{}s \PYGZdq{}impulse response\PYGZdq{}}

\PYG{l+s+sd}{Saves \PYGZdq{}impulse\PYGZdq{} spectrum (just a spike at lambda=5000 angstrom) as \PYGZdq{}flux.norm\PYGZdq{},}
\PYG{l+s+sd}{then runs {}`nulbad{}` repeatedly to get a range of Gaussian profiles.}

\PYG{l+s+sd}{\PYGZdq{}\PYGZdq{}\PYGZdq{}}

\PYG{k+kn}{import} \PYG{n+nn}{f311}\PYG{n+nn}{.}\PYG{n+nn}{pyfant} \PYG{k}{as} \PYG{n+nn}{pf}
\PYG{k+kn}{import} \PYG{n+nn}{f311}\PYG{n+nn}{.}\PYG{n+nn}{explorer} \PYG{k}{as} \PYG{n+nn}{ex}
\PYG{k+kn}{import} \PYG{n+nn}{matplotlib}\PYG{n+nn}{.}\PYG{n+nn}{pyplot} \PYG{k}{as} \PYG{n+nn}{plt}
\PYG{k+kn}{import} \PYG{n+nn}{a99}
\PYG{k+kn}{import} \PYG{n+nn}{f311}\PYG{n+nn}{.}\PYG{n+nn}{filetypes} \PYG{k}{as} \PYG{n+nn}{ft}
\PYG{k+kn}{import} \PYG{n+nn}{numpy} \PYG{k}{as} \PYG{n+nn}{np}


\PYG{c+c1}{\PYGZsh{} FWHM (full width at half of maximum) of Gaussian profiles in angstrom}
\PYG{n}{FWHMS} \PYG{o}{=} \PYG{p}{[}\PYG{l+m+mf}{0.03}\PYG{p}{,} \PYG{l+m+mf}{0.06}\PYG{p}{,} \PYG{l+m+mf}{0.09}\PYG{p}{,} \PYG{l+m+mf}{0.12}\PYG{p}{,} \PYG{l+m+mf}{0.15}\PYG{p}{,} \PYG{l+m+mf}{0.20}\PYG{p}{,} \PYG{l+m+mf}{0.25}\PYG{p}{,} \PYG{l+m+mf}{0.3}\PYG{p}{]}

\PYG{k}{if} \PYG{n+nv+vm}{\PYGZus{}\PYGZus{}name\PYGZus{}\PYGZus{}} \PYG{o}{==} \PYG{l+s+s2}{\PYGZdq{}}\PYG{l+s+s2}{\PYGZus{}\PYGZus{}main\PYGZus{}\PYGZus{}}\PYG{l+s+s2}{\PYGZdq{}}\PYG{p}{:}
    \PYG{c+c1}{\PYGZsh{} Copies files main.dat and abonds.dat to local directory (for given star)}
    \PYG{n}{pf}\PYG{o}{.}\PYG{n}{copy\PYGZus{}star}\PYG{p}{(}\PYG{n}{starname}\PYG{o}{=}\PYG{l+s+s2}{\PYGZdq{}}\PYG{l+s+s2}{sun\PYGZhy{}grevesse\PYGZhy{}1996}\PYG{l+s+s2}{\PYGZdq{}}\PYG{p}{)}
    \PYG{c+c1}{\PYGZsh{} Creates symbolic links to all non\PYGZhy{}star\PYGZhy{}specific files}
    \PYG{n}{pf}\PYG{o}{.}\PYG{n}{link\PYGZus{}to\PYGZus{}data}\PYG{p}{(}\PYG{p}{)}

    \PYG{c+c1}{\PYGZsh{} \PYGZsh{} 1) Creates \PYGZdq{}impulse\PYGZdq{} spectrum}
    \PYG{n}{fsp} \PYG{o}{=} \PYG{n}{ft}\PYG{o}{.}\PYG{n}{FileSpectrumPfant}\PYG{p}{(}\PYG{p}{)}
    \PYG{n}{sp} \PYG{o}{=} \PYG{n}{ft}\PYG{o}{.}\PYG{n}{Spectrum}\PYG{p}{(}\PYG{p}{)}
    \PYG{n}{N} \PYG{o}{=} \PYG{l+m+mi}{2001}
    \PYG{n}{sp}\PYG{o}{.}\PYG{n}{x} \PYG{o}{=} \PYG{p}{(}\PYG{n}{np}\PYG{o}{.}\PYG{n}{arange}\PYG{p}{(}\PYG{l+m+mi}{0}\PYG{p}{,} \PYG{n}{N}\PYG{p}{,} \PYG{n}{dtype}\PYG{o}{=}\PYG{n+nb}{float}\PYG{p}{)}\PYG{o}{\PYGZhy{}}\PYG{p}{(}\PYG{n}{N}\PYG{o}{\PYGZhy{}}\PYG{l+m+mi}{1}\PYG{p}{)}\PYG{o}{/}\PYG{l+m+mi}{2}\PYG{p}{)}\PYG{o}{*}\PYG{l+m+mf}{0.001}\PYG{o}{+}\PYG{l+m+mi}{5000}
    \PYG{n}{sp}\PYG{o}{.}\PYG{n}{y} \PYG{o}{=} \PYG{n}{np}\PYG{o}{.}\PYG{n}{zeros}\PYG{p}{(}\PYG{p}{(}\PYG{n}{N}\PYG{p}{,}\PYG{p}{)}\PYG{p}{,} \PYG{n}{dtype}\PYG{o}{=}\PYG{n+nb}{float}\PYG{p}{)}
    \PYG{n}{sp}\PYG{o}{.}\PYG{n}{y}\PYG{p}{[}\PYG{n+nb}{int}\PYG{p}{(}\PYG{p}{(}\PYG{n}{N}\PYG{o}{\PYGZhy{}}\PYG{l+m+mi}{1}\PYG{p}{)}\PYG{o}{/}\PYG{l+m+mi}{2}\PYG{p}{)}\PYG{p}{]} \PYG{o}{=} \PYG{l+m+mf}{1.}

    \PYG{n}{fsp}\PYG{o}{.}\PYG{n}{spectrum} \PYG{o}{=} \PYG{n}{sp}
    \PYG{n}{fsp}\PYG{o}{.}\PYG{n}{save\PYGZus{}as}\PYG{p}{(}\PYG{l+s+s2}{\PYGZdq{}}\PYG{l+s+s2}{flux.norm}\PYG{l+s+s2}{\PYGZdq{}}\PYG{p}{)}

    \PYG{c+c1}{\PYGZsh{} \PYGZsh{} 2) Convolutions}
    \PYG{n}{spectra} \PYG{o}{=} \PYG{p}{[}\PYG{p}{]}
    \PYG{k}{for} \PYG{n}{fwhm} \PYG{o+ow}{in} \PYG{n}{FWHMS}\PYG{p}{:}
        \PYG{n}{enulbad} \PYG{o}{=} \PYG{n}{pf}\PYG{o}{.}\PYG{n}{Nulbad}\PYG{p}{(}\PYG{p}{)}
        \PYG{n}{enulbad}\PYG{o}{.}\PYG{n}{conf}\PYG{o}{.}\PYG{n}{opt}\PYG{o}{.}\PYG{n}{fwhm} \PYG{o}{=} \PYG{n}{fwhm}
        \PYG{n}{enulbad}\PYG{o}{.}\PYG{n}{run}\PYG{p}{(}\PYG{p}{)}
        \PYG{n}{enulbad}\PYG{o}{.}\PYG{n}{load\PYGZus{}result}\PYG{p}{(}\PYG{p}{)}
        \PYG{n}{enulbad}\PYG{o}{.}\PYG{n}{clean}\PYG{p}{(}\PYG{p}{)}
        \PYG{c+c1}{\PYGZsh{} Appends convolved spectrum for comparison}
        \PYG{n}{spectra}\PYG{o}{.}\PYG{n}{append}\PYG{p}{(}\PYG{n}{enulbad}\PYG{o}{.}\PYG{n}{result}\PYG{p}{[}\PYG{l+s+s2}{\PYGZdq{}}\PYG{l+s+s2}{convolved}\PYG{l+s+s2}{\PYGZdq{}}\PYG{p}{]}\PYG{p}{)}

    \PYG{c+c1}{\PYGZsh{} \PYGZsh{} 3) Plots}
    \PYG{n}{f} \PYG{o}{=} \PYG{n}{plt}\PYG{o}{.}\PYG{n}{figure}\PYG{p}{(}\PYG{p}{)}
    \PYG{n}{ex}\PYG{o}{.}\PYG{n}{draw\PYGZus{}spectra\PYGZus{}overlapped}\PYG{p}{(}\PYG{n}{spectra}\PYG{p}{)}
    \PYG{n}{K} \PYG{o}{=} \PYG{l+m+mf}{0.7}
    \PYG{n}{a99}\PYG{o}{.}\PYG{n}{set\PYGZus{}figure\PYGZus{}size}\PYG{p}{(}\PYG{n}{plt}\PYG{o}{.}\PYG{n}{gcf}\PYG{p}{(}\PYG{p}{)}\PYG{p}{,} \PYG{l+m+mi}{1300}\PYG{o}{*}\PYG{n}{K}\PYG{p}{,} \PYG{l+m+mi}{500}\PYG{o}{*}\PYG{n}{K}\PYG{p}{)}
    \PYG{n}{plt}\PYG{o}{.}\PYG{n}{tight\PYGZus{}layout}\PYG{p}{(}\PYG{p}{)}
    \PYG{n}{plt}\PYG{o}{.}\PYG{n}{savefig}\PYG{p}{(}\PYG{l+s+s2}{\PYGZdq{}}\PYG{l+s+s2}{gaussian\PYGZhy{}profiles.png}\PYG{l+s+s2}{\PYGZdq{}}\PYG{p}{)}
    \PYG{n}{plt}\PYG{o}{.}\PYG{n}{show}\PYG{p}{(}\PYG{p}{)}
\end{sphinxVerbatim}

\begin{figure}[htbp]
\centering

\noindent\sphinxincludegraphics{{gaussian-profiles}.png}
\end{figure}


\subsection{Plot hydrogen profiles}
\label{\detokenize{pyfant:plot-hydrogen-profiles}}
\begin{sphinxVerbatim}[commandchars=\\\{\}]
\PYG{l+s+sd}{\PYGZdq{}\PYGZdq{}\PYGZdq{}}
\PYG{l+s+sd}{Calculates hydrogen lines profiles, then plots them in several 3D subplots}
\PYG{l+s+sd}{\PYGZdq{}\PYGZdq{}\PYGZdq{}}

\PYG{k+kn}{import} \PYG{n+nn}{f311}\PYG{n+nn}{.}\PYG{n+nn}{pyfant} \PYG{k}{as} \PYG{n+nn}{pf}
\PYG{k+kn}{import} \PYG{n+nn}{f311}\PYG{n+nn}{.}\PYG{n+nn}{explorer} \PYG{k}{as} \PYG{n+nn}{ex}
\PYG{k+kn}{import} \PYG{n+nn}{f311}\PYG{n+nn}{.}\PYG{n+nn}{filetypes} \PYG{k}{as} \PYG{n+nn}{ft}
\PYG{k+kn}{import} \PYG{n+nn}{f311}\PYG{n+nn}{.}\PYG{n+nn}{physics} \PYG{k}{as} \PYG{n+nn}{ph}
\PYG{k+kn}{import} \PYG{n+nn}{a99}
\PYG{k+kn}{import} \PYG{n+nn}{os}
\PYG{k+kn}{import} \PYG{n+nn}{shutil}
\PYG{k+kn}{import} \PYG{n+nn}{matplotlib}\PYG{n+nn}{.}\PYG{n+nn}{pyplot} \PYG{k}{as} \PYG{n+nn}{plt}
\PYG{k+kn}{from} \PYG{n+nn}{mpl\PYGZus{}toolkits}\PYG{n+nn}{.}\PYG{n+nn}{mplot3d} \PYG{k}{import} \PYG{n}{Axes3D}  \PYG{c+c1}{\PYGZsh{} yes, required (see below)}

\PYG{k}{def} \PYG{n+nf}{mylog}\PYG{p}{(}\PYG{o}{*}\PYG{n}{args}\PYG{p}{)}\PYG{p}{:}
    \PYG{n+nb}{print}\PYG{p}{(}\PYG{l+s+s2}{\PYGZdq{}}\PYG{l+s+s2}{\PYGZca{}\PYGZca{} }\PYG{l+s+si}{\PYGZob{}\PYGZcb{}}\PYG{l+s+s2}{\PYGZdq{}}\PYG{o}{.}\PYG{n}{format}\PYG{p}{(}\PYG{l+s+s2}{\PYGZdq{}}\PYG{l+s+s2}{, }\PYG{l+s+s2}{\PYGZdq{}}\PYG{o}{.}\PYG{n}{join}\PYG{p}{(}\PYG{n}{args}\PYG{p}{)}\PYG{p}{)}\PYG{p}{)}


\PYG{k}{def} \PYG{n+nf}{main}\PYG{p}{(}\PYG{n}{flag\PYGZus{}cleanup}\PYG{o}{=}\PYG{k+kc}{True}\PYG{p}{)}\PYG{p}{:}
    \PYG{n}{tmpdir} \PYG{o}{=} \PYG{n}{a99}\PYG{o}{.}\PYG{n}{new\PYGZus{}filename}\PYG{p}{(}\PYG{l+s+s2}{\PYGZdq{}}\PYG{l+s+s2}{hydrogen\PYGZhy{}profiles}\PYG{l+s+s2}{\PYGZdq{}}\PYG{p}{)}

    \PYG{c+c1}{\PYGZsh{} Saves current directory}
    \PYG{n}{pwd} \PYG{o}{=} \PYG{n}{os}\PYG{o}{.}\PYG{n}{getcwd}\PYG{p}{(}\PYG{p}{)}
    \PYG{n}{mylog}\PYG{p}{(}\PYG{l+s+s2}{\PYGZdq{}}\PYG{l+s+s2}{Creating directory }\PYG{l+s+s2}{\PYGZsq{}}\PYG{l+s+si}{\PYGZob{}\PYGZcb{}}\PYG{l+s+s2}{\PYGZsq{}}\PYG{l+s+s2}{...}\PYG{l+s+s2}{\PYGZdq{}}\PYG{o}{.}\PYG{n}{format}\PYG{p}{(}\PYG{n}{tmpdir}\PYG{p}{)}\PYG{p}{)}
    \PYG{n}{os}\PYG{o}{.}\PYG{n}{mkdir}\PYG{p}{(}\PYG{n}{tmpdir}\PYG{p}{)}
    \PYG{k}{try}\PYG{p}{:}
        \PYG{n}{pf}\PYG{o}{.}\PYG{n}{link\PYGZus{}to\PYGZus{}data}\PYG{p}{(}\PYG{p}{)}
        \PYG{n}{\PYGZus{}main}\PYG{p}{(}\PYG{p}{)}
    \PYG{k}{finally}\PYG{p}{:}
        \PYG{c+c1}{\PYGZsh{} Restores current directory}
        \PYG{n}{os}\PYG{o}{.}\PYG{n}{chdir}\PYG{p}{(}\PYG{n}{pwd}\PYG{p}{)}
        \PYG{c+c1}{\PYGZsh{} Removes temporary directory}
        \PYG{k}{if} \PYG{n}{flag\PYGZus{}cleanup}\PYG{p}{:}
            \PYG{n}{mylog}\PYG{p}{(}\PYG{l+s+s2}{\PYGZdq{}}\PYG{l+s+s2}{Removing directory }\PYG{l+s+s2}{\PYGZsq{}}\PYG{l+s+si}{\PYGZob{}\PYGZcb{}}\PYG{l+s+s2}{\PYGZsq{}}\PYG{l+s+s2}{...}\PYG{l+s+s2}{\PYGZdq{}}\PYG{o}{.}\PYG{n}{format}\PYG{p}{(}\PYG{n}{tmpdir}\PYG{p}{)}\PYG{p}{)}
            \PYG{n}{shutil}\PYG{o}{.}\PYG{n}{rmtree}\PYG{p}{(}\PYG{n}{tmpdir}\PYG{p}{)}
        \PYG{k}{else}\PYG{p}{:}
            \PYG{n}{mylog}\PYG{p}{(}\PYG{l+s+s2}{\PYGZdq{}}\PYG{l+s+s2}{Not cleaning up.}\PYG{l+s+s2}{\PYGZdq{}}\PYG{p}{)}


\PYG{k}{def} \PYG{n+nf}{\PYGZus{}main}\PYG{p}{(}\PYG{p}{)}\PYG{p}{:}
    \PYG{n}{fm} \PYG{o}{=} \PYG{n}{ft}\PYG{o}{.}\PYG{n}{FileMain}\PYG{p}{(}\PYG{p}{)}
    \PYG{n}{fm}\PYG{o}{.}\PYG{n}{init\PYGZus{}default}\PYG{p}{(}\PYG{p}{)}
    \PYG{n}{fm}\PYG{o}{.}\PYG{n}{llzero}\PYG{p}{,} \PYG{n}{fm}\PYG{o}{.}\PYG{n}{llfin} \PYG{o}{=} \PYG{l+m+mf}{1000.}\PYG{p}{,} \PYG{l+m+mf}{200000.}  \PYG{c+c1}{\PYGZsh{} spectral synthesis range in Angstrom}

    \PYG{n}{ei} \PYG{o}{=} \PYG{n}{pf}\PYG{o}{.}\PYG{n}{Innewmarcs}\PYG{p}{(}\PYG{p}{)}
    \PYG{n}{ei}\PYG{o}{.}\PYG{n}{conf}\PYG{o}{.}\PYG{n}{file\PYGZus{}main} \PYG{o}{=} \PYG{n}{fm}
    \PYG{n}{ei}\PYG{o}{.}\PYG{n}{run}\PYG{p}{(}\PYG{p}{)}
    \PYG{n}{ei}\PYG{o}{.}\PYG{n}{clean}\PYG{p}{(}\PYG{p}{)}

    \PYG{n}{eh} \PYG{o}{=} \PYG{n}{pf}\PYG{o}{.}\PYG{n}{Hydro2}\PYG{p}{(}\PYG{p}{)}
    \PYG{n}{eh}\PYG{o}{.}\PYG{n}{conf}\PYG{o}{.}\PYG{n}{file\PYGZus{}main} \PYG{o}{=} \PYG{n}{fm}
    \PYG{n}{eh}\PYG{o}{.}\PYG{n}{run}\PYG{p}{(}\PYG{p}{)}
    \PYG{n}{eh}\PYG{o}{.}\PYG{n}{load\PYGZus{}result}\PYG{p}{(}\PYG{p}{)}
    \PYG{n}{eh}\PYG{o}{.}\PYG{n}{clean}\PYG{p}{(}\PYG{p}{)}

    \PYG{n}{\PYGZus{}plot\PYGZus{}profiles}\PYG{p}{(}\PYG{n}{eh}\PYG{o}{.}\PYG{n}{result}\PYG{p}{[}\PYG{l+s+s2}{\PYGZdq{}}\PYG{l+s+s2}{profiles}\PYG{l+s+s2}{\PYGZdq{}}\PYG{p}{]}\PYG{p}{)}


\PYG{k}{def} \PYG{n+nf}{\PYGZus{}plot\PYGZus{}profiles}\PYG{p}{(}\PYG{n}{profiles}\PYG{p}{)}\PYG{p}{:}
    \PYG{n}{fig} \PYG{o}{=} \PYG{n}{plt}\PYG{o}{.}\PYG{n}{figure}\PYG{p}{(}\PYG{p}{)}
    \PYG{n}{i} \PYG{o}{=} \PYG{l+m+mi}{0}
    \PYG{k}{for} \PYG{n}{filename}\PYG{p}{,} \PYG{n}{ftoh} \PYG{o+ow}{in} \PYG{n}{profiles}\PYG{o}{.}\PYG{n}{items}\PYG{p}{(}\PYG{p}{)}\PYG{p}{:}
        \PYG{k}{if} \PYG{n}{ftoh} \PYG{o+ow}{is} \PYG{o+ow}{not} \PYG{k+kc}{None}\PYG{p}{:}
            \PYG{n}{mylog}\PYG{p}{(}\PYG{l+s+s2}{\PYGZdq{}}\PYG{l+s+s2}{Drawing }\PYG{l+s+s2}{\PYGZsq{}}\PYG{l+s+si}{\PYGZob{}\PYGZcb{}}\PYG{l+s+s2}{\PYGZsq{}}\PYG{l+s+s2}{...}\PYG{l+s+s2}{\PYGZdq{}}\PYG{o}{.}\PYG{n}{format}\PYG{p}{(}\PYG{n}{filename}\PYG{p}{)}\PYG{p}{)}
            \PYG{c+c1}{\PYGZsh{} ax = plt.subplot(2, 3, i+1)}
            \PYG{n}{ax} \PYG{o}{=} \PYG{n}{fig}\PYG{o}{.}\PYG{n}{add\PYGZus{}subplot}\PYG{p}{(}\PYG{l+m+mi}{2}\PYG{p}{,} \PYG{l+m+mi}{3}\PYG{p}{,} \PYG{n}{i}\PYG{o}{+}\PYG{l+m+mi}{1}\PYG{p}{,} \PYG{n}{projection}\PYG{o}{=}\PYG{l+s+s1}{\PYGZsq{}}\PYG{l+s+s1}{3d}\PYG{l+s+s1}{\PYGZsq{}}\PYG{p}{)}
            \PYG{n}{ax}\PYG{o}{.}\PYG{n}{set\PYGZus{}title}\PYG{p}{(}\PYG{n}{filename}\PYG{p}{)}
            \PYG{n}{ex}\PYG{o}{.}\PYG{n}{draw\PYGZus{}toh}\PYG{p}{(}\PYG{n}{ftoh}\PYG{p}{,} \PYG{n}{ax}\PYG{p}{)}
            \PYG{n}{i} \PYG{o}{+}\PYG{o}{=} \PYG{l+m+mi}{1}

    \PYG{n}{plt}\PYG{o}{.}\PYG{n}{tight\PYGZus{}layout}\PYG{p}{(}\PYG{p}{)}
    \PYG{n}{plt}\PYG{o}{.}\PYG{n}{savefig}\PYG{p}{(}\PYG{l+s+s2}{\PYGZdq{}}\PYG{l+s+s2}{hydrogen\PYGZhy{}profiles.png}\PYG{l+s+s2}{\PYGZdq{}}\PYG{p}{)}
    \PYG{n}{plt}\PYG{o}{.}\PYG{n}{show}\PYG{p}{(}\PYG{p}{)}


\PYG{k}{if} \PYG{n+nv+vm}{\PYGZus{}\PYGZus{}name\PYGZus{}\PYGZus{}} \PYG{o}{==} \PYG{l+s+s2}{\PYGZdq{}}\PYG{l+s+s2}{\PYGZus{}\PYGZus{}main\PYGZus{}\PYGZus{}}\PYG{l+s+s2}{\PYGZdq{}}\PYG{p}{:}
    \PYG{n}{main}\PYG{p}{(}\PYG{n}{flag\PYGZus{}cleanup}\PYG{o}{=}\PYG{k+kc}{True}\PYG{p}{)}
\end{sphinxVerbatim}

\begin{figure}[htbp]
\centering

\noindent\sphinxincludegraphics{{hydrogen-profiles}.png}
\end{figure}


\section{API reference}
\label{\detokenize{pyfant:api-reference}}
\DUrole{xref,std,std-doc}{autodoc/f311.pyfant}


\chapter{Conversion of molecular lines lists}
\label{\detokenize{convmol:conversion-of-molecular-lines-lists}}\label{\detokenize{convmol::doc}}

\section{Introduction}
\label{\detokenize{convmol:introduction}}
Conversion between different formats of files containing molecular spectral lines data.

Conversion inputs:
\begin{itemize}
\item {} 
Robert Kurucz molecular line lists (fully implemented (old and new Kurucz format)) {[}Kurucz{]}

\item {} 
HITRAN Online database (partially implemented)

\item {} 
VALD3 (to do)

\item {} 
TurboSpectrum (to do)

\end{itemize}

\begin{sphinxadmonition}{note}{Note:}
This package is currently under construction (2017-11-13)
\end{sphinxadmonition}

Conversion output:
\begin{itemize}
\item {} 
PFANT molecular lines file (such as “molecules.dat”)

\end{itemize}


\section{Most relevant applications in F311 package}
\label{\detokenize{convmol:most-relevant-applications-in-f311-package}}

\subsection{Graphical applications}
\label{\detokenize{convmol:graphical-applications}}\begin{itemize}
\item {} 
{\hyperref[\detokenize{autoscripts/script-convmol::doc}]{\sphinxcrossref{\DUrole{doc}{convmol.py}}}}: Conversion of molecular lines data to PFANT format

\item {} 
{\hyperref[\detokenize{autoscripts/script-mced::doc}]{\sphinxcrossref{\DUrole{doc}{mced.py}}}}: Editor for molecular constants file

\item {} 
{\hyperref[\detokenize{autoscripts/script-moldbed::doc}]{\sphinxcrossref{\DUrole{doc}{moldbed.py}}}}: Editor for molecules SQLite database

\end{itemize}


\subsection{Command-line tools}
\label{\detokenize{convmol:command-line-tools}}\begin{itemize}
\item {} 
{\hyperref[\detokenize{autoscripts/script-hitran-scraper::doc}]{\sphinxcrossref{\DUrole{doc}{hitran-scraper.py}}}}: Retrieves molecular lines from the HITRAN database {[}Gordon2016{]}

\item {} 
{\hyperref[\detokenize{autoscripts/script-nist-scraper::doc}]{\sphinxcrossref{\DUrole{doc}{nist-scraper.py}}}}: Retrieves and prints a table of molecular constants from the NIST Chemistry Web Book.

\end{itemize}


\section{How the conversion is made}
\label{\detokenize{convmol:how-the-conversion-is-made}}

\subsection{Input molecular constants obtained from NIST database {[}NISTref{]} (all given in unit: cm**-1)}
\label{\detokenize{convmol:input-molecular-constants-obtained-from-nist-database-nistref-all-given-in-unit-cm-1}}\begin{itemize}
\item {} 
\sphinxstyleemphasis{omega\_e}: vibrational constant \textendash{} first term

\item {} 
\sphinxstyleemphasis{omega\_ex\_e}: vibrational constant \textendash{} second term

\item {} 
\sphinxstyleemphasis{omega\_ey\_e}: vibrational constant \textendash{} third term

\item {} 
\sphinxstyleemphasis{B\_e}: rotational constant in equilibrium position

\item {} 
\sphinxstyleemphasis{alpha\_e}: rotational constant \textendash{} first term

\item {} 
\sphinxstyleemphasis{D\_e}: centrifugal distortion constant

\item {} 
\sphinxstyleemphasis{beta\_e}: rotational constant \textendash{} first term, centrifugal force

\item {} 
\sphinxstyleemphasis{A}: Coupling counstant

\item {} 
\sphinxstyleemphasis{M2l}: multiplicity of the initial state (1 for singlet, 2 for doublet, 3 for triplet and so on)

\item {} 
\sphinxstyleemphasis{M2l}: multiplicity of the final state

\item {} 
\sphinxstyleemphasis{LambdaL}: ?SPDF? of the initial state (0 for Sigma, 1 for Pi, 2 for Delta, 3 for Phi)

\item {} 
\sphinxstyleemphasis{Lambda2L}: ?SPDF? of the initial state

\end{itemize}

\begin{sphinxadmonition}{hint}{Hint:}
These values were downloaded from NIST for several molecules and can be navigated through in the applications \sphinxcode{convmol.py} or \sphinxcode{mced.py}.

Molecular constants can be downloaded from NIST using script \sphinxcode{nist-scraper.py}
\end{sphinxadmonition}


\subsection{Input data from line list files (\sphinxstyleemphasis{e.g.} {[}Kurucz{]})}
\label{\detokenize{convmol:input-data-from-line-list-files-e-g-kurucz}}\begin{itemize}
\item {} 
\sphinxstyleemphasis{lambda}: wavelength (angstrom)

\item {} 
\sphinxstyleemphasis{vl}: vibrational quantum number of the initial state

\item {} 
\sphinxstyleemphasis{v2l}: vibrational quantum number of the final state

\item {} 
\sphinxstyleemphasis{spinl}

\item {} 
\sphinxstyleemphasis{spin2l}

\item {} 
\sphinxstyleemphasis{JL}: rotational quantum number of the initial state

\item {} 
\sphinxstyleemphasis{J2l}: rotational quantum number of the final state

\end{itemize}


\subsection{Calculated outputs}
\label{\detokenize{convmol:calculated-outputs}}
The following values are calculated using application \sphinxcode{convmol.py} and stored as a PFANT molecular lines file (such as “molecules.dat”).


\subsubsection{\sphinxstyleemphasis{Jl}/\sphinxstyleemphasis{J2l}-independent}
\label{\detokenize{convmol:jl-j2l-independent}}\begin{itemize}
\item {} 
\sphinxstyleemphasis{qv}: Franck-Condon factor

\item {} 
\sphinxstyleemphasis{Bv}: rotational constant

\item {} 
\sphinxstyleemphasis{Dv}: rotational constant

\item {} 
\sphinxstyleemphasis{Gv}: rotational constant

\end{itemize}

These terms are calculated as follows:

\begin{sphinxVerbatim}[commandchars=\\\{\}]
\PYG{n}{qv} \PYG{o}{=} \PYG{n}{qv}\PYG{p}{(}\PYG{n}{molecule}\PYG{p}{,} \PYG{n}{system}\PYG{p}{,} \PYG{n}{vl}\PYG{p}{,} \PYG{n}{v2l}\PYG{p}{)} \PYG{o+ow}{is} \PYG{n}{calculated} \PYG{n}{using} \PYG{n}{code} \PYG{n}{by} \PYG{n}{Singh} \PYG{p}{[}\PYG{n}{Sing1998}\PYG{p}{]}\PYG{o}{.}
                                   \PYG{n}{The} \PYG{n}{Franck}\PYG{o}{\PYGZhy{}}\PYG{n}{Condon} \PYG{n}{factors} \PYG{n}{were} \PYG{n}{already} \PYG{n}{calculate} \PYG{k}{for} \PYG{n}{several}
                                   \PYG{n}{different} \PYG{n}{molecules} \PYG{o+ow}{and} \PYG{n}{are} \PYG{n}{tabulated} \PYG{n}{inside} \PYG{n}{file} \PYG{l+s+s2}{\PYGZdq{}}\PYG{l+s+s2}{moldb.sqlite}\PYG{l+s+s2}{\PYGZdq{}}

\PYG{n}{Bv} \PYG{o}{=} \PYG{n}{B\PYGZus{}e} \PYG{o}{\PYGZhy{}} \PYG{n}{alpha\PYGZus{}e} \PYG{o}{*} \PYG{p}{(}\PYG{n}{v2l} \PYG{o}{+} \PYG{l+m+mf}{0.5}\PYG{p}{)}

\PYG{n}{Dv} \PYG{o}{=} \PYG{p}{(}\PYG{n}{D\PYGZus{}e} \PYG{o}{+} \PYG{n}{beta\PYGZus{}e} \PYG{o}{*} \PYG{p}{(}\PYG{n}{v2l} \PYG{o}{+} \PYG{l+m+mf}{0.5}\PYG{p}{)}\PYG{p}{)} \PYG{o}{*} \PYG{l+m+mf}{1.0e+06}

\PYG{n}{Gv} \PYG{o}{=} \PYG{n}{omega\PYGZus{}e} \PYG{o}{*} \PYG{p}{(}\PYG{n}{v2l} \PYG{o}{+} \PYG{l+m+mf}{0.5}\PYG{p}{)} \PYG{o}{\PYGZhy{}} \PYG{n}{omega\PYGZus{}ex\PYGZus{}e} \PYG{o}{*} \PYG{p}{(}\PYG{n}{v2l} \PYG{o}{+} \PYG{l+m+mf}{0.5}\PYG{p}{)} \PYG{o}{*}\PYG{o}{*} \PYG{l+m+mi}{2} \PYG{o}{+} \PYG{n}{omega\PYGZus{}ey\PYGZus{}e} \PYG{o}{*} \PYG{p}{(}\PYG{n}{v2l} \PYG{o}{+} \PYG{l+m+mf}{0.5}\PYG{p}{)} \PYG{o}{*}\PYG{o}{*} \PYG{l+m+mi}{3} \PYG{o}{\PYGZhy{}}
     \PYG{n}{omega\PYGZus{}e} \PYG{o}{/} \PYG{l+m+mf}{2.0} \PYG{o}{\PYGZhy{}} \PYG{n}{omega\PYGZus{}ex\PYGZus{}e} \PYG{o}{/} \PYG{l+m+mf}{4.0} \PYG{o}{+} \PYG{n}{omega\PYGZus{}ey\PYGZus{}e} \PYG{o}{/} \PYG{l+m+mf}{8.0}
\end{sphinxVerbatim}


\subsubsection{\sphinxstyleemphasis{Jl}/\sphinxstyleemphasis{J2l}-dependent (\sphinxstyleemphasis{i.e.}, for each spectral line)}
\label{\detokenize{convmol:jl-j2l-dependent-i-e-for-each-spectral-line}}\begin{itemize}
\item {} 
\sphinxstyleemphasis{LS}: line strength for given by formulas in {[}Kovacs1969{]}, Chapter 3; Hönl-London factor

\item {} 
\sphinxstyleemphasis{S}: normalized line strength

\end{itemize}

\sphinxstyleemphasis{LS} is calculated using a different formula depending on:
\begin{enumerate}
\item {} 
the multiplicities of the transition (currently implemented only cases where the initial and
final state have same multiplicity)

\item {} 
the value and/or sign of (\sphinxstyleemphasis{DeltaLambda} = \sphinxstyleemphasis{LambdaL} - \sphinxstyleemphasis{Lambda2l});

\item {} 
whether \sphinxstyleemphasis{A} is a positive or negative number;

\item {} 
the branch of the spectral line (see below how to determine the branch)

\end{enumerate}

So:

\begin{sphinxVerbatim}[commandchars=\\\{\}]
\PYG{n}{formula} \PYG{o}{=} \PYG{n}{KovacsFormula}\PYG{p}{(}\PYG{n}{i}\PYG{p}{,} \PYG{n}{ii}\PYG{p}{,} \PYG{n}{iii}\PYG{p}{,} \PYG{n}{iv}\PYG{p}{)}

\PYG{n}{LS} \PYG{o}{=} \PYG{n}{formula}\PYG{p}{(}\PYG{n}{almost} \PYG{n}{every} \PYG{n+nb}{input} \PYG{n}{variable}\PYG{p}{)}
\end{sphinxVerbatim}

\begin{sphinxadmonition}{hint}{Hint:}
All the line strength formulas and logic to determine which formula to use are
in module \sphinxcode{f311.physics.multiplicity}. The latter contains references to the formulas and
tables from {[}Kovacs{]} that were used for each specific (i, ii, iii, iv) case.
\end{sphinxadmonition}

\begin{sphinxadmonition}{note}{\label{convmol:index-0}Todo:}
Explain term formulas “u+/-“, “c+/-“
\end{sphinxadmonition}


\paragraph{Normalization of the line strength}
\label{\detokenize{convmol:normalization-of-the-line-strength}}
Normalization is applied so that, for a given \sphinxstyleemphasis{J2l},:

\begin{sphinxVerbatim}[commandchars=\\\{\}]
\PYG{n+nb}{sum}\PYG{p}{(}\PYG{p}{[}\PYG{n}{S}\PYG{p}{[}\PYG{n}{branch}\PYG{p}{]} \PYG{k}{for} \PYG{n}{branch} \PYG{o+ow}{in} \PYG{n}{all\PYGZus{}branches}\PYG{p}{]}\PYG{p}{)} \PYG{o}{==} \PYG{l+m+mi}{1}
\end{sphinxVerbatim}

To achieve this:

\begin{sphinxVerbatim}[commandchars=\\\{\}]
\PYG{n}{S} \PYG{o}{=} \PYG{n}{LS} \PYG{o}{*} \PYG{l+m+mf}{2.} \PYG{o}{/} \PYG{p}{(}\PYG{p}{(}\PYG{l+m+mi}{2} \PYG{o}{*} \PYG{n}{spin2l} \PYG{o}{+} \PYG{l+m+mi}{1}\PYG{p}{)} \PYG{o}{*} \PYG{p}{(}\PYG{l+m+mi}{2} \PYG{o}{*} \PYG{n}{J2l} \PYG{o}{+} \PYG{l+m+mi}{1}\PYG{p}{)} \PYG{o}{*} \PYG{p}{(}\PYG{l+m+mi}{2} \PYG{o}{\PYGZhy{}} \PYG{n}{delta\PYGZus{}k}\PYG{p}{)}\PYG{p}{)}
\end{sphinxVerbatim}

Where:

\begin{sphinxVerbatim}[commandchars=\\\{\}]
\PYG{n}{spin2l} \PYG{o}{=} \PYG{p}{(}\PYG{n}{M2l}\PYG{o}{\PYGZhy{}}\PYG{l+m+mi}{1}\PYG{p}{)}\PYG{o}{/}\PYG{l+m+mi}{2}
\end{sphinxVerbatim}


\paragraph{How to determine the branch}
\label{\detokenize{convmol:how-to-determine-the-branch}}
The branch “label” follows one of the following conventions:

\begin{sphinxVerbatim}[commandchars=\\\{\}]
\PYG{n}{singlets}\PYG{p}{:} \PYG{n}{branch} \PYG{n}{consists} \PYG{n}{of} \PYG{n}{a} \PYG{l+s+s2}{\PYGZdq{}}\PYG{l+s+s2}{\PYGZlt{}letter\PYGZgt{}}\PYG{l+s+s2}{\PYGZdq{}}\PYG{p}{,} \PYG{n}{where} \PYG{n}{letter} \PYG{n}{may} \PYG{n}{be} \PYG{n}{either} \PYG{l+s+s2}{\PYGZdq{}}\PYG{l+s+s2}{P}\PYG{l+s+s2}{\PYGZdq{}}\PYG{p}{,} \PYG{l+s+s2}{\PYGZdq{}}\PYG{l+s+s2}{Q}\PYG{l+s+s2}{\PYGZdq{}}\PYG{p}{,} \PYG{o+ow}{or} \PYG{l+s+s2}{\PYGZdq{}}\PYG{l+s+s2}{R}\PYG{l+s+s2}{\PYGZdq{}}

\PYG{n}{doublets}\PYG{p}{,} \PYG{n}{triplets} \PYG{n}{etc}\PYG{p}{:}

    \PYG{k}{if} \PYG{n}{spin} \PYG{o}{==} \PYG{n}{spinl} \PYG{o}{==} \PYG{n}{spin2l}\PYG{p}{:} \PYG{n}{branch} \PYG{n}{consists} \PYG{n}{of} \PYG{l+s+s2}{\PYGZdq{}}\PYG{l+s+s2}{\PYGZlt{}letter\PYGZgt{}\PYGZlt{}spin\PYGZgt{}}\PYG{l+s+s2}{\PYGZdq{}}

    \PYG{k}{if} \PYG{n}{spinl} \PYG{o}{\PYGZlt{}}\PYG{o}{\PYGZgt{}} \PYG{n}{spin2l}\PYG{p}{:} \PYG{n}{branch} \PYG{n}{consists} \PYG{n}{of} \PYG{l+s+s2}{\PYGZdq{}}\PYG{l+s+s2}{\PYGZlt{}letter\PYGZgt{}\PYGZlt{}spinl\PYGZgt{}\PYGZlt{}spin2l\PYGZgt{}}\PYG{l+s+s2}{\PYGZdq{}}
\end{sphinxVerbatim}

The branch letter is determined as follows:

\begin{sphinxVerbatim}[commandchars=\\\{\}]
\PYG{k}{if} \PYG{n}{Jl} \PYG{o}{\PYGZlt{}} \PYG{n}{J2l}\PYG{p}{:}  \PYG{l+s+s2}{\PYGZdq{}}\PYG{l+s+s2}{P}\PYG{l+s+s2}{\PYGZdq{}}
\PYG{k}{if} \PYG{n}{Jl} \PYG{o}{==} \PYG{n}{J2l}\PYG{p}{:} \PYG{l+s+s2}{\PYGZdq{}}\PYG{l+s+s2}{Q}\PYG{l+s+s2}{\PYGZdq{}}
\PYG{k}{if} \PYG{n}{Jl} \PYG{o}{\PYGZgt{}} \PYG{n}{J2l}\PYG{p}{:}  \PYG{l+s+s2}{\PYGZdq{}}\PYG{l+s+s2}{R}\PYG{l+s+s2}{\PYGZdq{}}
\end{sphinxVerbatim}


\section{API documentation}
\label{\detokenize{convmol:api-documentation}}
\DUrole{xref,std,std-doc}{autodoc/f311.convmol}


\section{Bibliography}
\label{\detokenize{convmol:bibliography}}
\sphinxstylestrong{{[}Kovacs1969{]}} Istvan Kovacs, Rotational Structure in the spectra of diatomic molecules. American Elsevier, 1969

\sphinxstylestrong{{[}Sing1998{]}} unpublished

\sphinxstylestrong{{[}NISTRef{]}} \sphinxurl{http://webbook.nist.gov/chemistry/}

\sphinxstylestrong{{[}Kurucz{]}} \sphinxurl{http://kurucz.harvard.edu/molecules.html}


\chapter{File explorer and editors}
\label{\detokenize{explorer::doc}}\label{\detokenize{explorer:file-explorer-and-editors}}

\section{Introduction}
\label{\detokenize{explorer:introduction}}
File edit \& visualization, including file-explorer-like \sphinxcode{explorer.py} (\hyperref[\detokenize{explorer:figexplorer}]{Figure \ref{\detokenize{explorer:figexplorer}}}).
\phantomsection\label{\detokenize{explorer:figexplorer}}
\begin{figure}[htbp]
\centering

\noindent\sphinxincludegraphics{{explorer}.png}
\label{\detokenize{explorer:figexplorer}}\end{figure}


\section{List of applications}
\label{\detokenize{explorer:list-of-applications}}
\begin{sphinxVerbatim}[commandchars=\\\{\}]
\PYG{n}{programs}\PYG{o}{.}\PYG{n}{py} \PYG{o}{\PYGZhy{}}\PYG{n}{p} \PYG{n}{explorer}
\end{sphinxVerbatim}

Graphical applications:
\begin{itemize}
\item {} 
\sphinxcode{abed.py} \textendash{} Abundances file editor

\item {} 
\sphinxcode{ated.py} \textendash{} Atomic lines file editor

\item {} 
\sphinxcode{cubeed.py} \textendash{} Data Cube Editor, import/export WebSim-COMPASS data cubes

\item {} 
\sphinxcode{explorer.py} \textendash{} F311 Explorer \textendash{}  list, visualize, and edit data files (\sphinxstyleemphasis{à la} File Manager)

\item {} 
\sphinxcode{mained.py} \textendash{} Main configuration file editor.

\item {} 
\sphinxcode{mled.py} \textendash{} Molecular lines file editor.

\item {} 
\sphinxcode{splisted.py} \textendash{} Spectrum List Editor

\item {} 
\sphinxcode{tune-zinf.py} \textendash{} Tunes the “zinf” parameter for each atomic line in atomic lines file

\end{itemize}

Command-line tools:
\begin{itemize}
\item {} 
\sphinxcode{create-grid.py} \textendash{} Merges several atmospheric models into a single file (\_i.e.\_, the “grid”)

\item {} 
\sphinxcode{cut-atoms.py} \textendash{} Cuts atomic lines file to wavelength interval specified

\item {} 
\sphinxcode{cut-molecules.py} \textendash{} Cuts molecular lines file to wavelength interval specified

\item {} 
\sphinxcode{cut-spectrum.py} \textendash{} Cuts spectrum file to wavelength interval specified

\item {} 
\sphinxcode{plot-spectra.py} \textendash{} Plots spectra on screen or creates PDF file

\item {} 
\sphinxcode{vald3-to-atoms.py} \textendash{} Converts VALD3 atomic/molecular lines file to PFANT atomic lines file.

\end{itemize}


\section{API reference}
\label{\detokenize{explorer:api-reference}}
\DUrole{xref,std,std-doc}{autodoc/f311.explorer}


\chapter{File handling API}
\label{\detokenize{filetypes::doc}}\label{\detokenize{filetypes:file-handling-api}}

\section{Introduction}
\label{\detokenize{filetypes:introduction}}
\sphinxstyleemphasis{f311.filetypes} represents most of the non-visual essence of the F311 project.
the. That package has classes to handle many different file formats used in Astronomy.

All classes descend from \sphinxcode{DataFile} containing some basic methods:
\begin{itemize}
\item {} 
\sphinxcode{load()}: loads file from disk into internal object variables

\item {} 
\sphinxcode{save\_as()}: saves file to disk

\item {} 
\sphinxcode{init\_default()}: initializes file object with default information

\end{itemize}


\section{Supported file types}
\label{\detokenize{filetypes:supported-file-types}}
The following table was generated in 12/Nov/2017. The “Editor” column shows the applications in
the F311 project that can handle these files (\sphinxstyleemphasis{i.e.}, load/edit/save).

All file types are also recognized by \sphinxcode{explorer.py}.


\begin{savenotes}\sphinxatlongtablestart\begin{longtable}{|l|l|l|l|}
\hline
\sphinxstylethead{\sphinxstyletheadfamily 
Description
\unskip}\relax &\sphinxstylethead{\sphinxstyletheadfamily 
Default filename
\unskip}\relax &\sphinxstylethead{\sphinxstyletheadfamily 
Class name
\unskip}\relax &\sphinxstylethead{\sphinxstyletheadfamily 
Editors
\unskip}\relax \\
\hline
\endfirsthead

\multicolumn{4}{c}%
{\makebox[0pt]{\sphinxtablecontinued{\tablename\ \thetable{} -- continued from previous page}}}\\
\hline
\sphinxstylethead{\sphinxstyletheadfamily 
Description
\unskip}\relax &\sphinxstylethead{\sphinxstyletheadfamily 
Default filename
\unskip}\relax &\sphinxstylethead{\sphinxstyletheadfamily 
Class name
\unskip}\relax &\sphinxstylethead{\sphinxstyletheadfamily 
Editors
\unskip}\relax \\
\hline
\endhead

\hline
\multicolumn{4}{r}{\makebox[0pt][r]{\sphinxtablecontinued{Continued on next page}}}\\
\endfoot

\endlastfoot

“Lambda-flux” Spectrum (2-column text file)
&&
FileSpectrumXY
&
\sphinxcode{splisted.py}
\\
\hline
Atmospheric model or grid of models (with opacities
 included)
&
grid.moo
&
FileMoo
&\\
\hline
Configuration file for molecular lines conversion GUI
 (Python code)
&
configconvmol.py
&
FileConfigConvMol
&\\
\hline
Database of Molecular Constants
&
moldb.sqlite
&
FileMolDB
&
\sphinxcode{convmol.py}, \sphinxcode{moldbed.py}
\\
\hline
FITS Sparse Data Cube (storage to take less disk space)
&
default.sparsecube
&
FileSparseCube
&\\
\hline
FITS Spectrum
&&
FileSpectrumFits
&
\sphinxcode{splisted.py}
\\
\hline
FITS Spectrum List
&
default.splist
&
FileSpectrumList
&
\sphinxcode{splisted.py}
\\
\hline
FITS WebSim Compass Data Cube
&
default.fullcube
&
FileFullCube
&
\sphinxcode{cubeed.py}
\\
\hline
FITS file with frames named INPUT\_*, MODEL\_*,
 RESIDUAL\_* (Galfit software output)
&&
FileGalfit
&\\
\hline
File containing Franck-Condon Factors (FCFs)
&&
FileFCF
&\\
\hline
HITRAN Molecules Catalogue
&
hitrandb.sqlite
&
FileHitranDB
&\\
\hline
Kurucz molecular lines file
&&
FileKuruczMolecule
&\\
\hline
Kurucz molecular lines file, old format \#0
&&
FileKuruczMoleculeOld
&\\
\hline
Kurucz molecular lines file, old format \#1
&&
FileKuruczMoleculeOld1
&\\
\hline
MARCS “.opa” (opacity model) file format.
&
modeles.opa
&
FileOpa
&\\
\hline
MARCS Atmospheric Model (text file)
&&
FileModTxt
&\\
\hline
Molecular constants config file (Python code)
&
configmolconsts.py
&
FileMolConsts
&
\sphinxcode{mced.py}
\\
\hline
PFANT “Absoru2” file
&
absoru2.dat
&
FileAbsoru2
&\\
\hline
PFANT Atmospheric Model (binary file)
&
modeles.mod
&
FileModBin
&\\
\hline
PFANT Atomic Lines
&
atoms.dat
&
FileAtoms
&
\sphinxcode{ated.py}
\\
\hline
PFANT Command-line Options
&
options.py
&
FileOptions
&
\sphinxcode{x.py}
\\
\hline
PFANT Hydrogen Line Profile
&
thalpha
&
FileToH
&\\
\hline
PFANT Hygrogen Lines Map
&
hmap.dat
&
FileHmap
&\\
\hline
PFANT Main Stellar Configuration
&
main.dat
&
FileMain
&
\sphinxcode{mained.py}, \sphinxcode{x.py}
\\
\hline
PFANT Molecular Lines
&
molecules.dat
&
FileMolecules
&
\sphinxcode{mled.py}
\\
\hline
PFANT Partition Function
&
partit.dat
&
FilePartit
&\\
\hline
PFANT Spectrum (\sphinxtitleref{nulbad} output)
&&
FileSpectrumNulbad
&
\sphinxcode{splisted.py}
\\
\hline
PFANT Spectrum (\sphinxtitleref{pfant} output)
&
flux.norm
&
FileSpectrumPfant
&
\sphinxcode{splisted.py}
\\
\hline
PFANT Stellar Chemical Abundances
&
abonds.dat
&
FileAbonds
&
\sphinxcode{abed.py}, \sphinxcode{x.py}
\\
\hline
PFANT Stellar Dissociation Equilibrium Information
&
dissoc.dat
&
FileDissoc
&\\
\hline
Plez molecular lines file, TiO format
&&
FilePlezTiO
&\\
\hline
VALD3 atomic or molecular lines file
&&
FileVald3
&\\
\hline
WebSim-COMPASS “.par” (parameters) file
&&
FilePar
&\\
\hline
\sphinxtitleref{x.py} Differential Abundances X FWHMs (Python source)
&
abxfwhm.py
&
FileAbXFwhm
&
\sphinxcode{x.py}
\\
\hline
\end{longtable}\sphinxatlongtableend\end{savenotes}

By the way, the above table was generated with the following code:

\begin{sphinxVerbatim}[commandchars=\\\{\}]
\PYG{k+kn}{import} \PYG{n+nn}{filetypes} \PYG{k+kn}{as} \PYG{n+nn}{ft}
\PYG{k}{print}\PYG{p}{(}\PYG{l+s+s2}{\PYGZdq{}}\PYG{l+s+se}{\PYGZbs{}n}\PYG{l+s+s2}{\PYGZdq{}}\PYG{o}{.}\PYG{n}{join}\PYG{p}{(}\PYG{n}{ft}\PYG{o}{.}\PYG{n}{tabulate\PYGZus{}filetypes\PYGZus{}rest}\PYG{p}{(}\PYG{l+m+mi}{55}\PYG{p}{)}\PYG{p}{)}\PYG{p}{)}
\end{sphinxVerbatim}


\section{Examples}
\label{\detokenize{filetypes:examples}}

\subsection{Convert 1D spectral file to FITS format}
\label{\detokenize{filetypes:convert-1d-spectral-file-to-fits-format}}
\begin{sphinxVerbatim}[commandchars=\\\{\}]
\PYG{c+ch}{\PYGZsh{}!/usr/bin/env python}
\PYG{l+s+sd}{\PYGZdq{}\PYGZdq{}\PYGZdq{}Converts 1D spectral file of any supported type to FITS format.}

\PYG{l+s+sd}{The new file is saved with name \PYGZdq{}\PYGZlt{}original\PYGZhy{}filename\PYGZgt{}.fits\PYGZdq{}.}

\PYG{l+s+sd}{TODO handle non\PYGZhy{}equally spaced wavelength values}
\PYG{l+s+sd}{\PYGZdq{}\PYGZdq{}\PYGZdq{}}

\PYG{k+kn}{import} \PYG{n+nn}{f311}\PYG{n+nn}{.}\PYG{n+nn}{filetypes} \PYG{k}{as} \PYG{n+nn}{ft}
\PYG{k+kn}{import} \PYG{n+nn}{sys}
\PYG{k+kn}{import} \PYG{n+nn}{logging}

\PYG{k}{if} \PYG{n+nv+vm}{\PYGZus{}\PYGZus{}name\PYGZus{}\PYGZus{}} \PYG{o}{==} \PYG{l+s+s2}{\PYGZdq{}}\PYG{l+s+s2}{\PYGZus{}\PYGZus{}main\PYGZus{}\PYGZus{}}\PYG{l+s+s2}{\PYGZdq{}}\PYG{p}{:}
    \PYG{k}{if} \PYG{n+nb}{len}\PYG{p}{(}\PYG{n}{sys}\PYG{o}{.}\PYG{n}{argv}\PYG{p}{)} \PYG{o}{\PYGZlt{}} \PYG{l+m+mi}{2} \PYG{o+ow}{or} \PYG{n+nb}{any}\PYG{p}{(}\PYG{p}{[}\PYG{n}{x}\PYG{o}{.}\PYG{n}{startswith}\PYG{p}{(}\PYG{l+s+s2}{\PYGZdq{}}\PYG{l+s+s2}{\PYGZhy{}}\PYG{l+s+s2}{\PYGZdq{}}\PYG{p}{)} \PYG{k}{for} \PYG{n}{x} \PYG{o+ow}{in} \PYG{n}{sys}\PYG{o}{.}\PYG{n}{argv}\PYG{p}{[}\PYG{l+m+mi}{1}\PYG{p}{:}\PYG{p}{]}\PYG{p}{]}\PYG{p}{)}\PYG{p}{:}
        \PYG{n+nb}{print}\PYG{p}{(}\PYG{n+nv+vm}{\PYGZus{}\PYGZus{}doc\PYGZus{}\PYGZus{}}\PYG{o}{+}\PYG{l+s+s2}{\PYGZdq{}}\PYG{l+s+se}{\PYGZbs{}n}\PYG{l+s+s2}{Usage:}\PYG{l+s+se}{\PYGZbs{}n}\PYG{l+s+se}{\PYGZbs{}n}\PYG{l+s+s2}{    convert\PYGZhy{}to\PYGZhy{}fits.py filename0 [filename1 [filename2 [...]]]}\PYG{l+s+se}{\PYGZbs{}n}\PYG{l+s+s2}{\PYGZdq{}}\PYG{p}{)}
        \PYG{n}{sys}\PYG{o}{.}\PYG{n}{exit}\PYG{p}{(}\PYG{p}{)}

    \PYG{k}{for} \PYG{n}{filename} \PYG{o+ow}{in} \PYG{n}{sys}\PYG{o}{.}\PYG{n}{argv}\PYG{p}{[}\PYG{l+m+mi}{1}\PYG{p}{:}\PYG{p}{]}\PYG{p}{:}
        \PYG{n+nb}{print}\PYG{p}{(}\PYG{l+s+s2}{\PYGZdq{}}\PYG{l+s+s2}{Converting file }\PYG{l+s+s2}{\PYGZsq{}}\PYG{l+s+si}{\PYGZob{}\PYGZcb{}}\PYG{l+s+s2}{\PYGZsq{}}\PYG{l+s+s2}{...}\PYG{l+s+s2}{\PYGZdq{}}\PYG{o}{.}\PYG{n}{format}\PYG{p}{(}\PYG{n}{filename}\PYG{p}{)}\PYG{p}{)}

        \PYG{k}{try}\PYG{p}{:}
            \PYG{n}{spectrum} \PYG{o}{=} \PYG{n}{ft}\PYG{o}{.}\PYG{n}{load\PYGZus{}spectrum}\PYG{p}{(}\PYG{n}{filename}\PYG{p}{)}

            \PYG{k}{if} \PYG{n}{spectrum} \PYG{o+ow}{is} \PYG{k+kc}{None}\PYG{p}{:}
                \PYG{n+nb}{print}\PYG{p}{(}\PYG{l+s+s2}{\PYGZdq{}}\PYG{l+s+s2}{File }\PYG{l+s+s2}{\PYGZsq{}}\PYG{l+s+si}{\PYGZob{}\PYGZcb{}}\PYG{l+s+s2}{\PYGZsq{}}\PYG{l+s+s2}{ not recognized as a 1D spectral file}\PYG{l+s+s2}{\PYGZdq{}}\PYG{o}{.}\PYG{n}{format}\PYG{p}{(}\PYG{n}{filename}\PYG{p}{)}\PYG{p}{)}
                \PYG{k}{continue}

            \PYG{n}{filename\PYGZus{}new} \PYG{o}{=} \PYG{n}{filename}\PYG{o}{+}\PYG{l+s+s2}{\PYGZdq{}}\PYG{l+s+s2}{.fits}\PYG{l+s+s2}{\PYGZdq{}}

            \PYG{n}{fnew} \PYG{o}{=} \PYG{n}{ft}\PYG{o}{.}\PYG{n}{FileSpectrumFits}\PYG{p}{(}\PYG{p}{)}
            \PYG{n}{fnew}\PYG{o}{.}\PYG{n}{spectrum} \PYG{o}{=} \PYG{n}{spectrum}
            \PYG{n}{fnew}\PYG{o}{.}\PYG{n}{save\PYGZus{}as}\PYG{p}{(}\PYG{n}{filename\PYGZus{}new}\PYG{p}{)}

            \PYG{n+nb}{print}\PYG{p}{(}\PYG{l+s+s2}{\PYGZdq{}}\PYG{l+s+s2}{Successfully saved }\PYG{l+s+s2}{\PYGZsq{}}\PYG{l+s+si}{\PYGZob{}\PYGZcb{}}\PYG{l+s+s2}{\PYGZsq{}}\PYG{l+s+s2}{\PYGZdq{}}\PYG{o}{.}\PYG{n}{format}\PYG{p}{(}\PYG{n}{filename\PYGZus{}new}\PYG{p}{)}\PYG{p}{)}
        \PYG{k}{except}\PYG{p}{:}
            \PYG{n}{logging}\PYG{o}{.}\PYG{n}{exception}\PYG{p}{(}\PYG{l+s+s2}{\PYGZdq{}}\PYG{l+s+s2}{Error converting file }\PYG{l+s+s2}{\PYGZsq{}}\PYG{l+s+si}{\PYGZob{}\PYGZcb{}}\PYG{l+s+s2}{\PYGZsq{}}\PYG{l+s+s2}{\PYGZdq{}}\PYG{o}{.}\PYG{n}{format}\PYG{p}{(}\PYG{n}{filename}\PYG{p}{)}\PYG{p}{)}
\end{sphinxVerbatim}


\subsection{Import Kurucz’ molecular linelist file}
\label{\detokenize{filetypes:import-kurucz-molecular-linelist-file}}
\begin{sphinxVerbatim}[commandchars=\\\{\}]
\PYG{l+s+sd}{\PYGZdq{}\PYGZdq{}\PYGZdq{}}
\PYG{l+s+sd}{This example loads file \PYGZdq{}c2dabrookek.asc\PYGZdq{} and prints a memory representation of its first line.}

\PYG{l+s+sd}{This file can be obtained at http://kurucz.harvard.edu/molecules/c2/. First lines of file:}

\PYG{l+s+sd}{{}`{}`{}`}
\PYG{l+s+sd}{  287.7558\PYGZhy{}14.533 23.0  2354.082 24.0 \PYGZhy{}37095.578 6063a00e1  3d10e3  12 677  34741.495}
\PYG{l+s+sd}{  287.7564\PYGZhy{}14.955 22.0  2282.704 23.0 \PYGZhy{}37024.124 6063a00f1  3d10f3  12 677  34741.419}
\PYG{l+s+sd}{  287.7582\PYGZhy{}14.490 21.0  2214.696 22.0 \PYGZhy{}36955.900 6063a00e1  3d10e3  12 677  34741.205}
\PYG{l+s+sd}{  287.7613\PYGZhy{}15.004 24.0  2428.453 25.0 \PYGZhy{}37169.280 6063a00f1  3d10f3  12 677  34740.828}
\PYG{l+s+sd}{  287.7650\PYGZhy{}14.899 20.0  2149.765 21.0 \PYGZhy{}36890.147 6063a00f1  3d10f3  12 677  34740.382}
\PYG{l+s+sd}{{}`{}`{}`}
\PYG{l+s+sd}{\PYGZdq{}\PYGZdq{}\PYGZdq{}}

\PYG{k+kn}{import} \PYG{n+nn}{f311}\PYG{n+nn}{.}\PYG{n+nn}{filetypes} \PYG{k}{as} \PYG{n+nn}{ft}

\PYG{n}{f} \PYG{o}{=} \PYG{n}{ft}\PYG{o}{.}\PYG{n}{load\PYGZus{}any\PYGZus{}file}\PYG{p}{(}\PYG{l+s+s2}{\PYGZdq{}}\PYG{l+s+s2}{c2dabrookek.asc}\PYG{l+s+s2}{\PYGZdq{}}\PYG{p}{)}

\PYG{n+nb}{print}\PYG{p}{(}\PYG{n+nb}{repr}\PYG{p}{(}\PYG{n}{f}\PYG{p}{[}\PYG{l+m+mi}{0}\PYG{p}{]}\PYG{p}{)}\PYG{o}{.}\PYG{n}{replace}\PYG{p}{(}\PYG{l+s+s2}{\PYGZdq{}}\PYG{l+s+s2}{, }\PYG{l+s+s2}{\PYGZdq{}}\PYG{p}{,} \PYG{l+s+s2}{\PYGZdq{}}\PYG{l+s+s2}{,}\PYG{l+s+se}{\PYGZbs{}n}\PYG{l+s+s2}{              }\PYG{l+s+s2}{\PYGZdq{}}\PYG{p}{)}\PYG{p}{)}

\end{sphinxVerbatim}

\begin{sphinxVerbatim}[commandchars=\\\{\}]
KuruczMolLine(lambda\PYGZus{}=2877.558,
              loggf=\PYGZhy{}14.533,
              J2l=23.0,
              E2l=2354.082,
              Jl=24.0,
              El=37095.578,
              atomn0=6,
              atomn1=6,
              state2l=\PYGZsq{}a\PYGZsq{},
              v2l=0,
              lambda\PYGZus{}doubling2l=\PYGZsq{}e\PYGZsq{},
              spin2l=1,
              statel=\PYGZsq{}d\PYGZsq{},
              vl=10,
              lambda\PYGZus{}doublingl=\PYGZsq{}e\PYGZsq{},
              spinl=3,
              iso=12)
\end{sphinxVerbatim}


\section{API reference}
\label{\detokenize{filetypes:api-reference}}
\DUrole{xref,std,std-doc}{autodoc/f311.filetypes}


\chapter{Selected topics on Physics}
\label{\detokenize{physics::doc}}\label{\detokenize{physics:selected-topics-on-physics}}

\section{Introduction}
\label{\detokenize{physics:introduction}}
Selected Physics-related resources:
\begin{itemize}
\item {} 
Photometry (AB/Vega/Standard)

\item {} 
Spectrum-to-RGB (red, green, blue) color conversion

\item {} 
Air-to-vacuum (\& vice versa) wavelenght conversion

\item {} 
Calculation of Hönl-London factors according to formulas in Kovács’ 1969 {[}1{]}

\end{itemize}


\section{Examples}
\label{\detokenize{physics:examples}}

\subsection{Air-to-vacuum (\& vice versa) wavelength conversion}
\label{\detokenize{physics:air-to-vacuum-vice-versa-wavelength-conversion}}
The following code reproduces the figure
shown in VALD3 Wiki (\sphinxurl{http://www.astro.uu.se/valdwiki/Air-to-vacuum\%20conversion})
(“comparison of the Morton and the inverse transformation by NP between 2000 Å and 100000 Å.”)

\begin{sphinxVerbatim}[commandchars=\\\{\}]
\PYG{k+kn}{import} \PYG{n+nn}{matplotlib}\PYG{n+nn}{.}\PYG{n+nn}{pyplot} \PYG{k}{as} \PYG{n+nn}{plt}
\PYG{k+kn}{import} \PYG{n+nn}{numpy} \PYG{k}{as} \PYG{n+nn}{np}
\PYG{k+kn}{import} \PYG{n+nn}{f311}\PYG{n+nn}{.}\PYG{n+nn}{physics} \PYG{k}{as} \PYG{n+nn}{ph}
\PYG{n}{\(\lambda\)vac} \PYG{o}{=} \PYG{l+m+mi}{10}\PYG{o}{*}\PYG{o}{*}\PYG{n}{np}\PYG{o}{.}\PYG{n}{linspace}\PYG{p}{(}\PYG{n}{np}\PYG{o}{.}\PYG{n}{log10}\PYG{p}{(}\PYG{l+m+mi}{2000}\PYG{p}{)}\PYG{p}{,} \PYG{n}{np}\PYG{o}{.}\PYG{n}{log10}\PYG{p}{(}\PYG{l+m+mi}{1000000}\PYG{p}{)}\PYG{p}{,} \PYG{l+m+mi}{2000}\PYG{p}{)}
\PYG{n}{y} \PYG{o}{=} \PYG{n}{ph}\PYG{o}{.}\PYG{n}{air\PYGZus{}to\PYGZus{}vacuum}\PYG{p}{(}\PYG{n}{ph}\PYG{o}{.}\PYG{n}{vacuum\PYGZus{}to\PYGZus{}air}\PYG{p}{(}\PYG{n}{\(\lambda\)vac}\PYG{p}{)}\PYG{p}{)}\PYG{o}{\PYGZhy{}}\PYG{n}{\(\lambda\)vac}
\PYG{n}{plt}\PYG{o}{.}\PYG{n}{semilogx}\PYG{p}{(}\PYG{n}{\(\lambda\)vac}\PYG{p}{,} \PYG{n}{y}\PYG{p}{)}
\PYG{n}{plt}\PYG{o}{.}\PYG{n}{xlabel}\PYG{p}{(}\PYG{l+s+s2}{\PYGZdq{}}\PYG{l+s+s2}{\PYGZdl{}}\PYG{l+s+s2}{\PYGZbs{}}\PYG{l+s+s2}{lambda\PYGZdl{} in Angstroem}\PYG{l+s+s2}{\PYGZdq{}}\PYG{p}{)}
\PYG{n}{plt}\PYG{o}{.}\PYG{n}{ylabel}\PYG{p}{(}\PYG{l+s+s2}{\PYGZdq{}}\PYG{l+s+s2}{\PYGZdl{}}\PYG{l+s+s2}{\PYGZbs{}}\PYG{l+s+s2}{Delta}\PYG{l+s+s2}{\PYGZbs{}}\PYG{l+s+s2}{lambda\PYGZdl{}}\PYG{l+s+s2}{\PYGZdq{}}\PYG{p}{)}
\PYG{n}{plt}\PYG{o}{.}\PYG{n}{xlim}\PYG{p}{(}\PYG{p}{[}\PYG{n}{\(\lambda\)vac}\PYG{p}{[}\PYG{l+m+mi}{0}\PYG{p}{]}\PYG{o}{\PYGZhy{}}\PYG{l+m+mi}{50}\PYG{p}{,} \PYG{n}{\(\lambda\)vac}\PYG{p}{[}\PYG{o}{\PYGZhy{}}\PYG{l+m+mi}{1}\PYG{p}{]}\PYG{p}{]}\PYG{p}{)}
\PYG{n}{plt}\PYG{o}{.}\PYG{n}{title}\PYG{p}{(}\PYG{l+s+s2}{\PYGZdq{}}\PYG{l+s+s2}{air\PYGZus{}to\PYGZus{}vacuum(vacuum\PYGZus{}to\PYGZus{}air(\PYGZdl{}}\PYG{l+s+s2}{\PYGZbs{}}\PYG{l+s+s2}{lambda\PYGZus{}}\PYG{l+s+si}{\PYGZob{}vac\PYGZcb{}}\PYG{l+s+s2}{\PYGZdl{}))\PYGZhy{}\PYGZdl{}}\PYG{l+s+s2}{\PYGZbs{}}\PYG{l+s+s2}{lambda\PYGZus{}}\PYG{l+s+si}{\PYGZob{}vac\PYGZcb{}}\PYG{l+s+s2}{\PYGZdl{}}\PYG{l+s+s2}{\PYGZdq{}}\PYG{p}{)}
\PYG{n}{plt}\PYG{o}{.}\PYG{n}{tight\PYGZus{}layout}\PYG{p}{(}\PYG{p}{)}
\PYG{n}{plt}\PYG{o}{.}\PYG{n}{show}\PYG{p}{(}\PYG{p}{)}
\end{sphinxVerbatim}

\sphinxincludegraphics{{air-vac}.png}


\subsection{Calculate the magnitude of a spectrum}
\label{\detokenize{physics:calculate-the-magnitude-of-a-spectrum}}
The following example compares flux-to-magnitude conversion of the Vega spectrum
for different magnitude systems.

\begin{sphinxVerbatim}[commandchars=\\\{\}]
\PYG{k+kn}{import} \PYG{n+nn}{f311}\PYG{n+nn}{.}\PYG{n+nn}{physics} \PYG{k}{as} \PYG{n+nn}{ph}
\PYG{k+kn}{import} \PYG{n+nn}{tabulate}
\PYG{n}{systems} \PYG{o}{=} \PYG{p}{[}\PYG{l+s+s2}{\PYGZdq{}}\PYG{l+s+s2}{stdflux}\PYG{l+s+s2}{\PYGZdq{}}\PYG{p}{,} \PYG{l+s+s2}{\PYGZdq{}}\PYG{l+s+s2}{ab}\PYG{l+s+s2}{\PYGZdq{}}\PYG{p}{,} \PYG{l+s+s2}{\PYGZdq{}}\PYG{l+s+s2}{vega}\PYG{l+s+s2}{\PYGZdq{}}\PYG{p}{]}
\PYG{n}{bands} \PYG{o}{=} \PYG{l+s+s2}{\PYGZdq{}}\PYG{l+s+s2}{UBVRIJHK}\PYG{l+s+s2}{\PYGZdq{}}
\PYG{n}{sp} \PYG{o}{=} \PYG{n}{ph}\PYG{o}{.}\PYG{n}{get\PYGZus{}vega\PYGZus{}spectrum}\PYG{p}{(}\PYG{p}{)}
\PYG{n}{rows} \PYG{o}{=} \PYG{p}{[}\PYG{p}{(}\PYG{p}{[}\PYG{n}{band}\PYG{p}{]}\PYG{o}{+}\PYG{p}{[}\PYG{n}{ph}\PYG{o}{.}\PYG{n}{calc\PYGZus{}mag}\PYG{p}{(}\PYG{n}{sp}\PYG{p}{,} \PYG{n}{band}\PYG{p}{,} \PYG{n}{system}\PYG{p}{)} \PYG{k}{for} \PYG{n}{system} \PYG{o+ow}{in} \PYG{n}{systems}\PYG{p}{]}\PYG{p}{)} \PYG{k}{for} \PYG{n}{band} \PYG{o+ow}{in} \PYG{n}{bands}\PYG{p}{]}
\PYG{n+nb}{print}\PYG{p}{(}\PYG{n}{tabulate}\PYG{o}{.}\PYG{n}{tabulate}\PYG{p}{(}\PYG{n}{rows}\PYG{p}{,} \PYG{p}{[}\PYG{l+s+s2}{\PYGZdq{}}\PYG{l+s+s2}{band}\PYG{l+s+s2}{\PYGZdq{}}\PYG{p}{]}\PYG{o}{+}\PYG{n}{systems}\PYG{p}{)}\PYG{p}{)}
\end{sphinxVerbatim}

This code results in the following table:

\begin{sphinxVerbatim}[commandchars=\\\{\}]
\PYG{n}{band}        \PYG{n}{stdflux}          \PYG{n}{ab}    \PYG{n}{vega}
\PYG{o}{\PYGZhy{}}\PYG{o}{\PYGZhy{}}\PYG{o}{\PYGZhy{}}\PYG{o}{\PYGZhy{}}\PYG{o}{\PYGZhy{}}\PYG{o}{\PYGZhy{}}  \PYG{o}{\PYGZhy{}}\PYG{o}{\PYGZhy{}}\PYG{o}{\PYGZhy{}}\PYG{o}{\PYGZhy{}}\PYG{o}{\PYGZhy{}}\PYG{o}{\PYGZhy{}}\PYG{o}{\PYGZhy{}}\PYG{o}{\PYGZhy{}}\PYG{o}{\PYGZhy{}}\PYG{o}{\PYGZhy{}}\PYG{o}{\PYGZhy{}}  \PYG{o}{\PYGZhy{}}\PYG{o}{\PYGZhy{}}\PYG{o}{\PYGZhy{}}\PYG{o}{\PYGZhy{}}\PYG{o}{\PYGZhy{}}\PYG{o}{\PYGZhy{}}\PYG{o}{\PYGZhy{}}\PYG{o}{\PYGZhy{}}\PYG{o}{\PYGZhy{}}\PYG{o}{\PYGZhy{}}  \PYG{o}{\PYGZhy{}}\PYG{o}{\PYGZhy{}}\PYG{o}{\PYGZhy{}}\PYG{o}{\PYGZhy{}}\PYG{o}{\PYGZhy{}}\PYG{o}{\PYGZhy{}}
\PYG{n}{U}        \PYG{l+m+mf}{0.00572505}   \PYG{l+m+mf}{0.761594}       \PYG{o}{\PYGZhy{}}\PYG{l+m+mi}{0}
\PYG{n}{B}        \PYG{l+m+mf}{0.0696287}   \PYG{o}{\PYGZhy{}}\PYG{l+m+mf}{0.10383}        \PYG{o}{\PYGZhy{}}\PYG{l+m+mi}{0}
\PYG{n}{V}        \PYG{l+m+mf}{0.0218067}    \PYG{l+m+mf}{0.0191189}      \PYG{o}{\PYGZhy{}}\PYG{l+m+mi}{0}
\PYG{n}{R}        \PYG{l+m+mf}{0.0359559}    \PYG{l+m+mf}{0.214645}       \PYG{o}{\PYGZhy{}}\PYG{l+m+mi}{0}
\PYG{n}{I}        \PYG{l+m+mf}{0.0661095}    \PYG{l+m+mf}{0.449825}       \PYG{o}{\PYGZhy{}}\PYG{l+m+mi}{0}
\PYG{n}{J}       \PYG{o}{\PYGZhy{}}\PYG{l+m+mf}{0.0150993}    \PYG{l+m+mf}{0.874666}       \PYG{o}{\PYGZhy{}}\PYG{l+m+mi}{0}
\PYG{n}{H}        \PYG{l+m+mf}{0.0315447}    \PYG{l+m+mf}{1.34805}        \PYG{o}{\PYGZhy{}}\PYG{l+m+mi}{0}
\PYG{n}{K}        \PYG{l+m+mf}{0.0246046}    \PYG{l+m+mf}{1.85948}        \PYG{o}{\PYGZhy{}}\PYG{l+m+mi}{0}
\end{sphinxVerbatim}


\subsection{Calculate Hönl-London factors for doublets}
\label{\detokenize{physics:calculate-honl-london-factors-for-doublets}}
In the following examples, a normalization factor is applied to the Hönl-London factors (HLF),
such that all HLFs for a given J must add up to 1.0:

\begin{sphinxVerbatim}[commandchars=\\\{\}]
\PYG{k+kn}{from} \PYG{n+nn}{f311} \PYG{k}{import} \PYG{n}{physics} \PYG{k}{as} \PYG{n}{ph}
\PYG{n}{S}\PYG{p}{,} \PYG{n}{DELTAK} \PYG{o}{=} \PYG{l+m+mf}{0.5}\PYG{p}{,} \PYG{l+m+mi}{0}  \PYG{c+c1}{\PYGZsh{} spin, delta Kronecker}
\PYG{n}{J} \PYG{o}{=} \PYG{l+m+mf}{1.5}
\PYG{n}{factor} \PYG{o}{=} \PYG{l+m+mi}{2}\PYG{o}{/}\PYG{p}{(}\PYG{p}{(}\PYG{l+m+mi}{2}\PYG{o}{*}\PYG{n}{J}\PYG{o}{+}\PYG{l+m+mi}{1}\PYG{p}{)}\PYG{o}{*}\PYG{p}{(}\PYG{l+m+mi}{2}\PYG{o}{*}\PYG{n}{S}\PYG{o}{+}\PYG{l+m+mi}{1}\PYG{p}{)}\PYG{o}{*}\PYG{p}{(}\PYG{l+m+mi}{2}\PYG{o}{\PYGZhy{}}\PYG{n}{DELTAK}\PYG{p}{)}\PYG{p}{)}
\PYG{n}{normalized} \PYG{o}{=} \PYG{p}{[}\PYG{n}{f}\PYG{p}{(}\PYG{n}{J}\PYG{p}{)}\PYG{o}{*}\PYG{n}{factor} \PYG{k}{for} \PYG{n}{f} \PYG{o+ow}{in} \PYG{n}{ph}\PYG{o}{.}\PYG{n}{doublet}\PYG{o}{.}\PYG{n}{get\PYGZus{}honllondon\PYGZus{}formulas}\PYG{p}{(}\PYG{l+m+mi}{0}\PYG{p}{,} \PYG{l+m+mi}{1}\PYG{p}{)}\PYG{o}{.}\PYG{n}{values}\PYG{p}{(}\PYG{p}{)}\PYG{p}{]}
\PYG{n+nb}{print}\PYG{p}{(}\PYG{n+nb}{sum}\PYG{p}{(}\PYG{n}{normalized}\PYG{p}{)}\PYG{p}{)}
\end{sphinxVerbatim}

This code should output:

\begin{sphinxVerbatim}[commandchars=\\\{\}]
\PYG{l+m+mf}{1.0}
\end{sphinxVerbatim}

The formulas for the HLFs were taken from the book \sphinxstyleemphasis{Istvan Kovacs, “Rotational Structure in the spectra of diatomic molecules. American Elsevier, 1969}


\section{API reference}
\label{\detokenize{physics:api-reference}}
\DUrole{xref,std,std-doc}{autodoc/f311.physics}


\chapter{Adaptive Optics Systems Simulation Support (\sphinxstyleliteralintitle{f311.aosss})}
\label{\detokenize{aosss:adaptive-optics-systems-simulation-support-f311-aosss}}\label{\detokenize{aosss::doc}}
The \sphinxstyleemphasis{aosss} package helps to automatize the simulation of Adaptive Optics Systems.


\section{Quick Start}
\label{\detokenize{aosss:quick-start}}

\subsection{List \sphinxstyleemphasis{aosss} applications}
\label{\detokenize{aosss:list-aosss-applications}}
\begin{sphinxVerbatim}[commandchars=\\\{\}]
\PYG{n}{programs}\PYG{o}{.}\PYG{n}{py} \PYG{o}{\PYGZhy{}}\PYG{n}{p} \PYG{n}{aosss}


\PYG{n}{Graphical} \PYG{n}{applications}
\PYG{o}{\PYGZhy{}}\PYG{o}{\PYGZhy{}}\PYG{o}{\PYGZhy{}}\PYG{o}{\PYGZhy{}}\PYG{o}{\PYGZhy{}}\PYG{o}{\PYGZhy{}}\PYG{o}{\PYGZhy{}}\PYG{o}{\PYGZhy{}}\PYG{o}{\PYGZhy{}}\PYG{o}{\PYGZhy{}}\PYG{o}{\PYGZhy{}}\PYG{o}{\PYGZhy{}}\PYG{o}{\PYGZhy{}}\PYG{o}{\PYGZhy{}}\PYG{o}{\PYGZhy{}}\PYG{o}{\PYGZhy{}}\PYG{o}{\PYGZhy{}}\PYG{o}{\PYGZhy{}}\PYG{o}{\PYGZhy{}}\PYG{o}{\PYGZhy{}}\PYG{o}{\PYGZhy{}}\PYG{o}{\PYGZhy{}}

  \PYG{n}{wavelength}\PYG{o}{\PYGZhy{}}\PYG{n}{chart}\PYG{o}{.}\PYG{n}{py} \PYG{o}{.}\PYG{o}{.}\PYG{o}{.}\PYG{o}{.}\PYG{o}{.}\PYG{o}{.}\PYG{o}{.}\PYG{o}{.}\PYG{o}{.} \PYG{n}{Draws} \PYG{n}{chart} \PYG{n}{showing} \PYG{n}{spectral} \PYG{n}{lines} \PYG{n}{of} \PYG{n}{interest}\PYG{p}{,}
                                \PYG{n}{spectrograph} \PYG{n}{wavelength} \PYG{n}{ranges}\PYG{p}{,} \PYG{n}{ESO} \PYG{n}{atmospheric}
                                \PYG{n}{model}\PYG{p}{,} \PYG{n}{etc}\PYG{o}{.}

\PYG{n}{Command}\PYG{o}{\PYGZhy{}}\PYG{n}{line} \PYG{n}{tools}
\PYG{o}{\PYGZhy{}}\PYG{o}{\PYGZhy{}}\PYG{o}{\PYGZhy{}}\PYG{o}{\PYGZhy{}}\PYG{o}{\PYGZhy{}}\PYG{o}{\PYGZhy{}}\PYG{o}{\PYGZhy{}}\PYG{o}{\PYGZhy{}}\PYG{o}{\PYGZhy{}}\PYG{o}{\PYGZhy{}}\PYG{o}{\PYGZhy{}}\PYG{o}{\PYGZhy{}}\PYG{o}{\PYGZhy{}}\PYG{o}{\PYGZhy{}}\PYG{o}{\PYGZhy{}}\PYG{o}{\PYGZhy{}}\PYG{o}{\PYGZhy{}}\PYG{o}{\PYGZhy{}}

  \PYG{n}{create}\PYG{o}{\PYGZhy{}}\PYG{n}{simulation}\PYG{o}{\PYGZhy{}}\PYG{n}{reports}\PYG{o}{.}\PYG{n}{py}  \PYG{n}{Creates} \PYG{n}{HTML} \PYG{n}{reports} \PYG{k+kn}{from} \PYG{n+nn}{WebSim}\PYG{o}{\PYGZhy{}}\PYG{n}{COMPASS} \PYG{n}{output}
                                \PYG{n}{files}
  \PYG{n}{create}\PYG{o}{\PYGZhy{}}\PYG{n}{spectrum}\PYG{o}{\PYGZhy{}}\PYG{n}{lists}\PYG{o}{.}\PYG{n}{py} \PYG{o}{.}\PYG{o}{.}\PYG{o}{.}\PYG{o}{.} \PYG{n}{Create} \PYG{n}{several} \PYG{o}{.}\PYG{n}{splist} \PYG{p}{(}\PYG{n}{spectrum} \PYG{n+nb}{list}\PYG{p}{)} \PYG{n}{files}
                                \PYG{k+kn}{from} \PYG{n+nn}{WebSim}\PYG{o}{\PYGZhy{}}\PYG{n}{COMPASS} \PYG{n}{output} \PYG{n}{files}\PYG{p}{;} \PYG{n}{groups} \PYG{n}{spectra}
                                \PYG{n}{that} \PYG{n}{share} \PYG{n}{same} \PYG{n}{wavelength} \PYG{n}{vector}
  \PYG{n}{get}\PYG{o}{\PYGZhy{}}\PYG{n}{compass}\PYG{o}{.}\PYG{n}{py} \PYG{o}{.}\PYG{o}{.}\PYG{o}{.}\PYG{o}{.}\PYG{o}{.}\PYG{o}{.}\PYG{o}{.}\PYG{o}{.}\PYG{o}{.}\PYG{o}{.}\PYG{o}{.}\PYG{o}{.}\PYG{o}{.}\PYG{o}{.} \PYG{n}{Downloads} \PYG{n}{WebSim}\PYG{o}{\PYGZhy{}}\PYG{n}{COMPASS} \PYG{n}{simulations}
  \PYG{n+nb}{list}\PYG{o}{\PYGZhy{}}\PYG{n}{mosaic}\PYG{o}{\PYGZhy{}}\PYG{n}{modes}\PYG{o}{.}\PYG{n}{py} \PYG{o}{.}\PYG{o}{.}\PYG{o}{.}\PYG{o}{.}\PYG{o}{.}\PYG{o}{.}\PYG{o}{.}\PYG{o}{.} \PYG{n}{Lists} \PYG{n}{MOSAIC} \PYG{n}{Spectrograph} \PYG{n}{modes}
  \PYG{n}{organize}\PYG{o}{\PYGZhy{}}\PYG{n}{directory}\PYG{o}{.}\PYG{n}{py} \PYG{o}{.}\PYG{o}{.}\PYG{o}{.}\PYG{o}{.}\PYG{o}{.}\PYG{o}{.}\PYG{o}{.} \PYG{n}{Organizes} \PYG{n}{simulation} \PYG{n}{directory} \PYG{p}{(}\PYG{n}{creates} \PYG{n}{folders}\PYG{p}{,}
                                \PYG{n}{moves} \PYG{n}{files}\PYG{p}{,} \PYG{n}{creates} \PYG{l+s+s1}{\PYGZsq{}}\PYG{l+s+s1}{index.html}\PYG{l+s+s1}{\PYGZsq{}}\PYG{p}{)}
\end{sphinxVerbatim}

\sphinxstylestrong{Note} All the programs above can be called with the \sphinxcode{-{-}help} or \sphinxcode{-h}
option for more information


\subsection{Find wavelength region for simulation}
\label{\detokenize{aosss:find-wavelength-region-for-simulation}}
\begin{sphinxVerbatim}[commandchars=\\\{\}]
\PYG{n}{wavelength}\PYG{o}{\PYGZhy{}}\PYG{n}{chart}\PYG{o}{.}\PYG{n}{py}
\end{sphinxVerbatim}

\sphinxincludegraphics{{chart-z-0}.png}

\sphinxstylestrong{Figure} \textendash{} Lines with zero redshift

This application creates a chart stacking the MOSAIC spectrograph wavelength coverages and
an ESO Earth atmospheric model. This may serve either as a reference to MOSAIC wavelength invervals for each
mode (on this, see also \sphinxcode{list-mosaic-modes.py}) or to verify the Earth atmospheric emission/trasmission
in a wavelength region of observational interest.

It is also possible to inform a redshift so that the chemical lines will be accordingly displaced:

\sphinxincludegraphics{{chart-z-35}.png}

\sphinxstylestrong{Figure} \textendash{} \sphinxcode{z=3.5}


\subsection{Download simulation results}
\label{\detokenize{aosss:download-simulation-results}}
The following example assumes that simulations coded from 1700 to 1721 already finished on the
WebSim-COMPASS server.

\sphinxcode{get-compass.py} is a Python script based on \sphinxcode{get-compass.sh} which can be downloaded from the
WebSim-COMPASS webpage. The former enhances the latter in which:
\begin{itemize}
\item {} 
It can download several simulations in a single command

\item {} 
It is possible to specify the “stage” of the simulation pipeline to download results from. For example,
it is possible to download only the “spintg” file, skipping the large data cubes from intermediary stages.

\end{itemize}

\begin{sphinxVerbatim}[commandchars=\\\{\}]
\PYG{n}{get}\PYG{o}{\PYGZhy{}}\PYG{n}{compass}\PYG{o}{.}\PYG{n}{py} \PYG{l+m+mi}{1700}\PYG{o}{\PYGZhy{}}\PYG{l+m+mi}{1721} \PYG{o}{\PYGZhy{}}\PYG{o}{\PYGZhy{}}\PYG{n}{stage} \PYG{n}{spintg}
\end{sphinxVerbatim}

will download results for simulations \sphinxstyleemphasis{C001700}, \sphinxstyleemphasis{C001701}, …,
\sphinxstyleemphasis{C001721} \sphinxstylestrong{into the local directory}, after which you will see files
\sphinxcode{C*.fits}, \sphinxcode{C*.par}, \sphinxcode{C*.out}


\subsection{Organize simulation results}
\label{\detokenize{aosss:organize-simulation-results}}

\subsubsection{Group resulting spectra in a single file}
\label{\detokenize{aosss:group-resulting-spectra-in-a-single-file}}
This step is required for later analysis using \sphinxcode{splisted.py}

The following command will group all files “C*\_spintg.fits” into a single “.splist” (Spectrum List) file,
which can later be opened using \sphinxcode{splisted.py}

\begin{sphinxVerbatim}[commandchars=\\\{\}]
\PYGZdl{} create\PYGZhy{}spectrum\PYGZhy{}lists.py
.
.
.
[INFO    ] Created file \PYGZsq{}./group\PYGZhy{}spintg\PYGZhy{}00\PYGZhy{}C001700\PYGZhy{}C001721.splist\PYGZsq{}
[INFO    ] Created file \PYGZsq{}./group\PYGZhy{}spintg\PYGZhy{}01\PYGZhy{}C001712\PYGZhy{}C001712.splist\PYGZsq{}
\end{sphinxVerbatim}


\subsubsection{Create reports (optional)}
\label{\detokenize{aosss:create-reports-optional}}
This step creates HTML pages (one for each simulation) that help to navigate through the simulation
results.

\begin{sphinxVerbatim}[commandchars=\\\{\}]
\PYG{n}{create}\PYG{o}{\PYGZhy{}}\PYG{n}{simulation}\PYG{o}{\PYGZhy{}}\PYG{n}{reports}\PYG{o}{.}\PYG{n}{py} \PYG{l+m+mi}{1700}\PYG{o}{\PYGZhy{}}\PYG{l+m+mi}{1721}
\end{sphinxVerbatim}


\subsubsection{Organize the directory}
\label{\detokenize{aosss:organize-the-directory}}
At this point, the current directory has a large number of files (“.fits”, “.html”, “.png”, etc.),
whereas for our analysis, only the “.splist” file is required.

\sphinxcode{organize-directory.py} will:
\begin{itemize}
\item {} 
create a directory named “raw” where it will copy “.fits”, “.par” and “.out” files

\item {} 
create a directory named “reports” where it will copy “.html” and “.png” files. In addition, it will
create a file “index.html” that will serve as an index for the “.html” files

\end{itemize}

\begin{sphinxVerbatim}[commandchars=\\\{\}]
organize\PYGZhy{}directory.py
.
.
.
[INFO    ]   \PYGZhy{} Move 108 objects
[INFO    ]   \PYGZhy{} Create \PYGZsq{}reports/index.html\PYGZsq{}
Continue (Y/n)?
\end{sphinxVerbatim}


\subsection{Browse through reports}
\label{\detokenize{aosss:browse-through-reports}}
\begin{sphinxVerbatim}[commandchars=\\\{\}]
\PYG{n}{cd} \PYG{n}{reports}
\PYG{n}{xdg}\PYG{o}{\PYGZhy{}}\PYG{n+nb}{open} \PYG{n}{index}\PYG{o}{.}\PYG{n}{html}
\end{sphinxVerbatim}

will open file “index.html” in browser

\sphinxincludegraphics{{index-html}.png}

\sphinxstylestrong{Figure} \textendash{} Reports index


\subsection{Edit Spectrum List file}
\label{\detokenize{aosss:edit-spectrum-list-file}}
If you types the commands above to visualize reports, you will need to go back one directory level:

\begin{sphinxVerbatim}[commandchars=\\\{\}]
\PYG{n}{cd} \PYG{o}{.}\PYG{o}{.}
\end{sphinxVerbatim}

Now open the Spectrum List Editor (part of the f311 package):

\begin{sphinxVerbatim}[commandchars=\\\{\}]
\PYG{n}{splisted}\PYG{o}{.}\PYG{n}{py} \PYG{n}{group}\PYG{o}{\PYGZhy{}}\PYG{n}{spintg}\PYG{o}{\PYGZhy{}}\PYG{l+m+mi}{00}\PYG{o}{\PYGZhy{}}\PYG{n}{C001700}\PYG{o}{\PYGZhy{}}\PYG{n}{C001721}\PYG{o}{.}\PYG{n}{splist}
\end{sphinxVerbatim}

In the following steps, we will:
\begin{itemize}
\item {} 
Plot the spectra

\item {} 
Calculate the Signal-to-noise ratio (SNR)

\item {} 
Plot the Detector Integration Time (DIT) \sphinxstyleemphasis{vs} the SNR

\end{itemize}
\begin{enumerate}
\item {} 
Select all the spectra: click inside the table, then press \sphinxstylestrong{Ctrl+A}

\end{enumerate}

\sphinxincludegraphics{{splisted-tut-0}.png}
\begin{enumerate}
\setcounter{enumi}{1}
\item {} 
Click on “Plot Overlapped”. A plot window opens. From this plot, we can see that the region
16508-16534 seems to be free of atmospheric contamination. You may close the plot window

\end{enumerate}

\sphinxincludegraphics{{splisted-tut-1}.png}
\begin{enumerate}
\setcounter{enumi}{2}
\item {} 
Click on “To Scalar”. Another window opens

\item {} 
Type “ToScalar\_SNR(16508, 16534)”

\item {} 
Click on “OK”

\end{enumerate}

\sphinxincludegraphics{{splisted-tut-2}.png}
\begin{enumerate}
\setcounter{enumi}{5}
\item {} 
Notice that a new column “SNR” appear in the table. Click on “X-Y Plot”

\end{enumerate}

\sphinxincludegraphics{{splisted-tut-3}.png}
\begin{enumerate}
\setcounter{enumi}{6}
\item {} 
Select “Error bars”

\item {} 
Select “OBS\_DIT”

\item {} 
Click on “Redraw”

\end{enumerate}

\sphinxincludegraphics{{splisted-tut-4}.png}


\section{API reference}
\label{\detokenize{aosss:api-reference}}
\DUrole{xref,std,std-doc}{autodoc/f311.aosss}


\chapter{Index of applications (scripts)}
\label{\detokenize{scripts::doc}}\label{\detokenize{scripts:index-of-applications-scripts}}

\section{Script \sphinxstyleliteralintitle{create-simulation-reports.py}}
\label{\detokenize{autoscripts/script-create-simulation-reports::doc}}\label{\detokenize{autoscripts/script-create-simulation-reports:script-create-simulation-reports-py}}
\begin{sphinxVerbatim}[commandchars=\\\{\}]
usage: create\PYGZhy{}simulation\PYGZhy{}reports.py [\PYGZhy{}h] [\PYGZhy{}\PYGZhy{}dir [DIR]] [\PYGZhy{}\PYGZhy{}max N] N [N ...]

Creates HTML reports from WebSim\PYGZhy{}COMPASS output files

positional arguments:
  N            List of simulation numbers (single value and ranges accepted,
               e.g. 1004, 1004\PYGZhy{}1040)

optional arguments:
  \PYGZhy{}h, \PYGZhy{}\PYGZhy{}help   show this help message and exit
  \PYGZhy{}\PYGZhy{}dir [DIR]  Input directory (default: .)
  \PYGZhy{}\PYGZhy{}max N      Maximum allowed number of reports (default: 100)
\end{sphinxVerbatim}

This script belongs to package \sphinxstyleemphasis{f311.aosss}


\section{Script \sphinxstyleliteralintitle{create-spectrum-lists.py}}
\label{\detokenize{autoscripts/script-create-spectrum-lists::doc}}\label{\detokenize{autoscripts/script-create-spectrum-lists:script-create-spectrum-lists-py}}
\begin{sphinxVerbatim}[commandchars=\\\{\}]
usage: create\PYGZhy{}spectrum\PYGZhy{}lists.py [\PYGZhy{}h] [\PYGZhy{}\PYGZhy{}stage [STAGE]]

Create several .splist (spectrum list) files from WebSim\PYGZhy{}COMPASS output files; groups spectra that share same wavelength vector

All spectra in each .splist file will have the same wavelength vector

optional arguments:
  \PYGZhy{}h, \PYGZhy{}\PYGZhy{}help       show this help message and exit
  \PYGZhy{}\PYGZhy{}stage [STAGE]  Websim\PYGZhy{}Compass pipeline stage (will collect files named,
                   e.g., C000793\PYGZus{}\PYGZlt{}stage\PYGZgt{}.fits) (default: spintg)
\end{sphinxVerbatim}

This script belongs to package \sphinxstyleemphasis{f311.aosss}


\section{Script \sphinxstyleliteralintitle{get-compass.py}}
\label{\detokenize{autoscripts/script-get-compass::doc}}\label{\detokenize{autoscripts/script-get-compass:script-get-compass-py}}
\begin{sphinxVerbatim}[commandchars=\\\{\}]
usage: get\PYGZhy{}compass.py [\PYGZhy{}h] [\PYGZhy{}\PYGZhy{}max N] [\PYGZhy{}\PYGZhy{}stage [STAGE]] N [N ...]

Downloads WebSim\PYGZhy{}COMPASS simulations

Based on shell script by Mathieu Puech

**Note** Skips simulations for existing files in local directory starting with
         that simulation ID.
         Example: if it finds file(s) \PYGZdq{}C001006*\PYGZdq{}, will skip simulation C001006

**Note** Does not create any directory (actually creates it but deletes later).
         All files stored in local directory!

**Note** Will work only on if os.name == \PYGZdq{}posix\PYGZdq{} (Linux, UNIX ...)

positional arguments:
  N                List of simulation numbers (single value and ranges
                   accepted, e.g. 1004, 1004\PYGZhy{}1040)

optional arguments:
  \PYGZhy{}h, \PYGZhy{}\PYGZhy{}help       show this help message and exit
  \PYGZhy{}\PYGZhy{}max N          Maximum number of simulations to get (default: 100)
  \PYGZhy{}\PYGZhy{}stage [STAGE]  Websim\PYGZhy{}Compass pipeline stage: if specified, will download
                   files named, e.g., C000793\PYGZus{}\PYGZlt{}stage\PYGZgt{}.fits (**note**: .par and
                   .out files are always downloaded) (default: all)
\end{sphinxVerbatim}

This script belongs to package \sphinxstyleemphasis{f311.aosss}


\section{Script \sphinxstyleliteralintitle{list-mosaic-modes.py}}
\label{\detokenize{autoscripts/script-list-mosaic-modes:script-list-mosaic-modes-py}}\label{\detokenize{autoscripts/script-list-mosaic-modes::doc}}
\begin{sphinxVerbatim}[commandchars=\\\{\}]
usage: list\PYGZhy{}mosaic\PYGZhy{}modes.py [\PYGZhy{}h] [search]

Lists MOSAIC Spectrograph modes

positional arguments:
  search      Search string (optional) (e.g., \PYGZdq{}HMM\PYGZdq{}) (default: None)

optional arguments:
  \PYGZhy{}h, \PYGZhy{}\PYGZhy{}help  show this help message and exit
\end{sphinxVerbatim}

This script belongs to package \sphinxstyleemphasis{f311.aosss}


\section{Script \sphinxstyleliteralintitle{organize-directory.py}}
\label{\detokenize{autoscripts/script-organize-directory:script-organize-directory-py}}\label{\detokenize{autoscripts/script-organize-directory::doc}}
\begin{sphinxVerbatim}[commandchars=\\\{\}]
usage: organize\PYGZhy{}directory.py [\PYGZhy{}h]

Organizes simulation directory (creates folders, moves files, creates \PYGZsq{}index.html\PYGZsq{})

  \PYGZhy{} moves \PYGZsq{}root/report\PYGZhy{}*\PYGZsq{}       to \PYGZsq{}root/reports\PYGZsq{}
  \PYGZhy{} moves \PYGZsq{}root/C*\PYGZsq{}             to \PYGZsq{}root/raw\PYGZsq{}
  \PYGZhy{} moves \PYGZsq{}root/raw/simgroup*\PYGZsq{}  to \PYGZsq{}root/\PYGZsq{}
  \PYGZhy{} moves \PYGZsq{}root/raw/report\PYGZhy{}*\PYGZsq{}   to \PYGZsq{}root/reports\PYGZsq{}
  \PYGZhy{} moves \PYGZsq{}root/raw/group*.splist\PYGZsq{}   to \PYGZsq{}root\PYGZsq{}
  \PYGZhy{} [re]creates \PYGZsq{}root/reports/index.html\PYGZsq{}

This script can be run from one of these directories:
  \PYGZhy{} \PYGZsq{}root\PYGZsq{} \PYGZhy{}\PYGZhy{} a directory containing at least one of these directories: \PYGZsq{}reports\PYGZsq{}, \PYGZsq{}raw\PYGZsq{}
  \PYGZhy{} \PYGZsq{}root/raw\PYGZsq{}
  \PYGZhy{} \PYGZsq{}root/reports\PYGZsq{}

The script will use some rules to try to figure out where it is running from

optional arguments:
  \PYGZhy{}h, \PYGZhy{}\PYGZhy{}help  show this help message and exit
\end{sphinxVerbatim}

This script belongs to package \sphinxstyleemphasis{f311.aosss}


\section{Script \sphinxstyleliteralintitle{wavelength-chart.py}}
\label{\detokenize{autoscripts/script-wavelength-chart::doc}}\label{\detokenize{autoscripts/script-wavelength-chart:script-wavelength-chart-py}}
\begin{sphinxVerbatim}[commandchars=\\\{\}]
usage: wavelength\PYGZhy{}chart.py [\PYGZhy{}h] [\PYGZhy{}\PYGZhy{}plot]

Draws chart showing spectral lines of interest, spectrograph wavelength ranges, ESO atmospheric model, etc.

Two modes are available:
  \PYGZhy{} GUI mode (default): opens a GUI allowing for setup parameters
  \PYGZhy{} Plot mode (\PYGZhy{}\PYGZhy{}plot): plots the chart directly in default way

optional arguments:
  \PYGZhy{}h, \PYGZhy{}\PYGZhy{}help  show this help message and exit
  \PYGZhy{}\PYGZhy{}plot      Plot mode (default is GUI mode) (default: False)
\end{sphinxVerbatim}

This script belongs to package \sphinxstyleemphasis{f311.aosss}


\section{Script \sphinxstyleliteralintitle{hitran-scraper.py}}
\label{\detokenize{autoscripts/script-hitran-scraper:script-hitran-scraper-py}}\label{\detokenize{autoscripts/script-hitran-scraper::doc}}
\begin{sphinxVerbatim}[commandchars=\\\{\}]
usage: hitran\PYGZhy{}scraper.py [\PYGZhy{}h] [\PYGZhy{}t T] [M] [I] [llzero] [llfin]

Retrieves molecular lines from the HITRAN database [Gordon2016]

This script uses web scraping and the HAPI to save locally molecular lines from the HITRAN database.

While the HAPI provides the downloading facility, web scraping is used to get the lists of molecules
and isotopologues from the HITRAN webpages and get the IDs required to run the HAPI query.

The script is typically invoked several times, each time with an additional argument.

References:

[Gordon2016] I.E. Gordon, L.S. Rothman, C. Hill, R.V. Kochanov, Y. Tan, P.F. Bernath, M. Birk,
    V. Boudon, A. Campargue, K.V. Chance, B.J. Drouin, J.\PYGZhy{}M. Flaud, R.R. Gamache, J.T. Hodges,
    D. Jacquemart, V.I. Perevalov, A. Perrin, K.P. Shine, M.\PYGZhy{}A.H. Smith, J. Tennyson, G.C. Toon,
    H. Tran, V.G. Tyuterev, A. Barbe, A.G. Császár, V.M. Devi, T. Furtenbacher, J.J. Harrison,
    J.\PYGZhy{}M. Hartmann, A. Jolly, T.J. Johnson, T. Karman, I. Kleiner, A.A. Kyuberis, J. Loos,
    O.M. Lyulin, S.T. Massie, S.N. Mikhailenko, N. Moazzen\PYGZhy{}Ahmadi, H.S.P. Müller, O.V. Naumenko,
    A.V. Nikitin, O.L. Polyansky, M. Rey, M. Rotger, S.W. Sharpe, K. Sung, E. Starikova,
    S.A. Tashkun, J. Vander Auwera, G. Wagner, J. Wilzewski, P. Wcisło, S. Yu, E.J. Zak,
    The HITRAN2016 Molecular Spectroscopic Database, J. Quant. Spectrosc. Radiat. Transf. (2017).
    doi:10.1016/j.jqsrt.2017.06.038.

positional arguments:
  M           HITRAN molecule number (default: (lists molecules))
  I           HITRAN isotopologue number (not unique, starts over at each
              molecule) (default: (lists isotopologues))
  llzero      Initial wavelength (Angstrom) (default: None)
  llfin       Final wavelength (Angstrom) (default: None)

optional arguments:
  \PYGZhy{}h, \PYGZhy{}\PYGZhy{}help  show this help message and exit
  \PYGZhy{}t T        Table Name (default: (molecular formula))
\end{sphinxVerbatim}

This script belongs to package \sphinxstyleemphasis{f311.convmol}


\subsection{Usage examples}
\label{\detokenize{autoscripts/script-hitran-scraper:usage-examples}}
\begin{sphinxVerbatim}[commandchars=\\\{\}]
\PYGZdl{} hitran\PYGZhy{}scraper.py

List of all HITRAN molecules
============================

  ID  Formula    Name
\PYGZhy{}\PYGZhy{}\PYGZhy{}\PYGZhy{}  \PYGZhy{}\PYGZhy{}\PYGZhy{}\PYGZhy{}\PYGZhy{}\PYGZhy{}\PYGZhy{}\PYGZhy{}\PYGZhy{}  \PYGZhy{}\PYGZhy{}\PYGZhy{}\PYGZhy{}\PYGZhy{}\PYGZhy{}\PYGZhy{}\PYGZhy{}\PYGZhy{}\PYGZhy{}\PYGZhy{}\PYGZhy{}\PYGZhy{}\PYGZhy{}\PYGZhy{}\PYGZhy{}\PYGZhy{}\PYGZhy{}\PYGZhy{}\PYGZhy{}
   1  H2O        Water
   2  CO2        Carbon Dioxide
   3  O3         Ozone
   4  N2O        Nitrous Oxide
   5  CO         Carbon Monoxide
   6  CH4        Methane
   7  O2         Molecular Oxygen
   8  NO         Nitric Oxide
   9  SO2        Sulfur Dioxide
  10  NO2        Nitrogen Dioxide
  11  NH3        Ammonia
  12  HNO3       Nitric Acid
  13  OH         Hydroxyl Radical
  14  HF         Hydrogen Fluoride
  15  HCl        Hydrogen Chloride
  16  HBr        Hydrogen Bromide
  17  HI         Hydrogen Iodide
  18  ClO        Chlorine Monoxide
  19  OCS        Carbonyl Sulfide
  20  H2CO       Formaldehyde
  21  HOCl       Hypochlorous Acid
  22  N2         Molecular Nitrogen
  23  HCN        Hydrogen Cyanide
  24  CH3Cl      Methyl Chloride
  25  H2O2       Hydrogen Peroxide
  26  C2H2       Acetylene
  27  C2H6       Ethane
  28  PH3        Phosphine
  29  COF2       Carbonyl Fluoride
  31  H2S        Hydrogen Sulfide
  32  HCOOH      Formic Acid
  33  HO2        Hydroperoxyl Radical
  34  O          Oxygen Atom
  36  NO+        Nitric Oxide Cation
  37  HOBr       Hypobromous Acid
  38  C2H4       Ethylene
  39  CH3OH      Methanol
  40  CH3Br      Methyl Bromide
  41  CH3CN      Methyl Cyanide
  43  C4H2       Diacetylene
  44  HC3N       Cyanoacetylene
  45  H2         Molecular Hydrogen
  46  CS         Carbon Monosulfide
  47  SO3        Sulfur trioxide

Now, to list isotopologues for a given molecule, please type:

    hitran\PYGZhy{}scraper.py \PYGZlt{}molecule ID\PYGZgt{}

where \PYGZlt{}molecule ID\PYGZgt{} is one of the IDs listed above.
\end{sphinxVerbatim}

Now suppose we want the molecule OH molecule:

\begin{sphinxVerbatim}[commandchars=\\\{\}]
\PYGZdl{} hitran\PYGZhy{}scraper.py 13

List of all isotopologues for molecule \PYGZsq{}OH\PYGZsq{} (Hydroxyl Radical)
==============================================================

m\PYGZus{}formula      ID    ID\PYGZus{}molecule  Formula      AFGL\PYGZus{}Code  Abundance
\PYGZhy{}\PYGZhy{}\PYGZhy{}\PYGZhy{}\PYGZhy{}\PYGZhy{}\PYGZhy{}\PYGZhy{}\PYGZhy{}\PYGZhy{}\PYGZhy{}  \PYGZhy{}\PYGZhy{}\PYGZhy{}\PYGZhy{}  \PYGZhy{}\PYGZhy{}\PYGZhy{}\PYGZhy{}\PYGZhy{}\PYGZhy{}\PYGZhy{}\PYGZhy{}\PYGZhy{}\PYGZhy{}\PYGZhy{}\PYGZhy{}\PYGZhy{}  \PYGZhy{}\PYGZhy{}\PYGZhy{}\PYGZhy{}\PYGZhy{}\PYGZhy{}\PYGZhy{}\PYGZhy{}\PYGZhy{}  \PYGZhy{}\PYGZhy{}\PYGZhy{}\PYGZhy{}\PYGZhy{}\PYGZhy{}\PYGZhy{}\PYGZhy{}\PYGZhy{}\PYGZhy{}\PYGZhy{}  \PYGZhy{}\PYGZhy{}\PYGZhy{}\PYGZhy{}\PYGZhy{}\PYGZhy{}\PYGZhy{}\PYGZhy{}\PYGZhy{}\PYGZhy{}\PYGZhy{}\PYGZhy{}\PYGZhy{}\PYGZhy{}\PYGZhy{}
OH              1             13  (16)OH              61  0.997473
OH              2             13  (18)OH              81  0.002
OH              3             13  (16)OD              62  1.553710 × 10\PYGZhy{}4


Now, to download lines, please type:

    hitran\PYGZhy{}scraper.py 13 \PYGZlt{}isotopologue ID\PYGZgt{} \PYGZlt{}llzero\PYGZgt{} \PYGZlt{}llfin\PYGZgt{}

where \PYGZlt{}isotopologue ID\PYGZgt{} is one the numbers from the \PYGZsq{}ID\PYGZsq{} column above,

and [\PYGZlt{}llzero\PYGZgt{}, \PYGZlt{}llfin\PYGZgt{}] defines the wavelength interval in Angstrom.
\end{sphinxVerbatim}

Now selecting the first isotopologue and specifying the visible wavelength range:

\begin{sphinxVerbatim}[commandchars=\\\{\}]
\PYGZdl{} hitran\PYGZhy{}scraper.py 13 1 3000 7000

Isotopologue selected:
======================

Field name    Value
\PYGZhy{}\PYGZhy{}\PYGZhy{}\PYGZhy{}\PYGZhy{}\PYGZhy{}\PYGZhy{}\PYGZhy{}\PYGZhy{}\PYGZhy{}\PYGZhy{}\PYGZhy{}  \PYGZhy{}\PYGZhy{}\PYGZhy{}\PYGZhy{}\PYGZhy{}\PYGZhy{}\PYGZhy{}\PYGZhy{}
m\PYGZus{}formula     OH
ID            1
ID\PYGZus{}molecule   13
Formula       (16)OH
AFGL\PYGZus{}Code     61
Abundance     0.997473

Wavelength interval (air): [3000.0, 7000.0] Angstrom
Wavenumber interval (vacuum): [14289.61969369552, 33342.42546386186] cm**\PYGZhy{}1
Table name: \PYGZsq{}(16)OH\PYGZsq{}

Fetching data...
===
=== BEGIN messages from HITRAN API ===
===
BEGIN DOWNLOAD: (16)OH
  65536 bytes written to ./(16)OH.data
  65536 bytes written to ./(16)OH.data
  65536 bytes written to ./(16)OH.data
  65536 bytes written to ./(16)OH.data
  65536 bytes written to ./(16)OH.data
  65536 bytes written to ./(16)OH.data
  65536 bytes written to ./(16)OH.data
  65536 bytes written to ./(16)OH.data
  65536 bytes written to ./(16)OH.data
  65536 bytes written to ./(16)OH.data
Header written to ./(16)OH.header
END DOWNLOAD
                     Lines parsed: 3855
PROCESSED
===
=== END messages from HITRAN API ===
===
...done
Please check files \PYGZsq{}(16)OH.header\PYGZsq{}, \PYGZsq{}(16)OH.data\PYGZsq{}
\end{sphinxVerbatim}


\subsection{Quick note on the HITRAN API}
\label{\detokenize{autoscripts/script-hitran-scraper:quick-note-on-the-hitran-api}}
The files created (‘(16)OH.header’, ‘(16)OH.data’) can be opened using the \sphinxhref{http://hitran.org/hapi}{HAPI}.
They are also accessed by the application \sphinxcode{convmol.py}.

The HAPI can be downloaded, but one version is also included with the f311 package. The following
is an example of how the HITRAN data could be accessed from the Python console:

\begin{sphinxVerbatim}[commandchars=\\\{\}]
\PYG{g+gp}{\PYGZgt{}\PYGZgt{}\PYGZgt{} }\PYG{k+kn}{from} \PYG{n+nn}{f311} \PYG{k+kn}{import} \PYG{n}{hapi}
\PYG{g+gp}{\PYGZgt{}\PYGZgt{}\PYGZgt{} }\PYG{n}{hapi}\PYG{o}{.}\PYG{n}{loadCache}\PYG{p}{(}\PYG{p}{)}
\PYG{g+go}{Using .}
\PYG{g+go}{(16)OH}
\PYG{g+go}{                     Lines parsed: 3855}
\PYG{g+gp}{\PYGZgt{}\PYGZgt{}\PYGZgt{} }\PYG{n}{oh\PYGZus{}data} \PYG{o}{=} \PYG{n}{hapi}\PYG{o}{.}\PYG{n}{LOCAL\PYGZus{}TABLE\PYGZus{}CACHE}\PYG{p}{[}\PYG{l+s+s2}{\PYGZdq{}}\PYG{l+s+s2}{(16)OH}\PYG{l+s+s2}{\PYGZdq{}}\PYG{p}{]}
\PYG{g+gp}{\PYGZgt{}\PYGZgt{}\PYGZgt{} }\PYG{n}{oh\PYGZus{}data}\PYG{o}{.}\PYG{n}{keys}\PYG{p}{(}\PYG{p}{)}
\PYG{g+go}{dict\PYGZus{}keys([\PYGZsq{}data\PYGZsq{}, \PYGZsq{}header\PYGZsq{}])}
\PYG{g+gp}{\PYGZgt{}\PYGZgt{}\PYGZgt{} }\PYG{n}{oh\PYGZus{}data}\PYG{p}{[}\PYG{l+s+s2}{\PYGZdq{}}\PYG{l+s+s2}{data}\PYG{l+s+s2}{\PYGZdq{}}\PYG{p}{]}\PYG{o}{.}\PYG{n}{keys}\PYG{p}{(}\PYG{p}{)}
\PYG{g+go}{dict\PYGZus{}keys([\PYGZsq{}ierr\PYGZsq{}, \PYGZsq{}gpp\PYGZsq{}, \PYGZsq{}molec\PYGZus{}id\PYGZsq{}, \PYGZsq{}global\PYGZus{}lower\PYGZus{}quanta\PYGZsq{}, \PYGZsq{}sw\PYGZsq{}, \PYGZsq{}gamma\PYGZus{}self\PYGZsq{}, \PYGZsq{}n\PYGZus{}air\PYGZsq{}, \PYGZsq{}elower\PYGZsq{}, \PYGZsq{}line\PYGZus{}mixing\PYGZus{}flag\PYGZsq{}, \PYGZsq{}local\PYGZus{}lower\PYGZus{}quanta\PYGZsq{}, \PYGZsq{}gp\PYGZsq{}, \PYGZsq{}global\PYGZus{}upper\PYGZus{}quanta\PYGZsq{}, \PYGZsq{}gamma\PYGZus{}air\PYGZsq{}, \PYGZsq{}local\PYGZus{}upper\PYGZus{}quanta\PYGZsq{}, \PYGZsq{}iref\PYGZsq{}, \PYGZsq{}local\PYGZus{}iso\PYGZus{}id\PYGZsq{}, \PYGZsq{}delta\PYGZus{}air\PYGZsq{}, \PYGZsq{}nu\PYGZsq{}, \PYGZsq{}a\PYGZsq{}])}
\end{sphinxVerbatim}

To work properly with these data in your code, you may have a look at the HAPI source code and manual, as this library is
superbly documented.

Within f311, the code in \sphinxcode{f311.convmol.conv\_hitran.hitran\_to\_sols()} contains a usage example of HITRAN data.


\section{Script \sphinxstyleliteralintitle{nist-scraper.py}}
\label{\detokenize{autoscripts/script-nist-scraper:script-nist-scraper-py}}\label{\detokenize{autoscripts/script-nist-scraper::doc}}
\begin{sphinxVerbatim}[commandchars=\\\{\}]
usage: nist\PYGZhy{}scraper.py [\PYGZhy{}h] formula

Retrieves and prints a table of molecular constants from the NIST Chemistry Web Book [NISTRef]

To do so, it uses web scraping to navigate through several pages and parse the desired information
from the book web pages.

It does not provide a way to list the molecules yet, but will give an error if the molecule is not
found in the NIST web book.

Example:

    print\PYGZhy{}nist.py OH

**Note** This script was designed to work with **diatomic molecules** and may not work with other
         molecules.

**Warning** The source material online was known to contain mistakes (such as an underscore instead
            of a minus signal to indicate a negative number). We have identified a few of these,
            and build some workarounds. However, we recommend a close look at the information parsed
            before use.

**Disclaimer** This script may stop working if the NIST people update the Chemistry Web Book.

References:

[NISTRef] http://webbook.nist.gov/chemistry/

positional arguments:
  formula     NIST formula

optional arguments:
  \PYGZhy{}h, \PYGZhy{}\PYGZhy{}help  show this help message and exit
\end{sphinxVerbatim}

This script belongs to package \sphinxstyleemphasis{f311.convmol}


\subsection{Usage examples}
\label{\detokenize{autoscripts/script-nist-scraper:usage-examples}}
Usage examples:

\begin{sphinxVerbatim}[commandchars=\\\{\}]
\PYG{n}{nist}\PYG{o}{\PYGZhy{}}\PYG{n}{scraper}\PYG{o}{.}\PYG{n}{py} \PYG{n}{TiO}
\end{sphinxVerbatim}

will print

\begin{sphinxVerbatim}[commandchars=\\\{\}]
*** titanium oxide ***

State       T\PYGZus{}e            omega\PYGZus{}e    omega\PYGZus{}ex\PYGZus{}e    omega\PYGZus{}ey\PYGZus{}e      B\PYGZus{}e    alpha\PYGZus{}e    gamma\PYGZus{}e       D\PYGZus{}e    beta\PYGZus{}e      r\PYGZus{}e  Trans.       nu\PYGZus{}00  A
\PYGZhy{}\PYGZhy{}\PYGZhy{}\PYGZhy{}\PYGZhy{}\PYGZhy{}\PYGZhy{}\PYGZhy{}\PYGZhy{}\PYGZhy{}  \PYGZhy{}\PYGZhy{}\PYGZhy{}\PYGZhy{}\PYGZhy{}\PYGZhy{}\PYGZhy{}\PYGZhy{}\PYGZhy{}\PYGZhy{}\PYGZhy{}  \PYGZhy{}\PYGZhy{}\PYGZhy{}\PYGZhy{}\PYGZhy{}\PYGZhy{}\PYGZhy{}\PYGZhy{}\PYGZhy{}  \PYGZhy{}\PYGZhy{}\PYGZhy{}\PYGZhy{}\PYGZhy{}\PYGZhy{}\PYGZhy{}\PYGZhy{}\PYGZhy{}\PYGZhy{}\PYGZhy{}\PYGZhy{}  \PYGZhy{}\PYGZhy{}\PYGZhy{}\PYGZhy{}\PYGZhy{}\PYGZhy{}\PYGZhy{}\PYGZhy{}\PYGZhy{}\PYGZhy{}\PYGZhy{}\PYGZhy{}  \PYGZhy{}\PYGZhy{}\PYGZhy{}\PYGZhy{}\PYGZhy{}\PYGZhy{}\PYGZhy{}  \PYGZhy{}\PYGZhy{}\PYGZhy{}\PYGZhy{}\PYGZhy{}\PYGZhy{}\PYGZhy{}\PYGZhy{}\PYGZhy{}  \PYGZhy{}\PYGZhy{}\PYGZhy{}\PYGZhy{}\PYGZhy{}\PYGZhy{}\PYGZhy{}\PYGZhy{}\PYGZhy{}  \PYGZhy{}\PYGZhy{}\PYGZhy{}\PYGZhy{}\PYGZhy{}\PYGZhy{}\PYGZhy{}\PYGZhy{}  \PYGZhy{}\PYGZhy{}\PYGZhy{}\PYGZhy{}\PYGZhy{}\PYGZhy{}\PYGZhy{}\PYGZhy{}  \PYGZhy{}\PYGZhy{}\PYGZhy{}\PYGZhy{}\PYGZhy{}\PYGZhy{}\PYGZhy{}  \PYGZhy{}\PYGZhy{}\PYGZhy{}\PYGZhy{}\PYGZhy{}\PYGZhy{}\PYGZhy{}\PYGZhy{}  \PYGZhy{}\PYGZhy{}\PYGZhy{}\PYGZhy{}\PYGZhy{}\PYGZhy{}\PYGZhy{}\PYGZhy{}  \PYGZhy{}\PYGZhy{}\PYGZhy{}
D           31920.0        1040                                                                                             D ↔ X     31940
e ⁱSigma⁺   a + 26598.1     845.2          4.2                  0.4892     0.0023              4.7e\PYGZhy{}07             1.695    e ↔ d R   24297.5
f ⁱDelta    (a + 19132)     890                                 0.50221                        6.4e\PYGZhy{}07             1.67292  f ↔ a R   19068.9
c ⁱPhi      a + 17890.2     909.6          4.19                 0.523      0.00313             3.9e\PYGZhy{}07             1.6393   c ↔ a R   17840.6
C \(\sp{\text{3}}\)Delta\PYGZus{}r  19617.0         838.26         4.76         0.047   0.48989    0.00306   \PYGZhy{}3e\PYGZhy{}05    6.7e\PYGZhy{}07             1.69383  C ↔ X R   19334
B \(\sp{\text{3}}\)Pi\PYGZus{}r     16331.3         875            5                    0.50617                        6.86e\PYGZhy{}07            1.66636  B ↔ X R   16066.7
b ⁱPi       a+11322.0\PYGZus{}3     911.2          3.72                 0.51337    0.00291             6.1e\PYGZhy{}07             1.65464  b ↔ d R    9054.02
A \(\sp{\text{3}}\)Phi\PYGZus{}r    14431.0         867.78         3.942                0.50739    0.00315   \PYGZhy{}1e\PYGZhy{}05    6.92e\PYGZhy{}07     2e\PYGZhy{}09  1.66436  A ↔ X R   14163
E \(\sp{\text{3}}\)Pi       12025.0         924.2          5.1                                                                              E ↔ X     11871
d ⁱSigma⁺   a + 2215.6     1014.6          4.64                 0.54922    0.00337             6e\PYGZhy{}07               1.59972
a ⁱDelta    a              1009.3          3.93                 0.5376     0.00298             5.9e\PYGZhy{}07             1.61692
X \(\sp{\text{3}}\)Delta\PYGZus{}r  197.5          1009.02         4.498       \PYGZhy{}0.0107  0.53541    0.00301   \PYGZhy{}1.1e\PYGZhy{}05  6.03e\PYGZhy{}07     3e\PYGZhy{}09  1.62022
\end{sphinxVerbatim}

\begin{sphinxVerbatim}[commandchars=\\\{\}]
\PYG{n}{nist}\PYG{o}{\PYGZhy{}}\PYG{n}{scraper}\PYG{o}{.}\PYG{n}{py} \PYG{n}{OH}
\end{sphinxVerbatim}

will print

\begin{sphinxVerbatim}[commandchars=\\\{\}]
*** Hydroxyl radical ***

State          T\PYGZus{}e    omega\PYGZus{}e    omega\PYGZus{}ex\PYGZus{}e  omega\PYGZus{}ey\PYGZus{}e        B\PYGZus{}e    alpha\PYGZus{}e    gamma\PYGZus{}e       D\PYGZus{}e  beta\PYGZus{}e        r\PYGZus{}e  Trans.        nu\PYGZus{}00        A
\PYGZhy{}\PYGZhy{}\PYGZhy{}\PYGZhy{}\PYGZhy{}\PYGZhy{}\PYGZhy{}\PYGZhy{}\PYGZhy{}  \PYGZhy{}\PYGZhy{}\PYGZhy{}\PYGZhy{}\PYGZhy{}\PYGZhy{}\PYGZhy{}  \PYGZhy{}\PYGZhy{}\PYGZhy{}\PYGZhy{}\PYGZhy{}\PYGZhy{}\PYGZhy{}\PYGZhy{}\PYGZhy{}  \PYGZhy{}\PYGZhy{}\PYGZhy{}\PYGZhy{}\PYGZhy{}\PYGZhy{}\PYGZhy{}\PYGZhy{}\PYGZhy{}\PYGZhy{}\PYGZhy{}\PYGZhy{}  \PYGZhy{}\PYGZhy{}\PYGZhy{}\PYGZhy{}\PYGZhy{}\PYGZhy{}\PYGZhy{}\PYGZhy{}\PYGZhy{}\PYGZhy{}\PYGZhy{}\PYGZhy{}  \PYGZhy{}\PYGZhy{}\PYGZhy{}\PYGZhy{}\PYGZhy{}\PYGZhy{}\PYGZhy{}  \PYGZhy{}\PYGZhy{}\PYGZhy{}\PYGZhy{}\PYGZhy{}\PYGZhy{}\PYGZhy{}\PYGZhy{}\PYGZhy{}  \PYGZhy{}\PYGZhy{}\PYGZhy{}\PYGZhy{}\PYGZhy{}\PYGZhy{}\PYGZhy{}\PYGZhy{}\PYGZhy{}  \PYGZhy{}\PYGZhy{}\PYGZhy{}\PYGZhy{}\PYGZhy{}\PYGZhy{}\PYGZhy{}\PYGZhy{}  \PYGZhy{}\PYGZhy{}\PYGZhy{}\PYGZhy{}\PYGZhy{}\PYGZhy{}\PYGZhy{}\PYGZhy{}  \PYGZhy{}\PYGZhy{}\PYGZhy{}\PYGZhy{}\PYGZhy{}\PYGZhy{}\PYGZhy{}  \PYGZhy{}\PYGZhy{}\PYGZhy{}\PYGZhy{}\PYGZhy{}\PYGZhy{}\PYGZhy{}\PYGZhy{}\PYGZhy{}  \PYGZhy{}\PYGZhy{}\PYGZhy{}\PYGZhy{}\PYGZhy{}\PYGZhy{}\PYGZhy{}\PYGZhy{}  \PYGZhy{}\PYGZhy{}\PYGZhy{}\PYGZhy{}\PYGZhy{}\PYGZhy{}\PYGZhy{}
C \(\sp{\text{2}}\)Sigma⁺  89459.1    1232.9        19.1                    4.247      0.078              0.0002              2.0461   C \(\rightarrow\) A R    55820.7
C \(\sp{\text{2}}\)Sigma⁺                                                                                                              (C \(\rightarrow\) X)    88223
D \(\sp{\text{2}}\)Sigma⁻  82130      2954                                 15.2179                        0.001616            1.08093  D ← X R    81759.8
B \(\sp{\text{2}}\)Sigma⁺  69774       660                                  5.086                         0.000929            1.8698   B \(\rightarrow\) A R    35965.5
A \(\sp{\text{2}}\)Sigma⁺  32684.1    3178.86       92.917                 17.358      0.7868     \PYGZhy{}0.016  0.002039            1.0121   A ↔ X R    32402.4
X \(\sp{\text{2}}\)Pi\PYGZus{}i        0      3737.76       84.8813                18.9108     0.7242             0.001938            0.96966  1/2 ← 3/2    126.23  \PYGZhy{}139.21
\end{sphinxVerbatim}


\section{Script \sphinxstyleliteralintitle{convmol.py}}
\label{\detokenize{autoscripts/script-convmol::doc}}\label{\detokenize{autoscripts/script-convmol:script-convmol-py}}
\begin{sphinxVerbatim}[commandchars=\\\{\}]
usage: convmol.py [\PYGZhy{}h] [\PYGZhy{}\PYGZhy{}fn\PYGZus{}moldb [FN\PYGZus{}MOLDB]] [\PYGZhy{}\PYGZhy{}fn\PYGZus{}molconsts [FN\PYGZus{}MOLCONSTS]]
                  [\PYGZhy{}\PYGZhy{}fn\PYGZus{}config [FN\PYGZus{}CONFIG]]

Conversion of molecular lines data to PFANT format

optional arguments:
  \PYGZhy{}h, \PYGZhy{}\PYGZhy{}help            show this help message and exit
  \PYGZhy{}\PYGZhy{}fn\PYGZus{}moldb [FN\PYGZus{}MOLDB]
                        File name for Database of Molecular Constants
                        (default: moldb.sqlite)
  \PYGZhy{}\PYGZhy{}fn\PYGZus{}molconsts [FN\PYGZus{}MOLCONSTS]
                        File name for Molecular constants config file (Python
                        code) (default: configmolconsts.py)
  \PYGZhy{}\PYGZhy{}fn\PYGZus{}config [FN\PYGZus{}CONFIG]
                        File name for Configuration file for molecular lines
                        conversion GUI (Python code) (default:
                        configconvmol.py)
\end{sphinxVerbatim}

This script belongs to package \sphinxstyleemphasis{f311.convmol}


\section{Script \sphinxstyleliteralintitle{mced.py}}
\label{\detokenize{autoscripts/script-mced::doc}}\label{\detokenize{autoscripts/script-mced:script-mced-py}}
\begin{sphinxVerbatim}[commandchars=\\\{\}]
usage: mced.py [\PYGZhy{}h] [fn]

Editor for molecular constants file

This application can edit files of class FileMolConsts.

positional arguments:
  fn          Molecular constants file name (default: configmolconsts.py)

optional arguments:
  \PYGZhy{}h, \PYGZhy{}\PYGZhy{}help  show this help message and exit
\end{sphinxVerbatim}

This script belongs to package \sphinxstyleemphasis{f311.convmol}


\section{Script \sphinxstyleliteralintitle{moldbed.py}}
\label{\detokenize{autoscripts/script-moldbed::doc}}\label{\detokenize{autoscripts/script-moldbed:script-moldbed-py}}
\begin{sphinxVerbatim}[commandchars=\\\{\}]
usage: moldbed.py [\PYGZhy{}h] [fn]

Editor for molecules SQLite database

This application can edit files of class FileMolDB.

positional arguments:
  fn          Molecules database file name (default: moldb.sqlite)

optional arguments:
  \PYGZhy{}h, \PYGZhy{}\PYGZhy{}help  show this help message and exit
\end{sphinxVerbatim}

This script belongs to package \sphinxstyleemphasis{f311.convmol}


\section{Script \sphinxstyleliteralintitle{create-grid.py}}
\label{\detokenize{autoscripts/script-create-grid:script-create-grid-py}}\label{\detokenize{autoscripts/script-create-grid::doc}}
\begin{sphinxVerbatim}[commandchars=\\\{\}]
usage: create\PYGZhy{}grid.py [\PYGZhy{}h] [\PYGZhy{}\PYGZhy{}pattern [PATTERN]]
                      [\PYGZhy{}\PYGZhy{}mode [\PYGZob{}opa,modtxt,modbin\PYGZcb{}]]
                      [fn\PYGZus{}output]

Merges several atmospheric models into a single file (\PYGZus{}i.e.\PYGZus{}, the \PYGZdq{}grid\PYGZdq{})

\PYGZdq{}Collects\PYGZdq{} several files in current directory and creates a single file
containing atmospheric model grid.

Working modes (option \PYGZdq{}\PYGZhy{}m\PYGZdq{}):
 \PYGZdq{}opa\PYGZdq{} (default mode): looks for MARCS[1] \PYGZdq{}.mod\PYGZdq{} and \PYGZdq{}.opa\PYGZdq{} text file pairs and
                       creates a *big* binary file containing *all* model
                       information including opacities.
                       Output will be in \PYGZdq{}.moo\PYGZdq{} format.

 \PYGZdq{}modtxt\PYGZdq{}: looks for MARCS \PYGZdq{}.mod\PYGZdq{} text files only. Resulting grid will not contain
           opacity information.
           Output will be in binary \PYGZdq{}.mod\PYGZdq{} format.

 \PYGZdq{}modbin\PYGZdq{}: looks for binary\PYGZhy{}format \PYGZdq{}.mod\PYGZdq{} files. Resulting grid will not contain
           opacity information.
           Output will be in binary \PYGZdq{}.mod\PYGZdq{} format.

References:
  [1] http://marcs.astro.uu.se/

.
.
.

positional arguments:
  fn\PYGZus{}output             output file name (default: \PYGZdq{}grid.moo\PYGZdq{} or \PYGZdq{}grid.mod\PYGZdq{},
                        depending on mode)

optional arguments:
  \PYGZhy{}h, \PYGZhy{}\PYGZhy{}help            show this help message and exit
  \PYGZhy{}\PYGZhy{}pattern [PATTERN]   file name pattern (with wildcards) (default: *.mod)
  \PYGZhy{}\PYGZhy{}mode [\PYGZob{}opa,modtxt,modbin\PYGZcb{}]
                        working mode (see description above) (default: opa)
\end{sphinxVerbatim}

This script belongs to package \sphinxstyleemphasis{f311.explorer}


\section{Script \sphinxstyleliteralintitle{cut-atoms.py}}
\label{\detokenize{autoscripts/script-cut-atoms::doc}}\label{\detokenize{autoscripts/script-cut-atoms:script-cut-atoms-py}}
\begin{sphinxVerbatim}[commandchars=\\\{\}]
usage: cut\PYGZhy{}atoms.py [\PYGZhy{}h] llzero llfin fn\PYGZus{}input fn\PYGZus{}output

Cuts atomic lines file to wavelength interval specified

The interval is [llzero, llfin]

positional arguments:
  llzero      lower wavelength boundary (angstrom)
  llfin       upper wavelength boundary (angstrom)
  fn\PYGZus{}input    input file name
  fn\PYGZus{}output   output file name

optional arguments:
  \PYGZhy{}h, \PYGZhy{}\PYGZhy{}help  show this help message and exit
\end{sphinxVerbatim}

This script belongs to package \sphinxstyleemphasis{f311.explorer}


\section{Script \sphinxstyleliteralintitle{cut-molecules.py}}
\label{\detokenize{autoscripts/script-cut-molecules::doc}}\label{\detokenize{autoscripts/script-cut-molecules:script-cut-molecules-py}}
\begin{sphinxVerbatim}[commandchars=\\\{\}]
usage: cut\PYGZhy{}molecules.py [\PYGZhy{}h] llzero llfin fn\PYGZus{}input fn\PYGZus{}output

Cuts molecular lines file to wavelength interval specified

The interval is [llzero, llfin]

positional arguments:
  llzero      lower wavelength boundary (angstrom)
  llfin       upper wavelength boundary (angstrom)
  fn\PYGZus{}input    input file name
  fn\PYGZus{}output   output file name

optional arguments:
  \PYGZhy{}h, \PYGZhy{}\PYGZhy{}help  show this help message and exit
\end{sphinxVerbatim}

This script belongs to package \sphinxstyleemphasis{f311.explorer}


\section{Script \sphinxstyleliteralintitle{cut-spectrum.py}}
\label{\detokenize{autoscripts/script-cut-spectrum::doc}}\label{\detokenize{autoscripts/script-cut-spectrum:script-cut-spectrum-py}}
\begin{sphinxVerbatim}[commandchars=\\\{\}]
usage: cut\PYGZhy{}spectrum.py [\PYGZhy{}h] llzero llfin fn\PYGZus{}input fn\PYGZus{}output

Cuts spectrum file to wavelength interval specified

Resulting spectrum Saved in 2\PYGZhy{}column ASCII format

The interval is [llzero, llfin]

positional arguments:
  llzero      lower wavelength boundary (angstrom)
  llfin       upper wavelength boundary (angstrom)
  fn\PYGZus{}input    input file name
  fn\PYGZus{}output   output file name

optional arguments:
  \PYGZhy{}h, \PYGZhy{}\PYGZhy{}help  show this help message and exit
\end{sphinxVerbatim}

This script belongs to package \sphinxstyleemphasis{f311.explorer}


\section{Script \sphinxstyleliteralintitle{plot-spectra.py}}
\label{\detokenize{autoscripts/script-plot-spectra:script-plot-spectra-py}}\label{\detokenize{autoscripts/script-plot-spectra::doc}}
\begin{sphinxVerbatim}[commandchars=\\\{\}]
usage: plot\PYGZhy{}spectra.py [\PYGZhy{}h] [\PYGZhy{}\PYGZhy{}ovl \textbar{} \PYGZhy{}\PYGZhy{}pieces \textbar{} \PYGZhy{}\PYGZhy{}pages] [\PYGZhy{}\PYGZhy{}aint [AINT]]
                       [\PYGZhy{}\PYGZhy{}fn\PYGZus{}output [FN\PYGZus{}OUTPUT]] [\PYGZhy{}\PYGZhy{}ymin [YMIN]]
                       [\PYGZhy{}r [NUM\PYGZus{}ROWS]]
                       fn [fn ...]

Plots spectra on screen or creates PDF file

It can work in four different modes:

a) grid of sub\PYGZhy{}plots, one for each spectrum (default mode)
   Example:
   plot\PYGZhy{}spectra.py flux.norm.nulbad measured.fits

b) single plot with all spectra overlapped (\PYGZdq{}\PYGZhy{}\PYGZhy{}ovl\PYGZdq{} option)
   Example:
   \PYGZgt{} plot\PYGZhy{}spectra.py \PYGZhy{}\PYGZhy{}ovl flux.norm.nulbad measured.fits

c) PDF file with a small wavelength interval per page (\PYGZdq{}\PYGZhy{}\PYGZhy{}pieces\PYGZdq{} option).
   This is useful to flick through a large wavelength range.
   Example:
   \PYGZgt{} plot\PYGZhy{}spectra.py \PYGZhy{}\PYGZhy{}pieces \PYGZhy{}\PYGZhy{}aint 7 flux.norm.nulbad measured.fits

d) PDF file with one spectrum per page (\PYGZdq{}\PYGZhy{}\PYGZhy{}pages\PYGZdq{} option).
   Example:
   \PYGZgt{} plot\PYGZhy{}spectra.py \PYGZhy{}\PYGZhy{}pages flux.*

Types of files supported:

  \PYGZhy{} pfant output, e.g., flux.norm;
  \PYGZhy{} nulbad output, e.g., flux.norm.nulbad;
  \PYGZhy{} 2\PYGZhy{}column \PYGZdq{}lambda\PYGZhy{}flux\PYGZdq{} generic text files;
  \PYGZhy{} FITS files.

positional arguments:
  fn                    name of spectrum file(s) (many types supported)
                        (wildcards allowed, e.g., \PYGZdq{}flux.*\PYGZdq{})

optional arguments:
  \PYGZhy{}h, \PYGZhy{}\PYGZhy{}help            show this help message and exit
  \PYGZhy{}\PYGZhy{}ovl                 Overlapped graphics (default: False)
  \PYGZhy{}\PYGZhy{}pieces              If set, will generate a PDF file with each page
                        containing one \PYGZdq{}piece\PYGZdq{} of the spectra of lengthgiven
                        by the \PYGZhy{}\PYGZhy{}aint option. (default: False)
  \PYGZhy{}\PYGZhy{}pages               If set, will generate a PDF file with one spectrum per
                        page (default: False)
  \PYGZhy{}\PYGZhy{}aint [AINT]         length of each piece\PYGZhy{}plot in wavelength units (used
                        only if \PYGZhy{}\PYGZhy{}pieces) (default: 10)
  \PYGZhy{}\PYGZhy{}fn\PYGZus{}output [FN\PYGZus{}OUTPUT]
                        PDF output file name (used only if \PYGZhy{}\PYGZhy{}pieces) (default:
                        (plot\PYGZhy{}spectra\PYGZhy{}\PYGZlt{}xxxx\PYGZgt{}.pdf))
  \PYGZhy{}\PYGZhy{}ymin [YMIN]         Minimum value for y\PYGZhy{}axis (default: (automatic))
  \PYGZhy{}r [NUM\PYGZus{}ROWS], \PYGZhy{}\PYGZhy{}num\PYGZus{}rows [NUM\PYGZus{}ROWS]
                        Number of rows in subplot grid (default: (automatic))
\end{sphinxVerbatim}

This script belongs to package \sphinxstyleemphasis{f311.explorer}


\section{Script \sphinxstyleliteralintitle{vald3-to-atoms.py}}
\label{\detokenize{autoscripts/script-vald3-to-atoms::doc}}\label{\detokenize{autoscripts/script-vald3-to-atoms:script-vald3-to-atoms-py}}
\begin{sphinxVerbatim}[commandchars=\\\{\}]
usage: vald3\PYGZhy{}to\PYGZhy{}atoms.py [\PYGZhy{}h] [\PYGZhy{}\PYGZhy{}min\PYGZus{}algf [MIN\PYGZus{}ALGF]] [\PYGZhy{}\PYGZhy{}max\PYGZus{}kiex [MAX\PYGZus{}KIEX]]
                         fn\PYGZus{}input [fn\PYGZus{}output]

Converts VALD3 atomic/molecular lines file to PFANT atomic lines file.

Molecular lines are skipped.

positional arguments:
  fn\PYGZus{}input              input file name
  fn\PYGZus{}output             output file name (default: atoms\PYGZhy{}untuned\PYGZhy{}\PYGZlt{}fn\PYGZus{}input\PYGZgt{})

optional arguments:
  \PYGZhy{}h, \PYGZhy{}\PYGZhy{}help            show this help message and exit
  \PYGZhy{}\PYGZhy{}min\PYGZus{}algf [MIN\PYGZus{}ALGF]
                        minimum algf (log gf) (default: \PYGZhy{}7)
  \PYGZhy{}\PYGZhy{}max\PYGZus{}kiex [MAX\PYGZus{}KIEX]
                        maximum kiex (default: 15)
\end{sphinxVerbatim}

This script belongs to package \sphinxstyleemphasis{f311.explorer}


\section{Script \sphinxstyleliteralintitle{abed.py}}
\label{\detokenize{autoscripts/script-abed:script-abed-py}}\label{\detokenize{autoscripts/script-abed::doc}}
\begin{sphinxVerbatim}[commandchars=\\\{\}]
usage: abed.py [\PYGZhy{}h] [fn]

Abundances file editor

positional arguments:
  fn          abundances file name (default: abonds.dat)

optional arguments:
  \PYGZhy{}h, \PYGZhy{}\PYGZhy{}help  show this help message and exit
\end{sphinxVerbatim}

This script belongs to package \sphinxstyleemphasis{f311.explorer}


\section{Script \sphinxstyleliteralintitle{ated.py}}
\label{\detokenize{autoscripts/script-ated::doc}}\label{\detokenize{autoscripts/script-ated:script-ated-py}}
\begin{sphinxVerbatim}[commandchars=\\\{\}]
usage: ated.py [\PYGZhy{}h] [fn]

Atomic lines file editor

positional arguments:
  fn          atoms file name (default: atoms.dat)

optional arguments:
  \PYGZhy{}h, \PYGZhy{}\PYGZhy{}help  show this help message and exit
\end{sphinxVerbatim}

This script belongs to package \sphinxstyleemphasis{f311.explorer}


\section{Script \sphinxstyleliteralintitle{cubeed.py}}
\label{\detokenize{autoscripts/script-cubeed:script-cubeed-py}}\label{\detokenize{autoscripts/script-cubeed::doc}}
\begin{sphinxVerbatim}[commandchars=\\\{\}]
usage: cubeed.py [\PYGZhy{}h] [fn]

Data Cube Editor, import/export WebSim\PYGZhy{}COMPASS data cubes

positional arguments:
  fn          file name, supports \PYGZsq{}FITS Sparse Data Cube (storage to take less
              disk space)\PYGZsq{} and \PYGZsq{}\PYGZsq{} (default: None)

optional arguments:
  \PYGZhy{}h, \PYGZhy{}\PYGZhy{}help  show this help message and exit
\end{sphinxVerbatim}

This script belongs to package \sphinxstyleemphasis{f311.explorer}


\section{Script \sphinxstyleliteralintitle{explorer.py}}
\label{\detokenize{autoscripts/script-explorer::doc}}\label{\detokenize{autoscripts/script-explorer:script-explorer-py}}
\begin{sphinxVerbatim}[commandchars=\\\{\}]
usage: explorer.py [\PYGZhy{}h] [dir]

F311 Explorer \PYGZhy{}\PYGZhy{}  list, visualize, and edit data files (\PYGZus{}à la\PYGZus{} File Manager)

positional arguments:
  dir         directory name (default: .)

optional arguments:
  \PYGZhy{}h, \PYGZhy{}\PYGZhy{}help  show this help message and exit
\end{sphinxVerbatim}

This script belongs to package \sphinxstyleemphasis{f311.explorer}


\section{Script \sphinxstyleliteralintitle{mained.py}}
\label{\detokenize{autoscripts/script-mained:script-mained-py}}\label{\detokenize{autoscripts/script-mained::doc}}
\begin{sphinxVerbatim}[commandchars=\\\{\}]
usage: mained.py [\PYGZhy{}h] [fn]

Main configuration file editor.

positional arguments:
  fn          main configuration file name (default: main.dat)

optional arguments:
  \PYGZhy{}h, \PYGZhy{}\PYGZhy{}help  show this help message and exit
\end{sphinxVerbatim}

This script belongs to package \sphinxstyleemphasis{f311.explorer}


\section{Script \sphinxstyleliteralintitle{mled.py}}
\label{\detokenize{autoscripts/script-mled:script-mled-py}}\label{\detokenize{autoscripts/script-mled::doc}}
\begin{sphinxVerbatim}[commandchars=\\\{\}]
usage: mled.py [\PYGZhy{}h] [fn]

Molecular lines file editor.

positional arguments:
  fn          molecules file name (default: molecules.dat)

optional arguments:
  \PYGZhy{}h, \PYGZhy{}\PYGZhy{}help  show this help message and exit
\end{sphinxVerbatim}

This script belongs to package \sphinxstyleemphasis{f311.explorer}


\section{Script \sphinxstyleliteralintitle{optionsed.py}}
\label{\detokenize{autoscripts/script-optionsed:script-optionsed-py}}\label{\detokenize{autoscripts/script-optionsed::doc}}
\begin{sphinxVerbatim}[commandchars=\\\{\}]
usage: optionsed.py [\PYGZhy{}h] [fn]

PFANT command\PYGZhy{}line options file editor.

positional arguments:
  fn          PFANT Command\PYGZhy{}line Options file name (default: options.py)

optional arguments:
  \PYGZhy{}h, \PYGZhy{}\PYGZhy{}help  show this help message and exit
\end{sphinxVerbatim}

This script belongs to package \sphinxstyleemphasis{f311.explorer}


\section{Script \sphinxstyleliteralintitle{splisted.py}}
\label{\detokenize{autoscripts/script-splisted:script-splisted-py}}\label{\detokenize{autoscripts/script-splisted::doc}}
\begin{sphinxVerbatim}[commandchars=\\\{\}]
usage: splisted.py [\PYGZhy{}h] [fn]

Spectrum List Editor

positional arguments:
  fn          file name, supports \PYGZsq{}FITS Spectrum List\PYGZsq{} only at the moment
              (default: None)

optional arguments:
  \PYGZhy{}h, \PYGZhy{}\PYGZhy{}help  show this help message and exit
\end{sphinxVerbatim}

This script belongs to package \sphinxstyleemphasis{f311.explorer}


\section{Script \sphinxstyleliteralintitle{tune-zinf.py}}
\label{\detokenize{autoscripts/script-tune-zinf::doc}}\label{\detokenize{autoscripts/script-tune-zinf:script-tune-zinf-py}}
\begin{sphinxVerbatim}[commandchars=\\\{\}]
usage: tune\PYGZhy{}zinf.py [\PYGZhy{}h] [\PYGZhy{}\PYGZhy{}min [MIN]] [\PYGZhy{}\PYGZhy{}max [MAX]] [\PYGZhy{}\PYGZhy{}inflate [INFLATE]]
                    [\PYGZhy{}\PYGZhy{}ge\PYGZus{}current] [\PYGZhy{}\PYGZhy{}no\PYGZus{}clean]
                    fn\PYGZus{}input [fn\PYGZus{}output]

Tunes the \PYGZdq{}zinf\PYGZdq{} parameter for each atomic line in atomic lines file

The \PYGZdq{}zinf\PYGZdq{} parameter is a distance in angstrom from the centre of an atomic
line. It specifies the calculation range for the line:
[centre\PYGZhy{}zinf, centre+zinf].

This script runs pfant for each atomic line to determine the width of each
atomic line and thus zinf.

Note: pfant is run using most of its default settings and will require the
following files to exist in the current directory:
  \PYGZhy{} main.dat
  \PYGZhy{} dissoc.dat
  \PYGZhy{} abonds.dat
  \PYGZhy{} modeles.mod
  \PYGZhy{} partit.dat
  \PYGZhy{} absoru2.dat

Note: the precision in the zinf found depends on the calculation step (\PYGZdq{}pas\PYGZdq{})
specified in main.dat. A higher \PYGZdq{}pas\PYGZdq{} means lower precision and a tendency to
get higher zinf\PYGZsq{}s. This is really not critical. pas=0.02 or pas=0.04 should do.

positional arguments:
  fn\PYGZus{}input             input file name
  fn\PYGZus{}output            output file name (default: \PYGZlt{}made\PYGZhy{}up filename\PYGZgt{})

optional arguments:
  \PYGZhy{}h, \PYGZhy{}\PYGZhy{}help           show this help message and exit
  \PYGZhy{}\PYGZhy{}min [MIN]          minimum zinf. If zinf found for a particular line is
                       smaller than this value, this value will be used
                       instead (default: 0.1)
  \PYGZhy{}\PYGZhy{}max [MAX]          maximum zinf. If zinf found for a particular line is
                       greater than this value, this value will be used
                       instead (default: 50.0)
  \PYGZhy{}\PYGZhy{}inflate [INFLATE]  Multiplicative constant to apply a \PYGZdq{}safety margin\PYGZdq{}.
                       Each zinf found will be multiplied by this value. For
                       example a value of INFLATE=1.1 means that all the
                       zinf\PYGZsq{}s saved will be 10 percent larger than those
                       calculated (default: 1.1)
  \PYGZhy{}\PYGZhy{}ge\PYGZus{}current         \PYGZdq{}Greater or Equal to current\PYGZdq{}: If this option is set,
                       the current zinf in the atomic lines file is used as a
                       lower boundary. (default: False)
  \PYGZhy{}\PYGZhy{}no\PYGZus{}clean           If set, will not remove the session directories.
                       (default: False)
\end{sphinxVerbatim}

This script belongs to package \sphinxstyleemphasis{f311.explorer}


\section{Script \sphinxstyleliteralintitle{copy-star.py}}
\label{\detokenize{autoscripts/script-copy-star::doc}}\label{\detokenize{autoscripts/script-copy-star:script-copy-star-py}}
\begin{sphinxVerbatim}[commandchars=\\\{\}]
usage: copy\PYGZhy{}star.py [\PYGZhy{}h] [\PYGZhy{}l] [\PYGZhy{}p] [directory]

Copies stellar data files (such as main.dat, abonds.dat, dissoc.dat) to local directory

examples of usage:
  \PYGZgt{} copy\PYGZhy{}star.py
  (displays menu)

  \PYGZgt{} copy\PYGZhy{}star.py arcturus
  (\PYGZdq{}arcturus\PYGZdq{} is the name of a subdirectory of PFANT/data)

  \PYGZgt{} copy\PYGZhy{}star.py \PYGZhy{}p /home/user/pfant\PYGZhy{}common\PYGZhy{}data
  (use option \PYGZdq{}\PYGZhy{}p\PYGZdq{} to specify path)

  \PYGZgt{} copy\PYGZhy{}star.py \PYGZhy{}l
  (lists subdirectories of PFANT/data , doesn\PYGZsq{}t copy anything)

positional arguments:
  directory   name of directory (either a subdirectory of PFANT/data or the
              path to a valid system directory (see modes of operation)
              (default: None)

optional arguments:
  \PYGZhy{}h, \PYGZhy{}\PYGZhy{}help  show this help message and exit
  \PYGZhy{}l, \PYGZhy{}\PYGZhy{}list  lists subdirectories of
              /home/j/Documents/projects/astro/github/PFANT/code/data
              (default: False)
  \PYGZhy{}p, \PYGZhy{}\PYGZhy{}path  system path mode (default: False)
\end{sphinxVerbatim}

This script belongs to package \sphinxstyleemphasis{f311.pyfant}


\section{Script \sphinxstyleliteralintitle{link.py}}
\label{\detokenize{autoscripts/script-link:script-link-py}}\label{\detokenize{autoscripts/script-link::doc}}
\begin{sphinxVerbatim}[commandchars=\\\{\}]
usage: link.py [\PYGZhy{}h] [\PYGZhy{}l] [\PYGZhy{}p] [\PYGZhy{}y] [directory]

Creates symbolic links to PFANT data files as an alternative to copying these (sometimes large) files into local directory

A star is specified by three data files whose typical names are:
main.dat, abonds.dat, and dissoc.dat .

The other data files (atomic/molecular lines, partition function, etc.)
are star\PYGZhy{}independent, and this script is a proposed solution to keep you from
copying these files for every new case.

How it works: link.py will look inside a given directory and create
symbolic links to files *.dat and *.mod.

The following files will be skipped:
  \PYGZhy{} main files, e.g. \PYGZdq{}main.dat\PYGZdq{}
  \PYGZhy{} dissoc files, e.g., \PYGZdq{}dissoc.dat\PYGZdq{}
  \PYGZhy{} abonds files, e.g., \PYGZdq{}abonds.dat\PYGZdq{}
  \PYGZhy{} .mod files with a single model inside, e.g., \PYGZdq{}modeles.mod\PYGZdq{}
  \PYGZhy{} hydrogen lines files, e.g., \PYGZdq{}thalpha\PYGZdq{}, \PYGZdq{}thbeta\PYGZdq{}

This script works in two different modes:

a) default mode: looks for files in a subdirectory of PFANT/data
   \PYGZgt{} link.py common
   (will create links to filess inside PFANT/data/common)

b) \PYGZdq{}\PYGZhy{}l\PYGZdq{} option: lists subdirectories of PFANT/data

c) \PYGZdq{}\PYGZhy{}p\PYGZdq{} option: looks for files in a directory specified.
   Examples:
   \PYGZgt{} link.py \PYGZhy{}p /home/user/pfant\PYGZhy{}common\PYGZhy{}data
   \PYGZgt{} link.py \PYGZhy{}p ../../pfant\PYGZhy{}common\PYGZhy{}data

Note: in Windows, this script must be run as administrator.

positional arguments:
  directory   name of directory (either a subdirectory of PFANT/data or the
              path to a valid system directory (see modes of operation)
              (default: common)

optional arguments:
  \PYGZhy{}h, \PYGZhy{}\PYGZhy{}help  show this help message and exit
  \PYGZhy{}l, \PYGZhy{}\PYGZhy{}list  lists subdirectories of
              /home/j/Documents/projects/astro/github/PFANT/code/data
              (default: False)
  \PYGZhy{}p, \PYGZhy{}\PYGZhy{}path  system path mode (default: False)
  \PYGZhy{}y, \PYGZhy{}\PYGZhy{}yes   Automatically answers \PYGZsq{}yes\PYGZsq{} to eventual question (default:
              False)
\end{sphinxVerbatim}

This script belongs to package \sphinxstyleemphasis{f311.pyfant}


\section{Script \sphinxstyleliteralintitle{merge-molecules.py}}
\label{\detokenize{autoscripts/script-merge-molecules:script-merge-molecules-py}}\label{\detokenize{autoscripts/script-merge-molecules::doc}}
\begin{sphinxVerbatim}[commandchars=\\\{\}]
usage: merge\PYGZhy{}molecules.py [\PYGZhy{}h] [\PYGZhy{}o [FN\PYGZus{}OUTPUT]] files [files ...]

Merges several PFANT molecular lines file into a single one

positional arguments:
  files                 files specification: list of files, wildcards allowed

optional arguments:
  \PYGZhy{}h, \PYGZhy{}\PYGZhy{}help            show this help message and exit
  \PYGZhy{}o [FN\PYGZus{}OUTPUT], \PYGZhy{}\PYGZhy{}fn\PYGZus{}output [FN\PYGZus{}OUTPUT]
                        output filename. If not specified, creates file such
                        as \PYGZsq{}molecules\PYGZhy{}merged\PYGZhy{}.0000.dat\PYGZsq{} (default: (automatic))
\end{sphinxVerbatim}

This script belongs to package \sphinxstyleemphasis{f311.pyfant}


\section{Script \sphinxstyleliteralintitle{run-multi.py}}
\label{\detokenize{autoscripts/script-run-multi:script-run-multi-py}}\label{\detokenize{autoscripts/script-run-multi::doc}}
\begin{sphinxVerbatim}[commandchars=\\\{\}]
usage: run\PYGZhy{}multi.py [\PYGZhy{}h] [\PYGZhy{}\PYGZhy{}abs ABS] [\PYGZhy{}\PYGZhy{}absoru ABSORU] [\PYGZhy{}\PYGZhy{}aint AINT]
                    [\PYGZhy{}\PYGZhy{}allow ALLOW] [\PYGZhy{}\PYGZhy{}amores AMORES] [\PYGZhy{}\PYGZhy{}convol CONVOL]
                    [\PYGZhy{}\PYGZhy{}explain EXPLAIN] [\PYGZhy{}\PYGZhy{}flam FLAM] [\PYGZhy{}\PYGZhy{}flprefix FLPREFIX]
                    [\PYGZhy{}\PYGZhy{}fn\PYGZus{}abonds FN\PYGZus{}ABONDS] [\PYGZhy{}\PYGZhy{}fn\PYGZus{}absoru2 FN\PYGZus{}ABSORU2]
                    [\PYGZhy{}\PYGZhy{}fn\PYGZus{}atoms FN\PYGZus{}ATOMS] [\PYGZhy{}\PYGZhy{}fn\PYGZus{}cv FN\PYGZus{}CV]
                    [\PYGZhy{}\PYGZhy{}fn\PYGZus{}dissoc FN\PYGZus{}DISSOC] [\PYGZhy{}\PYGZhy{}fn\PYGZus{}flux FN\PYGZus{}FLUX]
                    [\PYGZhy{}\PYGZhy{}fn\PYGZus{}hmap FN\PYGZus{}HMAP] [\PYGZhy{}\PYGZhy{}fn\PYGZus{}lines FN\PYGZus{}LINES]
                    [\PYGZhy{}\PYGZhy{}fn\PYGZus{}log FN\PYGZus{}LOG] [\PYGZhy{}\PYGZhy{}fn\PYGZus{}logging FN\PYGZus{}LOGGING]
                    [\PYGZhy{}\PYGZhy{}fn\PYGZus{}main FN\PYGZus{}MAIN] [\PYGZhy{}\PYGZhy{}fn\PYGZus{}modeles FN\PYGZus{}MODELES]
                    [\PYGZhy{}\PYGZhy{}fn\PYGZus{}modgrid FN\PYGZus{}MODGRID] [\PYGZhy{}\PYGZhy{}fn\PYGZus{}molecules FN\PYGZus{}MOLECULES]
                    [\PYGZhy{}\PYGZhy{}fn\PYGZus{}moo FN\PYGZus{}MOO] [\PYGZhy{}\PYGZhy{}fn\PYGZus{}opa FN\PYGZus{}OPA]
                    [\PYGZhy{}\PYGZhy{}fn\PYGZus{}partit FN\PYGZus{}PARTIT] [\PYGZhy{}\PYGZhy{}fn\PYGZus{}progress FN\PYGZus{}PROGRESS]
                    [\PYGZhy{}\PYGZhy{}fwhm FWHM] [\PYGZhy{}\PYGZhy{}interp INTERP] [\PYGZhy{}\PYGZhy{}kik KIK] [\PYGZhy{}\PYGZhy{}kq KQ]
                    [\PYGZhy{}\PYGZhy{}llfin LLFIN] [\PYGZhy{}\PYGZhy{}llzero LLZERO]
                    [\PYGZhy{}\PYGZhy{}logging\PYGZus{}console LOGGING\PYGZus{}CONSOLE]
                    [\PYGZhy{}\PYGZhy{}logging\PYGZus{}file LOGGING\PYGZus{}FILE]
                    [\PYGZhy{}\PYGZhy{}logging\PYGZus{}level LOGGING\PYGZus{}LEVEL] [\PYGZhy{}\PYGZhy{}no\PYGZus{}atoms NO\PYGZus{}ATOMS]
                    [\PYGZhy{}\PYGZhy{}no\PYGZus{}h NO\PYGZus{}H] [\PYGZhy{}\PYGZhy{}no\PYGZus{}molecules NO\PYGZus{}MOLECULES] [\PYGZhy{}\PYGZhy{}norm NORM]
                    [\PYGZhy{}\PYGZhy{}opa OPA] [\PYGZhy{}\PYGZhy{}pas PAS] [\PYGZhy{}\PYGZhy{}pat PAT] [\PYGZhy{}\PYGZhy{}play PLAY]
                    [\PYGZhy{}\PYGZhy{}sca SCA] [\PYGZhy{}\PYGZhy{}zinf ZINF] [\PYGZhy{}\PYGZhy{}zph ZPH] [\PYGZhy{}f FN\PYGZus{}ABXFWHM]
                    [\PYGZhy{}s CUSTOM\PYGZus{}SESSION\PYGZus{}ID]

Runs pfant and nulbad in \PYGZdq{}multi mode\PYGZdq{} (equivalent to Tab 4 in {}`{}`x.py{}`{}`) (several abundances X FWHM\PYGZsq{}s)

optional arguments:
  \PYGZhy{}h, \PYGZhy{}\PYGZhy{}help            show this help message and exit
  \PYGZhy{}\PYGZhy{}abs ABS
  \PYGZhy{}\PYGZhy{}absoru ABSORU
  \PYGZhy{}\PYGZhy{}aint AINT
  \PYGZhy{}\PYGZhy{}allow ALLOW
  \PYGZhy{}\PYGZhy{}amores AMORES
  \PYGZhy{}\PYGZhy{}convol CONVOL
  \PYGZhy{}\PYGZhy{}explain EXPLAIN
  \PYGZhy{}\PYGZhy{}flam FLAM
  \PYGZhy{}\PYGZhy{}flprefix FLPREFIX
  \PYGZhy{}\PYGZhy{}fn\PYGZus{}abonds FN\PYGZus{}ABONDS
  \PYGZhy{}\PYGZhy{}fn\PYGZus{}absoru2 FN\PYGZus{}ABSORU2
  \PYGZhy{}\PYGZhy{}fn\PYGZus{}atoms FN\PYGZus{}ATOMS
  \PYGZhy{}\PYGZhy{}fn\PYGZus{}cv FN\PYGZus{}CV
  \PYGZhy{}\PYGZhy{}fn\PYGZus{}dissoc FN\PYGZus{}DISSOC
  \PYGZhy{}\PYGZhy{}fn\PYGZus{}flux FN\PYGZus{}FLUX
  \PYGZhy{}\PYGZhy{}fn\PYGZus{}hmap FN\PYGZus{}HMAP
  \PYGZhy{}\PYGZhy{}fn\PYGZus{}lines FN\PYGZus{}LINES
  \PYGZhy{}\PYGZhy{}fn\PYGZus{}log FN\PYGZus{}LOG
  \PYGZhy{}\PYGZhy{}fn\PYGZus{}logging FN\PYGZus{}LOGGING
  \PYGZhy{}\PYGZhy{}fn\PYGZus{}main FN\PYGZus{}MAIN
  \PYGZhy{}\PYGZhy{}fn\PYGZus{}modeles FN\PYGZus{}MODELES
  \PYGZhy{}\PYGZhy{}fn\PYGZus{}modgrid FN\PYGZus{}MODGRID
  \PYGZhy{}\PYGZhy{}fn\PYGZus{}molecules FN\PYGZus{}MOLECULES
  \PYGZhy{}\PYGZhy{}fn\PYGZus{}moo FN\PYGZus{}MOO
  \PYGZhy{}\PYGZhy{}fn\PYGZus{}opa FN\PYGZus{}OPA
  \PYGZhy{}\PYGZhy{}fn\PYGZus{}partit FN\PYGZus{}PARTIT
  \PYGZhy{}\PYGZhy{}fn\PYGZus{}progress FN\PYGZus{}PROGRESS
  \PYGZhy{}\PYGZhy{}fwhm FWHM
  \PYGZhy{}\PYGZhy{}interp INTERP
  \PYGZhy{}\PYGZhy{}kik KIK
  \PYGZhy{}\PYGZhy{}kq KQ
  \PYGZhy{}\PYGZhy{}llfin LLFIN
  \PYGZhy{}\PYGZhy{}llzero LLZERO
  \PYGZhy{}\PYGZhy{}logging\PYGZus{}console LOGGING\PYGZus{}CONSOLE
  \PYGZhy{}\PYGZhy{}logging\PYGZus{}file LOGGING\PYGZus{}FILE
  \PYGZhy{}\PYGZhy{}logging\PYGZus{}level LOGGING\PYGZus{}LEVEL
  \PYGZhy{}\PYGZhy{}no\PYGZus{}atoms NO\PYGZus{}ATOMS
  \PYGZhy{}\PYGZhy{}no\PYGZus{}h NO\PYGZus{}H
  \PYGZhy{}\PYGZhy{}no\PYGZus{}molecules NO\PYGZus{}MOLECULES
  \PYGZhy{}\PYGZhy{}norm NORM
  \PYGZhy{}\PYGZhy{}opa OPA
  \PYGZhy{}\PYGZhy{}pas PAS
  \PYGZhy{}\PYGZhy{}pat PAT
  \PYGZhy{}\PYGZhy{}play PLAY
  \PYGZhy{}\PYGZhy{}sca SCA
  \PYGZhy{}\PYGZhy{}zinf ZINF
  \PYGZhy{}\PYGZhy{}zph ZPH
  \PYGZhy{}f FN\PYGZus{}ABXFWHM, \PYGZhy{}\PYGZhy{}fn\PYGZus{}abxfwhm FN\PYGZus{}ABXFWHM
                        Name of file specifying different abundances and
                        FWHM\PYGZsq{}s (default: abxfwhm.py)
  \PYGZhy{}s CUSTOM\PYGZus{}SESSION\PYGZus{}ID, \PYGZhy{}\PYGZhy{}custom\PYGZus{}session\PYGZus{}id CUSTOM\PYGZus{}SESSION\PYGZus{}ID
                        Name of directory where output files will be saved
                        (default: multi\PYGZhy{}session\PYGZhy{}\PYGZlt{}i\PYGZgt{})
\end{sphinxVerbatim}

This script belongs to package \sphinxstyleemphasis{f311.pyfant}


\section{Script \sphinxstyleliteralintitle{run4.py}}
\label{\detokenize{autoscripts/script-run4::doc}}\label{\detokenize{autoscripts/script-run4:script-run4-py}}
\begin{sphinxVerbatim}[commandchars=\\\{\}]
usage: run4.py [\PYGZhy{}h] [\PYGZhy{}\PYGZhy{}abs ABS] [\PYGZhy{}\PYGZhy{}absoru ABSORU] [\PYGZhy{}\PYGZhy{}aint AINT]
               [\PYGZhy{}\PYGZhy{}allow ALLOW] [\PYGZhy{}\PYGZhy{}amores AMORES] [\PYGZhy{}\PYGZhy{}convol CONVOL]
               [\PYGZhy{}\PYGZhy{}explain EXPLAIN] [\PYGZhy{}\PYGZhy{}flam FLAM] [\PYGZhy{}\PYGZhy{}flprefix FLPREFIX]
               [\PYGZhy{}\PYGZhy{}fn\PYGZus{}abonds FN\PYGZus{}ABONDS] [\PYGZhy{}\PYGZhy{}fn\PYGZus{}absoru2 FN\PYGZus{}ABSORU2]
               [\PYGZhy{}\PYGZhy{}fn\PYGZus{}atoms FN\PYGZus{}ATOMS] [\PYGZhy{}\PYGZhy{}fn\PYGZus{}cv FN\PYGZus{}CV] [\PYGZhy{}\PYGZhy{}fn\PYGZus{}dissoc FN\PYGZus{}DISSOC]
               [\PYGZhy{}\PYGZhy{}fn\PYGZus{}flux FN\PYGZus{}FLUX] [\PYGZhy{}\PYGZhy{}fn\PYGZus{}hmap FN\PYGZus{}HMAP] [\PYGZhy{}\PYGZhy{}fn\PYGZus{}lines FN\PYGZus{}LINES]
               [\PYGZhy{}\PYGZhy{}fn\PYGZus{}log FN\PYGZus{}LOG] [\PYGZhy{}\PYGZhy{}fn\PYGZus{}logging FN\PYGZus{}LOGGING] [\PYGZhy{}\PYGZhy{}fn\PYGZus{}main FN\PYGZus{}MAIN]
               [\PYGZhy{}\PYGZhy{}fn\PYGZus{}modeles FN\PYGZus{}MODELES] [\PYGZhy{}\PYGZhy{}fn\PYGZus{}modgrid FN\PYGZus{}MODGRID]
               [\PYGZhy{}\PYGZhy{}fn\PYGZus{}molecules FN\PYGZus{}MOLECULES] [\PYGZhy{}\PYGZhy{}fn\PYGZus{}moo FN\PYGZus{}MOO]
               [\PYGZhy{}\PYGZhy{}fn\PYGZus{}opa FN\PYGZus{}OPA] [\PYGZhy{}\PYGZhy{}fn\PYGZus{}partit FN\PYGZus{}PARTIT]
               [\PYGZhy{}\PYGZhy{}fn\PYGZus{}progress FN\PYGZus{}PROGRESS] [\PYGZhy{}\PYGZhy{}fwhm FWHM] [\PYGZhy{}\PYGZhy{}interp INTERP]
               [\PYGZhy{}\PYGZhy{}kik KIK] [\PYGZhy{}\PYGZhy{}kq KQ] [\PYGZhy{}\PYGZhy{}llfin LLFIN] [\PYGZhy{}\PYGZhy{}llzero LLZERO]
               [\PYGZhy{}\PYGZhy{}logging\PYGZus{}console LOGGING\PYGZus{}CONSOLE]
               [\PYGZhy{}\PYGZhy{}logging\PYGZus{}file LOGGING\PYGZus{}FILE] [\PYGZhy{}\PYGZhy{}logging\PYGZus{}level LOGGING\PYGZus{}LEVEL]
               [\PYGZhy{}\PYGZhy{}no\PYGZus{}atoms NO\PYGZus{}ATOMS] [\PYGZhy{}\PYGZhy{}no\PYGZus{}h NO\PYGZus{}H]
               [\PYGZhy{}\PYGZhy{}no\PYGZus{}molecules NO\PYGZus{}MOLECULES] [\PYGZhy{}\PYGZhy{}norm NORM] [\PYGZhy{}\PYGZhy{}opa OPA]
               [\PYGZhy{}\PYGZhy{}pas PAS] [\PYGZhy{}\PYGZhy{}pat PAT] [\PYGZhy{}\PYGZhy{}play PLAY] [\PYGZhy{}\PYGZhy{}sca SCA] [\PYGZhy{}\PYGZhy{}zinf ZINF]
               [\PYGZhy{}\PYGZhy{}zph ZPH]

Runs the four Fortran binaries in sequence: {}`innewmarcs{}`, {}`hydro2{}`, {}`pfant{}`, {}`nulbad{}`

Check session directory \PYGZdq{}session\PYGZhy{}\PYGZlt{}number\PYGZgt{}\PYGZdq{} for log files.

optional arguments:
  \PYGZhy{}h, \PYGZhy{}\PYGZhy{}help            show this help message and exit
  \PYGZhy{}\PYGZhy{}abs ABS
  \PYGZhy{}\PYGZhy{}absoru ABSORU
  \PYGZhy{}\PYGZhy{}aint AINT
  \PYGZhy{}\PYGZhy{}allow ALLOW
  \PYGZhy{}\PYGZhy{}amores AMORES
  \PYGZhy{}\PYGZhy{}convol CONVOL
  \PYGZhy{}\PYGZhy{}explain EXPLAIN
  \PYGZhy{}\PYGZhy{}flam FLAM
  \PYGZhy{}\PYGZhy{}flprefix FLPREFIX
  \PYGZhy{}\PYGZhy{}fn\PYGZus{}abonds FN\PYGZus{}ABONDS
  \PYGZhy{}\PYGZhy{}fn\PYGZus{}absoru2 FN\PYGZus{}ABSORU2
  \PYGZhy{}\PYGZhy{}fn\PYGZus{}atoms FN\PYGZus{}ATOMS
  \PYGZhy{}\PYGZhy{}fn\PYGZus{}cv FN\PYGZus{}CV
  \PYGZhy{}\PYGZhy{}fn\PYGZus{}dissoc FN\PYGZus{}DISSOC
  \PYGZhy{}\PYGZhy{}fn\PYGZus{}flux FN\PYGZus{}FLUX
  \PYGZhy{}\PYGZhy{}fn\PYGZus{}hmap FN\PYGZus{}HMAP
  \PYGZhy{}\PYGZhy{}fn\PYGZus{}lines FN\PYGZus{}LINES
  \PYGZhy{}\PYGZhy{}fn\PYGZus{}log FN\PYGZus{}LOG
  \PYGZhy{}\PYGZhy{}fn\PYGZus{}logging FN\PYGZus{}LOGGING
  \PYGZhy{}\PYGZhy{}fn\PYGZus{}main FN\PYGZus{}MAIN
  \PYGZhy{}\PYGZhy{}fn\PYGZus{}modeles FN\PYGZus{}MODELES
  \PYGZhy{}\PYGZhy{}fn\PYGZus{}modgrid FN\PYGZus{}MODGRID
  \PYGZhy{}\PYGZhy{}fn\PYGZus{}molecules FN\PYGZus{}MOLECULES
  \PYGZhy{}\PYGZhy{}fn\PYGZus{}moo FN\PYGZus{}MOO
  \PYGZhy{}\PYGZhy{}fn\PYGZus{}opa FN\PYGZus{}OPA
  \PYGZhy{}\PYGZhy{}fn\PYGZus{}partit FN\PYGZus{}PARTIT
  \PYGZhy{}\PYGZhy{}fn\PYGZus{}progress FN\PYGZus{}PROGRESS
  \PYGZhy{}\PYGZhy{}fwhm FWHM
  \PYGZhy{}\PYGZhy{}interp INTERP
  \PYGZhy{}\PYGZhy{}kik KIK
  \PYGZhy{}\PYGZhy{}kq KQ
  \PYGZhy{}\PYGZhy{}llfin LLFIN
  \PYGZhy{}\PYGZhy{}llzero LLZERO
  \PYGZhy{}\PYGZhy{}logging\PYGZus{}console LOGGING\PYGZus{}CONSOLE
  \PYGZhy{}\PYGZhy{}logging\PYGZus{}file LOGGING\PYGZus{}FILE
  \PYGZhy{}\PYGZhy{}logging\PYGZus{}level LOGGING\PYGZus{}LEVEL
  \PYGZhy{}\PYGZhy{}no\PYGZus{}atoms NO\PYGZus{}ATOMS
  \PYGZhy{}\PYGZhy{}no\PYGZus{}h NO\PYGZus{}H
  \PYGZhy{}\PYGZhy{}no\PYGZus{}molecules NO\PYGZus{}MOLECULES
  \PYGZhy{}\PYGZhy{}norm NORM
  \PYGZhy{}\PYGZhy{}opa OPA
  \PYGZhy{}\PYGZhy{}pas PAS
  \PYGZhy{}\PYGZhy{}pat PAT
  \PYGZhy{}\PYGZhy{}play PLAY
  \PYGZhy{}\PYGZhy{}sca SCA
  \PYGZhy{}\PYGZhy{}zinf ZINF
  \PYGZhy{}\PYGZhy{}zph ZPH
\end{sphinxVerbatim}

This script belongs to package \sphinxstyleemphasis{f311.pyfant}


\section{Script \sphinxstyleliteralintitle{x.py}}
\label{\detokenize{autoscripts/script-x::doc}}\label{\detokenize{autoscripts/script-x:script-x-py}}
\begin{sphinxVerbatim}[commandchars=\\\{\}]
usage: x.py [\PYGZhy{}h]

PFANT Launcher \PYGZhy{}\PYGZhy{} Graphical Interface for Spectral Synthesis

Single and multi modes.

Multi mode
\PYGZhy{}\PYGZhy{}\PYGZhy{}\PYGZhy{}\PYGZhy{}\PYGZhy{}\PYGZhy{}\PYGZhy{}\PYGZhy{}\PYGZhy{}

Runs pfant for different abundances for each element, then run nulbad for each
pfant result for different FWHMs.

The configuration is read from a .py file.

The user must specify a list of FWHM values for nulbad convolutions, and
a dictionary containing element symbols and respective list containing n\PYGZus{}abdif
differential abundances to be used for each element.

pfant will be run n\PYGZus{}abdif times, each time adding to each element in ab the i\PYGZhy{}th
value in the vector for the corresponding element.

nulbad will run n\PYGZus{}abdif*n\PYGZus{}fwhms times, where n\PYGZus{}fwhms is the number of different
FWHMs specified.

The result will be
\PYGZhy{} several spectra saved as  \PYGZdq{}\PYGZlt{}star name\PYGZgt{}\PYGZlt{}pfant name or counter\PYGZgt{}.sp\PYGZdq{}
\PYGZhy{} several \PYGZdq{}spectra list\PYGZdq{} files saved as \PYGZdq{}cv\PYGZus{}\PYGZlt{}FWHM\PYGZgt{}.spl\PYGZdq{}. As the file indicates,
  each \PYGZdq{}.spl\PYGZdq{} file will have the names of the spectrum files for a specific FWHM.
  .spl files are subject to input for lineplot.py by E.Cantelli
\PYGZhy{}\PYGZhy{}\PYGZhy{}\PYGZhy{}\PYGZhy{}\PYGZhy{}\PYGZhy{}\PYGZhy{}\PYGZhy{}

optional arguments:
  \PYGZhy{}h, \PYGZhy{}\PYGZhy{}help  show this help message and exit
\end{sphinxVerbatim}

This script belongs to package \sphinxstyleemphasis{f311.pyfant}



\renewcommand{\indexname}{Index}
\printindex
\end{document}